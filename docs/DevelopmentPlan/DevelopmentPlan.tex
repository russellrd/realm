\documentclass{article}

\usepackage{booktabs}
\usepackage{tabularx}

\title{Development Plan\\\progname}

\author{\authname}

\date{}

\input{../Comments}
%% Common Parts

\newcommand{\progname}{Software Engineering} % PUT YOUR PROGRAM NAME HERE
\newcommand{\authname}{Team \#13, ARC
    \\ Ahluwalia, Avanish
    \\ Davidson, Russell
    \\ Malik, Rafey
    \\ Zulfiqar, Abdul} % AUTHOR NAMES                  

\usepackage{hyperref}
    \hypersetup{colorlinks=true, linkcolor=blue, citecolor=blue, filecolor=blue,
                urlcolor=blue, unicode=false}
    \urlstyle{same}
                                


\begin{document}

\maketitle

\begin{table}[hp]
\caption{Revision History} \label{TblRevisionHistory}
\begin{tabularx}{\textwidth}{llX}
\toprule
\textbf{Date} & \textbf{Developer(s)} & \textbf{Change}\\
\midrule
2024-09-21 & Zulfiqar Abdul & Expected Technology, Coding Standard\\
2024-09-23 & Zulfiqar Abdul & Proof of Concept Demonstration Plan\\
... & ... & ...\\
\bottomrule
\end{tabularx}
\end{table}

\newpage{}

\wss{Put your introductory blurb here.  Often the blurb is a brief roadmap of
what is contained in the report.}

\wss{Additional information on the development plan can be found in the
\href{https://gitlab.cas.mcmaster.ca/courses/capstone/-/blob/main/Lectures/L02b_POCAndDevPlan/POCAndDevPlan.pdf?ref_type=heads}
{lecture slides}.}

\section{Confidential Information?}

This project will not have any confidential information to protect.

\section{IP to Protect}

This project will not have any IP to protect.

\section{Copyright License}

\wss{What copyright license is your team adopting.  Point to the license in your
repo.}

\section{Team Meeting Plan}

\wss{How often will you meet? where?}

\wss{If the meeting is a physical location (not virtual), out of an abundance of
caution for safety reasons you shouldn't put the location online}

\wss{How often will you meet with your industry advisor?  when?  where?}

\wss{Will meetings be virtual?  At least some meetings should likely be
in-person.}

\wss{How will the meetings be structured?  There should be a chair for all meetings.  There should be an agenda for all meetings.}

\section{Team Communication Plan}

\wss{Issues on GitHub should be part of your communication plan.}

\section{Team Member Roles}

\wss{You should identify the types of roles you anticipate, like notetaker,
leader, meeting chair, reviewer.  Assigning specific people to those roles is
not necessary at this stage.  In a student team the role of the individuals will
likely change throughout the year.}

\section{Workflow Plan}

\begin{itemize}
	\item How will you be using git, including branches, pull request, etc.?
	\item How will you be managing issues, including template issues, issue
	classification, etc.?
  \item Use of CI/CD
\end{itemize}

\section{Project Decomposition and Scheduling}

\begin{itemize}
  \item How will you be using GitHub projects?
  \item Include a link to your GitHub project
\end{itemize}

\wss{How will the project be scheduled?  This is the big picture schedule, not
details. You will need to reproduce information that is in the course outline
for deadlines.}

\section{Proof of Concept Demonstration Plan}

One of the biggest uncertainties of the project is the achievability of 
the precise spatial positioning of AR objects required by the user experience we envision for our product. 
One of the core ideas of our product is to create context specific AR experiences; 
this will not be possible if we cannot precisely recreate AR objects where they were originally placed.\\

The other major uncertainty in this project is the feasibility of 
allowing users to scan in their own 3D objects using only camera and sensor data available on a smartphone. 
This feature will be an important avenue for user generated content on the platform, 
without which, general users will primarily be limited to prebuilt 3D models provided by the platform.\\

Our POC demonstration will require both of these features, 
as well as the minimum UI required to implement the following user experience:\\

\begin{enumerate}
  \item User scans object, obtaining a 3D model they can place
  \item User virtually places said object in their surroundings
  \item User leaves the area, returns, and can see the object they placed in the correct position
\end{enumerate}


\section{Expected Technology}

From our investigation into AR technologies, Unity with AR Foundation (an extension built on ARCore and ARKit) is the most promising option to meet the needs of our project. Those needs being:
\begin{itemize}
  \item Cross platform support
  \item Visual UI editor
  \item Integration with powerful AR libraries with capabilities such as
  \begin{itemize}
        \item Surface detection
        \item Lighting detection
        \item Motion tracking
        \item Depth measurement
      \end{itemize}
\end{itemize}


As will be using Unity, our primary programming language will be C\#.\\

The best testing tool to use will be the NUnit based Unity Test Framework due to its integration with unity, 
and NUnit's relative popularity and similarity to JUnit, 
which will make finding learning resources easy. 
This framework will provide both unit testing as well as code coverage metrics.\\

Application profiling will be done using Unity's built in profiler whenever necessary\\

Version control will be managed using git, 
and the remote repository will be hosted on GitHub as it is the industry standard, 
and has integration with CI and project management tools.\\

Github Projects will be used for project management, 
the details of which are outlined in section 8 Project Decomposition and Scheduling.\\

Continuous integration will be done via GitHub actions. 
Unity testing and builds will be achieved using the 
\href{https://github.com/marketplace/actions/unity-builder}{Unity Builder} 
and \href{https://github.com/marketplace/actions/unity-test-runner}{Unity Test Runner} 
github actions provided by the GameCI open source project as a base. 
CI builds will be triggered on pushes to the dev branch and PRs to the main branch 
where changes have been made in the source code directory.\\

The project will primarily be developed using the Unity Engine, 
with VSCode / VS as the preferred code editor to enforce code standards. \\


\section{Coding Standard}

All code must be formatted by the Microsoft C\# formatter with default settings, 
and all suggestions from the Microsoft C\# linter must be followed. 
In general we will follow \href{https://learn.microsoft.com/en-us/dotnet/csharp/fundamentals/coding-style/coding-conventions}{Microsoft C\# code standards.}

\newpage{}

\section*{Appendix --- Reflection}

\wss{Not required for CAS 741}

\input{../Reflection.tex}

\begin{enumerate}
    \item Why is it important to create a development plan prior to starting the
    project?
    \item In your opinion, what are the advantages and disadvantages of using
    CI/CD?
    \item What disagreements did your group have in this deliverable, if any,
    and how did you resolve them?
\end{enumerate}

\newpage{}

\section*{Appendix --- Team Charter}

\wss{borrows from
\href{https://engineering.up.edu/industry_partnerships/files/team-charter.pdf}
{University of Portland Team Charter}}

\subsection*{External Goals}

\wss{What are your team's external goals for this project? These are not the
goals related to the functionality or quality fo the project.  These are the
goals on what the team wishes to achieve with the project.  Potential goals are
to win a prize at the Capstone EXPO, or to have something to talk about in
interviews, or to get an A+, etc.}

\subsection*{Attendance}

\subsubsection*{Expectations}

\wss{What are your team's expectations regarding meeting attendance (being on
time, leaving early, missing meetings, etc.)?}

\subsubsection*{Acceptable Excuse}

\wss{What constitutes an acceptable excuse for missing a meeting or a deadline?
What types of excuses will not be considered acceptable?}

\subsubsection*{In Case of Emergency}

\wss{What process will team members follow if they have an emergency and cannot
attend a team meeting or complete their individual work promised for a team
deliverable?}

\subsection*{Accountability and Teamwork}

\subsubsection*{Quality} 

\wss{What are your team's expectations regarding the quality
of team members' preparation for team meetings and the quality of the
deliverables that members bring to the team?}

\subsubsection*{Attitude}

\wss{What are your team's expectations regarding team members' ideas,
interactions with the team, cooperation, attitudes, and anything else regarding
team member contributions?  Do you want to introduce a code of conduct?  Do you
want a conflict resolution plan?  Can adopt existing codes of conduct.}

\subsubsection*{Stay on Track}

\wss{What methods will be used to keep the team on track? How will your team
ensure that members contribute as expected to the team and that the team
performs as expected? How will your team reward members who do well and manage
members whose performance is below expectations?  What are the consequences for
someone not contributing their fair share?}

\wss{You may wish to use the project management metrics collected for the TA and
instructor for this.}

\wss{You can set target metrics for attendance, commits, etc.  What are the
consequences if someone doesn't hit their targets?  Do they need to bring the
coffee to the next team meeting?  Does the team need to make an appointment with
their TA, or the instructor?  Are there incentives for reaching targets early?}

\subsubsection*{Team Building}

\wss{How will you build team cohesion (fun time, group rituals, etc.)? }

\subsubsection*{Decision Making} 

\wss{How will you make decisions in your group? Consensus?  Vote? How will you
handle disagreements? }

\end{document}