\documentclass{article}

\usepackage{booktabs}
\usepackage{tabularx}
\usepackage[table, x11names]{xcolor}

\title{Development Plan\\\progname}

\author{\authname}

\date{}

\input{../Comments}
%% Common Parts

\newcommand{\progname}{Software Engineering} % PUT YOUR PROGRAM NAME HERE
\newcommand{\authname}{Team \#13, ARC
    \\ Ahluwalia, Avanish
    \\ Davidson, Russell
    \\ Malik, Rafey
    \\ Zulfiqar, Abdul} % AUTHOR NAMES                  

\usepackage{hyperref}
    \hypersetup{colorlinks=true, linkcolor=blue, citecolor=blue, filecolor=blue,
                urlcolor=blue, unicode=false}
    \urlstyle{same}
                                


\begin{document}

\maketitle

\begin{table}[hp]
\caption{Revision History} \label{TblRevisionHistory}
\begin{tabularx}{\textwidth}{llX}
\toprule
\textbf{Date} & \textbf{Developer(s)} & \textbf{Change}\\
\midrule
2024-09-16 & Russell Davidson & General ideas for Sections 4, 5, 6, 7, and 8\\
2024-09-18 & Russell Davidson & Finished section 4\\
2024-09-21 & Zulfiqar Abdul & Expected Technology, Coding Standard\\
2024-09-21 & Russell Davidson & Finished sections 6, and 7\\
2024-09-23 & Zulfiqar Abdul & Proof of Concept Demonstration Plan\\
2024-09-23 & Russell Davidson & Finished sections 3, 5, and 8\\
\bottomrule
\end{tabularx}
\end{table}

\newpage{}

\wss{Put your introductory blurb here.  Often the blurb is a brief roadmap of
what is contained in the report.}

\wss{Additional information on the development plan can be found in the
\href{https://gitlab.cas.mcmaster.ca/courses/capstone/-/blob/main/Lectures/L02b_POCAndDevPlan/POCAndDevPlan.pdf?ref_type=heads}
{lecture slides}.}

\section{Confidential Information?}

This project will not have any confidential information to protect.

\section{IP to Protect}

This project will not have any IP to protect.

\section{Copyright License}

The Apache License 2.0 will be adopted for this project. The license can be found in the repo \href{https://github.com/russellrd/realm/blob/main/LICENSE}{here}.

\section{Team Meeting Plan} \label{team_meeting_plan}

Every week, there will be two in-person team meetings that can be extended in length if necessary:
\begin{itemize}
    \item Mondays at 2:30pm-4:20pm
    \item Wednesdays 4:30pm-5:30pm
\end{itemize}

Before deliverables are due, additional meetings may be held virtually. \\

This team has no industry advisor so no meeting times are needed. If one is found in the future, meeting times will be scheduled according to their availability. \\

All meetings will start with the progress review of all action items from the previous meeting. Then, each item in the agenda section of the GitHub issue for the current meeting will be discussed. Notes will be added to the GitHub issue as topics are being discussed and action items will be made for each task that should be done before the next meeting. Before ending the meeting, a new GitHub issue for the next scheduled meeting will be created with an agenda. \\

Every meeting will have a chair starting with Rafey and rotating through members every week.

\section{Team Communication Plan} \label{team_communication_plan}

Discord will be used for online meetings and asynchronous discussions. All scheduled meetings will be conducted as described in the previous section (4 Team Meeting Plan). \\

Communication with the TA will be done through MS Teams. \\

GitHub issues will be created for the following purposes:

\begin{enumerate}
    \item \href{https://github.com/russellrd/realm/blob/main/.github/ISSUE_TEMPLATE/lect.md}{Lecture attendance}
    \begin{itemize}
        \item Each lecture will have a corresponding GitHub issue where the attendance of team members and relevant questions related to the lecture content is recorded.
    \end{itemize}
    \item \href{https://github.com/russellrd/realm/blob/main/.github/ISSUE_TEMPLATE/team_meet.md}{Team Meeting}
    \begin{itemize}
        \item Each team meeting will have a GitHub issue made in the previous meeting with attendance and a meeting agenda.
    \end{itemize}
    \item \href{https://github.com/russellrd/realm/blob/main/.github/ISSUE_TEMPLATE/ta_meet.md}{TA Meeting}
    \begin{itemize}
        \item All meetings with the TA will be recorded in a GitHub issue with both an agenda and questions to ask the TA.
    \end{itemize}
    \item \href{https://github.com/russellrd/realm/blob/main/.github/ISSUE_TEMPLATE/sup_meet.md}{Supervisor Meeting}
    \begin{itemize}
        \item If a supervisor for the team is found, meetings with them would be cataloged with this issue type. This will more likely be used for meetings with project stakeholders.
    \end{itemize}
    \item Project Tracking (\href{https://github.com/russellrd/realm/blob/main/.github/ISSUE_TEMPLATE/bug.md}{bugs} and \href{https://github.com/russellrd/realm/blob/main/.github/ISSUE_TEMPLATE/feature_or_doc.md}{features/documentation})
    \begin{itemize}
        \item Project bugs/features/docs are to be tracked using GitHub issues for each desired change.
    \end{itemize}
\end{enumerate}

\section{Team Member Roles} \label{team_member_roles}

The team will have the following roles:
\begin{enumerate}
    \item \textbf{Meeting Chair}
    \begin{itemize}
        \item Make a Github issue for meeting notes and the agenda for the next meeting.
    \end{itemize}
    \item \textbf{Reviewer}
    \begin{itemize}
        \item Look over GitHub pull requests to check changes are acceptable or suggest edits.
    \end{itemize}
    \item \textbf{Augmented Reality: Subject Matter Expert}
    \begin{itemize}
        \item Works on, and render assistance to, parts of the project related to augmented reality (AR).
    \end{itemize}
    \item \textbf{Unity: Subject Matter Expert}
    \begin{itemize}
        \item Works on, and render assistance to, parts of the project related to the Unity game engine.
    \end{itemize}
    \item \textbf{Team Liaison} \textit{(Russell)}
    \begin{itemize}
        \item Monitors email for team updates by course instructors and keeps the rest of the group up-to-date with the latest communications.
    \end{itemize}
\end{enumerate}

The meeting chair and reviewer will rotate on a weekly basis in alphabetical order of last name. Other roles will be assigned to members by October 1, 2024.

\section{Workflow Plan} \label{workflow_plan}

The team will use git for version control and store the project files remotely on GitHub. All branches will have a prefix to signify what type of changes are being made followed by a descriptive name that can be linked to an issue in the \textit{GitHub Project}. When a change is finished, a new pull request will be created from a branch and contain a reference the \textit{GitHub Issue}(s) being addressed by the commits within the branch.\\

The GitHub repo will have all the properties outlined below:
\begin{itemize}
    \item \textbf{Default branch:} \textit{dev}
    \item \textbf{Main branch:} \textit{main}
    \item \textbf{Branch Prefix Types:}
    \begin{itemize}
        \item \textit{feature/} (New features)
        \item \textit{fix/} (Fixing bugs in already existing code)
        \item \textit{release/} (Designated release)
        \item \textit{docs/} (Changes to documentation)
    \end{itemize}
    \item \textbf{Pull Requests (PRs):}
    \begin{itemize}
        \item Use the PR template in repo
        \item Requires at least 1 reviewer for merge
        \item All members will be automatically added as a reviewer to all PRs
    \end{itemize}
    \item \textbf{Issues:}
    \begin{itemize}
        \item Will be created as described above in the \nameref{team_communication_plan}
        \item PRs will reference the corresponding issue(s) being addressed
    \end{itemize}
    \item \textbf{Releases:}
    \begin{itemize}
        \item Commits for features that complete an epic will be tagged and made into a release
    \end{itemize}
\end{itemize}

When not working directly with git, all work is to be added to the repo under one commit co-authored by all those who contributed. The work breakdown by each member will be described in the commit message. \\

After every push to the remote repo, a GitHub action will be run to build the project for both iOS and Android devices. The artifacts created in this process will be available to download. Additionally, automated tests may be added in the build process to guard against app breaking changes. This will function as the continuous integration (CI) for the project. Continuous delivery (CD) is not necessary for this project since builds will be installed on devices manually and will not require automated deployment. 

\section{Project Decomposition and Scheduling} \label{project_decomposition_and_scheduling}

\textit{GitHub Projects} will be used to organize the sub-tasks needed to complete different milestones. These sub-tasks will be represented by \textit{GitHub Issues} within the repo. Milestones will represent less granular tasks such as a major feature in the POC where many \textit{GitHub Issues} could be attached to a single milestone. \textit{GitHub Issues} will follow the project tracking templates mentioned in the \nameref{team_communication_plan}. In the templates, there are checklists at the bottom for all the changes that need to be made. Once they are all checked off, a PR should be created using the PR template and linked to the corresponding issue. Three views will be used within the \textit{GitHub Projects} to see the tasks and as a table, timeline, and kanban board. \\

The \textit{GitHub Project} can be found \href{ https://github.com/users/russellrd/projects/2}{here}. \\

This project will be scheduled with broad milestones representing major features that are interwoven with course deliverables. The planned milestone start and completion times are scheduled based on the deadlines of the course deliverables they directly contribute to. \\

\definecolor{course-defined}{HTML}{d3d3d3}

\begin{table}[hp]
    \caption{Planned Milestone Scheduling}
    \label{tab:milestone_table}
    \centering
    \begin{tabular}{cp{5cm}>{\centering\arraybackslash}p{1.75cm}>{\centering\arraybackslash}p{2.1cm}} 
    \toprule
         \textbf{Release}&  \textbf{Milestone}&  \textbf{Planned Start}& \textbf{Planned Completion}\\ 
    \hline \hline
 & Problem Statement, POC Plan, Development Plan& &2024-09-23\\ \hline
         N/A&  Learn Unity, C\#, AR&  2024-09-14
& 2024-10-01
\\ \hline
         0.0&  Setup Project Infrastructure&  2024-09-14
& 2024-10-01
\\ \hline
\rowcolor{course-defined}
 & Requirements Document [Rev 0]& &2024-10-09\\ \hline 
\rowcolor{course-defined}
 & Hazard Analysis [Rev 0]& &2024-10-23\\ \hline
\rowcolor{course-defined}
 & V\&V Plan [Rev 0]& &2024-11-01\\ \hline
         0.1&  Implement AR object placement&  2024-10-01
& 2024-11-11
\\ \hline
         0.2&  Implement 3D object scanning&  2024-10-01
& 2024-11-11
\\ \hline
\rowcolor{course-defined}
 & Proof of Concept& &2024-11-11\\ \hline
         0.3&  Design AR object database&  2024-10-15
& 2024-11-11
\\ \hline
         0.4&  Implement user account system with cloud storage&  2024-11-11
& 2024-11-20
\\ \hline
 0.5& Implement friends, groups, and object sharing& 2024-11-11
&2024-12-01
\\ \hline
 0.6& Implement tour system& 2025-01-01
&2025-01-14
\\ \hline
 1.0& Implement a variety of settings & 2025-01-01
&2025-01-14
\\ \hline
\rowcolor{course-defined}
 & Design Document [Rev 0]& &2025-01-15\\ \hline
 1.1& User Documentation& 2025-01-15
&2025-02-01
\\ \hline
 1.2& Usability Testing& 2025-01-15
&2025-02-01
\\ \hline
\rowcolor{course-defined}
 & Demonstration [Rev 0]& &2025-02-05\\ \hline
 \rowcolor{course-defined}
 & V\&V Report [Rev 0]& &2025-03-07\\ \hline
 \rowcolor{course-defined}
 & Final Demonstration [Rev 1]& &2025-03-24\\ \hline
 \rowcolor{course-defined}
 & EXPO Demonstration& &2025-04-01\\ \hline
 \rowcolor{course-defined}
 & Final Documentation [Rev 1]& &2025-04-01\\ \hline
 1.3& Implement 3D object generation via prompts& 2025-04-30
&2025-12-31
\\ \hline
 1.4& Implement desktop tour editor& 2025-04-30
&2025-12-31
\\ 
    \bottomrule
    \end{tabular}
\end{table}
Note: Course defined milestones have a \colorbox{course-defined}{grey colored background}

\section{Proof of Concept Demonstration Plan}

One of the biggest uncertainties of the project is the achievability of 
the precise spatial positioning of AR objects required by the user experience we envision for our product. 
One of the core ideas of our product is to create context specific AR experiences; 
this will not be possible if we cannot precisely recreate AR objects where they were originally placed.\\

The other major uncertainty in this project is the feasibility of 
allowing users to scan in their own 3D objects using only camera and sensor data available on a smartphone. 
This feature will be an important avenue for user generated content on the platform, 
without which, general users will primarily be limited to prebuilt 3D models provided by the platform.\\

Our POC demonstration will require both of these features, 
as well as the minimum UI required to implement the following user experience:\\

\begin{enumerate}
  \item User scans object, obtaining a 3D model they can place
  \item User virtually places said object in their surroundings
  \item User leaves the area, returns, and can see the object they placed in the correct position
\end{enumerate}


\section{Expected Technology}

From our investigation into AR technologies, Unity with AR Foundation (an extension built on ARCore and ARKit) is the most promising option to meet the needs of our project. Those needs being:
\begin{itemize}
  \item Cross platform support
  \item Visual UI editor
  \item Integration with powerful AR libraries with capabilities such as
  \begin{itemize}
        \item Surface detection
        \item Lighting detection
        \item Motion tracking
        \item Depth measurement
      \end{itemize}
\end{itemize}


As will be using Unity, our primary programming language will be C\#.\\

The best testing tool to use will be the NUnit based Unity Test Framework due to its integration with unity, 
and NUnit's relative popularity and similarity to JUnit, 
which will make finding learning resources easy. 
This framework will provide both unit testing as well as code coverage metrics.\\

Application profiling will be done using Unity's built in profiler whenever necessary\\

Version control will be managed using git, 
and the remote repository will be hosted on GitHub as it is the industry standard, 
and has integration with CI and project management tools.\\

Github Projects will be used for project management, 
the details of which are outlined in \nameref{project_decomposition_and_scheduling}.\\

Continuous integration will be done via GitHub actions. 
Unity testing and builds will be achieved using the 
\href{https://github.com/marketplace/actions/unity-builder}{Unity Builder} 
and \href{https://github.com/marketplace/actions/unity-test-runner}{Unity Test Runner} 
github actions provided by the GameCI open source project as a base. 
CI builds will be triggered on pushes to the dev branch and PRs to the main branch 
where changes have been made in the source code directory.\\

The project will primarily be developed using the Unity Engine, 
with VSCode / VS as the preferred code editor to enforce code standards. \\


\section{Coding Standard}

All code must be formatted by the Microsoft C\# formatter with default settings, 
and all suggestions from the Microsoft C\# linter must be followed. 
In general we will follow \href{https://learn.microsoft.com/en-us/dotnet/csharp/fundamentals/coding-style/coding-conventions}{Microsoft C\# code standards.}

\newpage{}

\section*{Appendix --- Reflection}

\wss{Not required for CAS 741}

\input{../Reflection.tex}

\begin{enumerate}
    \item Why is it important to create a development plan prior to starting the
    project?
    \item In your opinion, what are the advantages and disadvantages of using
    CI/CD?
    \item What disagreements did your group have in this deliverable, if any,
    and how did you resolve them?
\end{enumerate}

\newpage{}

\section*{Appendix --- Team Charter}

\wss{borrows from
\href{https://engineering.up.edu/industry_partnerships/files/team-charter.pdf}
{University of Portland Team Charter}}

\subsection*{External Goals}

\wss{What are your team's external goals for this project? These are not the
goals related to the functionality or quality fo the project.  These are the
goals on what the team wishes to achieve with the project.  Potential goals are
to win a prize at the Capstone EXPO, or to have something to talk about in
interviews, or to get an A+, etc.}

\subsection*{Attendance}

\subsubsection*{Expectations}

\wss{What are your team's expectations regarding meeting attendance (being on
time, leaving early, missing meetings, etc.)?}

\subsubsection*{Acceptable Excuse}

\wss{What constitutes an acceptable excuse for missing a meeting or a deadline?
What types of excuses will not be considered acceptable?}

\subsubsection*{In Case of Emergency}

\wss{What process will team members follow if they have an emergency and cannot
attend a team meeting or complete their individual work promised for a team
deliverable?}

\subsection*{Accountability and Teamwork}

\subsubsection*{Quality} 

\wss{What are your team's expectations regarding the quality
of team members' preparation for team meetings and the quality of the
deliverables that members bring to the team?}

\subsubsection*{Attitude}

\wss{What are your team's expectations regarding team members' ideas,
interactions with the team, cooperation, attitudes, and anything else regarding
team member contributions?  Do you want to introduce a code of conduct?  Do you
want a conflict resolution plan?  Can adopt existing codes of conduct.}

\subsubsection*{Stay on Track}

\wss{What methods will be used to keep the team on track? How will your team
ensure that members contribute as expected to the team and that the team
performs as expected? How will your team reward members who do well and manage
members whose performance is below expectations?  What are the consequences for
someone not contributing their fair share?}

\wss{You may wish to use the project management metrics collected for the TA and
instructor for this.}

\wss{You can set target metrics for attendance, commits, etc.  What are the
consequences if someone doesn't hit their targets?  Do they need to bring the
coffee to the next team meeting?  Does the team need to make an appointment with
their TA, or the instructor?  Are there incentives for reaching targets early?}

\subsubsection*{Team Building}

\wss{How will you build team cohesion (fun time, group rituals, etc.)? }

\subsubsection*{Decision Making} 

\wss{How will you make decisions in your group? Consensus?  Vote? How will you
handle disagreements? }

\end{document}