\documentclass{article}

\usepackage{tabularx}
\usepackage{booktabs}

\title{Problem Statement and Goals\\\progname}

\author{\authname}

\date{}

%% Comments

\usepackage{color}

\newif\ifcomments\commentstrue %displays comments
%\newif\ifcomments\commentsfalse %so that comments do not display

\ifcomments
\newcommand{\authornote}[3]{\textcolor{#1}{[#3 ---#2]}}
\newcommand{\todo}[1]{\textcolor{red}{[TODO: #1]}}
\else
\newcommand{\authornote}[3]{}
\newcommand{\todo}[1]{}
\fi

\newcommand{\wss}[1]{\authornote{blue}{SS}{#1}} 
\newcommand{\plt}[1]{\authornote{magenta}{TPLT}{#1}} %For explanation of the template
\newcommand{\an}[1]{\authornote{cyan}{Author}{#1}}

%% Common Parts

\newcommand{\progname}{Software Engineering} % PUT YOUR PROGRAM NAME HERE
\newcommand{\authname}{Team \#13, ARC
    \\ Avanish, Ahluwalia
    \\ Russell, Davidson
    \\ Rafey, Malik
    \\ Abdul, Zulfiqar} % AUTHOR NAMES                  

\usepackage{hyperref}
    \hypersetup{colorlinks=true, linkcolor=blue, citecolor=blue, filecolor=blue,
                urlcolor=blue, unicode=false}
    \urlstyle{same}
                                


\begin{document}

\maketitle

\begin{table}[hp]
\caption{Revision History} \label{TblRevisionHistory}
\begin{tabularx}{\textwidth}{llX}
\toprule
\textbf{Date} & \textbf{Developer(s)} & \textbf{Change}\\
\midrule
Date1 & Name(s) & Description of changes\\
Date2 & Name(s) & Description of changes\\
... & ... & ...\\
\bottomrule
\end{tabularx}
\end{table}

\section{Problem Statement}

\wss{You should check your problem statement with the
\href{https://github.com/smiths/capTemplate/blob/main/docs/Checklists/ProbState-Checklist.pdf}
{problem statement checklist}.} 

\wss{You can change the section headings, as long as you include the required
information.}

\subsection{Problem}

\subsection{Inputs and Outputs}

\wss{Characterize the problem in terms of ``high level'' inputs and outputs.  
Use abstraction so that you can avoid details.}

\subsection{Stakeholders}

\subsection{Environment}

\wss{Hardware and software environment}

\section{Goals}
This section of the document will outline the goals for the project.

\subsection{Create an AR app}
The app will provide augmented reality experiences and will allow users to interact with the real world using a smartphone camera. This will create immersive experiences that blend the physical and digital worlds.

\subsection{User-Friendly Interface for Social Interactions}
The app will allow users to add friends, chat, and share AR experiences. The focus will be on creating an intuitive and seamless interface that makes it easy to connect and interact with friends, both in private groups and publicly.

\subsection{Cater to Organizations’ Needs}
The app will include tools that help organizations, such as businesses, schools, or campuses, to create tours and virtual experiences. These will provide unique educational, promotional, or interactive content.

\subsection{Develop a Large Set of Unique AR Objects}
By the end of the project, the app will feature a wide variety of pre-built unique AR objects that users can interact with. These objects will be meaningful and should be usable for most of what users want to do within the app.

\subsection{Scan and Upload Objects}
Users can scan real-world objects and upload them into the app’s AR environment, allowing users to create digital representations of physical objects.

\subsection{2D Object Creation (Stickers)}
In addition to the pre-made 3D AR objects, users will be able to create and share their own 2D objects, such as stickers, within the app. These stickers can be placed in the AR environment, providing a fun and creative way for users to express themselves.

\subsection{Cross-Platform Availability}
The app will be developed for both Android and iOS platforms, ensuring that users on a wide range of devices can access the app, which will maximize app availability.

\section{Stretch Goals}

\subsection{User-Generated 3D AR Objects}
The app will allow users to create their own personalized 3D AR objects directly within the virtual world. Users can generate their own personal AR objects via prompts and can store , design and customize objects, and then share them with others.

\subsection{Desktop App for Custom AR Tours}
A desktop application will be developed for users who want to create custom AR tours, such as for parks, museums, or historical sites. This tool will allow more advanced users, such as educators or businesses, to design detailed, interactive tours that can be shared with the broader user base on the mobile app, or with visitors to the attractions.

\section{Challenge Level and Extras}

\wss{State your expected challenge level (advanced, general or basic).  The
challenge can come through the required domain knowledge, the implementation or
something else.  Usually the greater the novelty of a project the greater its
challenge level.  You should include your rationale for the selected level.
Approval of the level will be part of the discussion with the instructor for
approving the project.  The challenge level, with the approval (or request) of
the instructor, can be modified over the course of the term.}

\wss{Teams may wish to include extras as either potential bonus grades, or to
make up for a less advanced challenge level.  Potential extras include usability
testing, code walkthroughs, user documentation, formal proof, GenderMag
personas, Design Thinking, etc.  Normally the maximum number of extras will be
two.  Approval of the extras will be part of the discussion with the instructor
for approving the project.  The extras, with the approval (or request) of the
instructor, can be modified over the course of the term.}

This project is of general challenge level. An AR app project is considered more advanced than a basic app or webapp due to the technical complexity involved in developing an AR app. This would include 3D modeling, computer vision and real-time rendering. Additionally, we would integrate multiple technologies such as GPS and motion tracking, which can add to the complexity of our project. It is not considered an advanced project based on the criteria provided in the slides, as that category is restricted for more of a research based project.

The extras we included are user documentation and usability testing.

\newpage{}

\section*{Appendix --- Reflection}

\wss{Not required for CAS 741}

The purpose of reflection questions is to give you a chance to assess your own
learning and that of your group as a whole, and to find ways to improve in the
future. Reflection is an important part of the learning process.  Reflection is
also an essential component of a successful software development process.  

Reflections are most interesting and useful when they're honest, even if the
stories they tell are imperfect. You will be marked based on your depth of
thought and analysis, and not based on the content of the reflections
themselves. Thus, for full marks we encourage you to answer openly and honestly
and to avoid simply writing ``what you think the evaluator wants to hear.''

Please answer the following questions.  Some questions can be answered on the
team level, but where appropriate, each team member should write their own
response:


\begin{enumerate}
    \item What went well while writing this deliverable? 
    \item What pain points did you experience during this deliverable, and how
    did you resolve them?
    \item How did you and your team adjust the scope of your goals to ensure
    they are suitable for a Capstone project (not overly ambitious but also of
    appropriate complexity for a senior design project)?
\end{enumerate}  

\end{document}
