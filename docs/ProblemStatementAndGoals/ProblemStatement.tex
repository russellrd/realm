\documentclass{article}

\usepackage{tabularx}
\usepackage{booktabs}

\title{Problem Statement and Goals\\\progname}

\author{\authname}

\date{}

%% Comments

\usepackage{color}

\newif\ifcomments\commentstrue %displays comments
%\newif\ifcomments\commentsfalse %so that comments do not display

\ifcomments
\newcommand{\authornote}[3]{\textcolor{#1}{[#3 ---#2]}}
\newcommand{\todo}[1]{\textcolor{red}{[TODO: #1]}}
\else
\newcommand{\authornote}[3]{}
\newcommand{\todo}[1]{}
\fi

\newcommand{\wss}[1]{\authornote{blue}{SS}{#1}} 
\newcommand{\plt}[1]{\authornote{magenta}{TPLT}{#1}} %For explanation of the template
\newcommand{\an}[1]{\authornote{cyan}{Author}{#1}}

%% Common Parts

\newcommand{\progname}{Software Engineering} % PUT YOUR PROGRAM NAME HERE
\newcommand{\authname}{Team \#13, ARC
    \\ Avanish, Ahluwalia
    \\ Russell, Davidson
    \\ Rafey, Malik
    \\ Abdul, Zulfiqar} % AUTHOR NAMES                  

\usepackage{hyperref}
    \hypersetup{colorlinks=true, linkcolor=blue, citecolor=blue, filecolor=blue,
                urlcolor=blue, unicode=false}
    \urlstyle{same}
                                


\begin{document}

\maketitle

\begin{table}[hp]
\caption{Revision History} \label{TblRevisionHistory}
\begin{tabularx}{\textwidth}{llX}
\toprule
\textbf{Date} & \textbf{Developer(s)} & \textbf{Change}\\
\midrule
2024-09-24 & Rafey Malik & Added Goals, Stretch Goals, and Challenge Level and Extras\\
2024-09-24 & Avanish Ahluwalia & Added Problem Statement, System inputs and outputs, Stakeholders and Software and Hardware Environment\\
2024-09-24 & Rafey Malik & Added Reflection responses\\
\bottomrule
\end{tabularx}
\end{table}

\section{Problem Statement}

\subsection{Problem}
One of the big failures of popular social media platforms is that they encourage users to sit idly and spend time idolizing others who do exciting things in the real world. There is a notable lack of social platforms that encourage users to go out and interact with the world, while still providing a space for social interaction. Current social platforms also don’t encourage educational experiences like those provided at museums, parks, historical landmarks, etc., and educational experiences often fail to entertain the audience and keep them engaged on long tours. There needs to be more AR applications that can allow users to seamlessly create, share, and interact with augmented reality in both educational and social contexts.\\
The solution to this problem is to provide an enhanced platform where users can view AR objects in the physical world. Users would be able to create and share their objects with others. To tackle the problem of boring tours, the solution provides organizations with a platform to create interactive AR tours that include the presentation of AR objects combined with eye-catching animations.

\subsection{Inputs and Outputs}

\textbf{Inputs}:
\begin{itemize}
  \item Images: Images used to create AR objects
  \item Geographic data: Information about the user’s surroundings and position for placing AR objects
  \item User input from the application interface: Physical user input to navigate through different functionalities in the application
  \item AR Object generation prompts: Prompts provided by users to create custom AR objects
  \item Sensor Data: Readings from an array of sensors on the user’s mobile device such as the gyroscope, accelerometer, ambient light sensor and magnetometer
\end{itemize}

\begin{flushleft}
\textbf{Outputs}:
\end{flushleft}
\begin{itemize}
  \item Social hub: Personal profile, interaction with friends, sharing personal AR objects
  \item Virtual AR objects: Objects that are projected in the user’s physical 3D environment in real time
  \item Visual and audio feedback while utilizing core features
  \item User Analytics: Data visualizing statistics about user interactions and contributions
  \item Enhanced AR tours: A sequence of AR object presentations involving effects and animations with provided cues
\end{itemize}

\subsection{Stakeholders}

\begin{flushleft}
 \textbf{General public}: All age cohorts can use the application. The user base will mostly comprise social media enthusiasts who enjoy interactions with the physical world through their devices. They will have access to the app akin to any other social media.\linebreak

 \textbf{Tour Operators}: Organization involved in tour planning and setting up tour events. These organizations manage designated park areas, cities, and museums. Current setups for tours involve voiceovers and brochures which fail to keep the audience’s attention. The app will allow them to further their educational mandates. They will be able to create events that improve the tour experience.
\end{flushleft}

\subsection{Environment}

The application will require knowledge about the user’s surroundings. The hardware needed to run the platform would be smart mobile devices with a set of sensors that are present on all devices in the present world. The application must be able to operate on both Android and IOS systems. 

\section{Goals}
This section of the document will outline the goals for the project.

\subsection{Create an AR app}
The app will provide augmented reality experiences and will allow users to interact with the real world using a smartphone camera. This will create immersive experiences that blend the physical and digital worlds.

\subsection{User-Friendly Interface for Social Interactions}
The app will allow users to add friends, chat, and share AR experiences. The focus will be on creating an intuitive and seamless interface that makes it easy to connect and interact with friends, both in private groups and publicly.

\subsection{Cater to Organizations’ Needs}
The app will include tools that help organizations, such as businesses, schools, or campuses, to create tours and virtual experiences. These will provide unique educational, promotional, or interactive content.

\subsection{Develop a Large Set of Unique AR Objects}
By the end of the project, the app will feature a wide variety of pre-built unique AR objects that users can interact with. These objects will be meaningful and should be usable for most of what users want to do within the app.

\subsection{Scan and Upload Objects}
Users can scan real-world objects and upload them into the app’s AR environment, allowing users to create digital representations of physical objects.

\subsection{2D Object Creation (Stickers)}
In addition to the pre-made 3D AR objects, users will be able to create and share their own 2D objects, such as stickers, within the app. These stickers can be placed in the AR environment, providing a fun and creative way for users to express themselves.

\subsection{Cross-Platform Availability}
The app will be developed for both Android and iOS platforms, ensuring that users on a wide range of devices can access the app, which will maximize app availability.

\section{Stretch Goals}

\subsection{User-Generated 3D AR Objects}
The app will allow users to create their own personalized 3D AR objects directly within the virtual world. Users can generate their own personal AR objects via prompts and can store , design and customize objects, and then share them with others.

\subsection{Desktop App for Custom AR Tours}
A desktop application will be developed for users who want to create custom AR tours, such as for parks, museums, or historical sites. This tool will allow more advanced users, such as educators or businesses, to design detailed, interactive tours that can be shared with the broader user base on the mobile app, or with visitors to the attractions.

\section{Challenge Level and Extras}

This project is of general challenge level. An AR app project is considered more advanced than a basic app or webapp due to the technical complexity involved in developing an AR app. This would include 3D modeling, computer vision and real-time rendering. Additionally, we would integrate multiple technologies such as GPS and motion tracking, which can add to the complexity of our project. It is not considered an advanced project based on the criteria provided in the slides, as that category is restricted for more of a research based project.\\

The extras we included are user documentation and usability testing.\\

\newpage{}

\section*{Appendix --- Reflection}

The purpose of reflection questions is to give you a chance to assess your own
learning and that of your group as a whole, and to find ways to improve in the
future. Reflection is an important part of the learning process.  Reflection is
also an essential component of a successful software development process.  

Reflections are most interesting and useful when they're honest, even if the
stories they tell are imperfect. You will be marked based on your depth of
thought and analysis, and not based on the content of the reflections
themselves. Thus, for full marks we encourage you to answer openly and honestly
and to avoid simply writing ``what you think the evaluator wants to hear.''

Please answer the following questions.  Some questions can be answered on the
team level, but where appropriate, each team member should write their own
response:


\begin{enumerate}
    \item What went well while writing this deliverable? \\
    
    We initially discussed our ideas for various parts in this deliverable, and then we split up the deliverable amongst ourselves to write in further detail. This worked out well, as we were first all on the same page, and then were able to expand on these points individually.
    
    \item What pain points did you experience during this deliverable, and how
    did you resolve them? \\
    
    We disagreed on a couple of things within the deliverable, which halted our progress for a little bit. For example, we didn’t agree on which goals were regular goals and which were stretch goals. Some team members thought that certain goals belonged to either category. This was resolved by consulting the entire group and first agreeing on which goals we all agreed were regular goals. These were fundamental goals of our projects. After this, we went through each of the goals and discussed why they might be considered regular or stretch goals. We decided that any goals that are a little more ambitious than what we want the fundamentals of our project to be should be considered stretch goals. After agreeing on this criteria, it became easy to unanimously categorize the goals.
    
    \item How did you and your team adjust the scope of your goals to ensure
    they are suitable for a Capstone project (not overly ambitious but also of
    appropriate complexity for a senior design project)? \\
    
    We decided to make our goals reflect what it is we want to do with this project, while also keeping in mind our limitations. In our brainstorming phase, we were very ambitious with our goals, and then we cut them down based on what we deemed realistic given the scope of this project.
    
\end{enumerate}  

\end{document}
