\documentclass{article}

\usepackage{booktabs}
\usepackage{tabularx}
\usepackage{caption}
\usepackage{bookmark}
\usepackage{enumitem}
\usepackage{hyperref}
\usepackage{graphicx}
\usepackage{multirow}

% From: https://tex.stackexchange.com/questions/438876/how-to-cross-reference-text-with-a-custom-label-reference-text
\makeatletter
\newcommand{\labeltext}[3][]{%
    \@bsphack%
    \csname phantomsection\endcsname% in case hyperref is used
    \def\tst{#1}%
    \def\labelmarkup{\emph}% How to markup the label itself
    %\def\refmarkup{\labelmarkup}% How to markup the reference
    \def\refmarkup{}%
    \ifx\tst\empty\def\@currentlabel{\refmarkup{#2}}{\label{#3}}%
    \else\def\@currentlabel{\refmarkup{#1}}{\label{#3}}\fi%
    \@esphack%
    \labelmarkup{#2}% visible printed text.
}
\makeatother

\title{Software Requirements Specification: Realm\\\progname}

\author{\authname}

\date{}

\input{../Comments}
%% Common Parts

\newcommand{\progname}{Software Engineering} % PUT YOUR PROGRAM NAME HERE
\newcommand{\authname}{Team \#13, ARC
    \\ Ahluwalia, Avanish
    \\ Davidson, Russell
    \\ Malik, Rafey
    \\ Zulfiqar, Abdul} % AUTHOR NAMES                  

\usepackage{hyperref}
    \hypersetup{colorlinks=true, linkcolor=blue, citecolor=blue, filecolor=blue,
                urlcolor=blue, unicode=false}
    \urlstyle{same}
                                


\begin{document}

\maketitle

\newpage{}

\tableofcontents

\addcontentsline{toc}{section}{Revision History}
\section*{Revision History}

\begin{table}[hp]
    \caption{Revision History} \label{rev_history_table}
    \begin{tabularx}{\textwidth}{p{3cm}p{2cm}X}
        \toprule {\textbf{Date}} & {\textbf{Version}} & {\textbf{Notes}}                                                \\
        \midrule
        2024-10-11               & 1.0                & Initial SRS                                                     \\
        2024-10-29               & 1.1                & Reorganization and addition of NFRs, Completed missing sections \\
        \bottomrule
    \end{tabularx}
\end{table}

\section{Introduction}

\subsection{Document Purpose}

The purpose of this document is to outline key functional and non-functional requirements and constraints for the AR project, Realm. This will help plan and guide the developers throughout the development and testing of the app. The intended audience is for the developers and testers of this project. Additionally, it can serve as a reference to the stakeholders who require an understanding of the capabilities and expectations of the project.

\subsection{Product Scope}

Realm (version 1.4) is an AR platform that aims to create a cohesive augmented reality experience that enables users to engage in social, educational and entertaining activities through the use of AR content. The app will allow users to place AR objects, react to them, interact with friends, and participate in educational tours.

The key objectives of this platform include fostering social interactions through shared and customizable AR experiences and promoting educational content via AR tours.

\subsection{Definitions, Acronyms and Abbreviations}
\label{sub:def_acr_abb}

\begin{itemize}
    \item \labeltext{AR object}{def:ar_obj}: A 2D/3D model representing an object that can be presented in AR
          \begin{itemize}
              \item \labeltext{Artifact}{def:artifact}: a 3D AR Object
              \item \labeltext{Sticker}{def:sticker}: a 2D AR Object
          \end{itemize}
    \item \labeltext{AR object instance}{def:ar_obj_instance}: An instance of an AR object placed in the real world, defined by an AR object, and a location
    \item \labeltext{AR object cluster}{def:ar_obj_cls}: A group of AR objects present within a confined area.
    \item \labeltext{Users}{def:user}: A term for anyone who uses the app.
    \item \labeltext{Organization users}{def:org_user}: Users who belong to an organization that have the ability to create tours. They are affiliated with a particular approved organization who have the ability to modify tours within their organization’s domain.
    \item \labeltext{General users}{def:gen_user}: Users that can access most of the app functionalities except creating tours.
    \item \labeltext{Admins}{def:admin}: People who have access to nothing else but the admin interface within the app.
    \item \labeltext{Sub-Realm}{def:sub_realm}: A construct that specifies a set of users that are it’s members, and a set of AR objects that are shared between its members
    \item \labeltext{Encryption standard}{def:encryption_standard}: A set of algorithms used to encode data to ensure that it can be viewed by authorized users only.
    \item \labeltext{Restricted Area}{def:restricted_area}: A space that should not be accessible to the general public.
\end{itemize}

\subsection{References}
\label{sub:references}

\begingroup
\raggedright
\bibliography{ref}
\endgroup
\bibliographystyle{ieeetr}


\section{Product Overview}

\subsection{Product Perspective}

Realm, as specified in this SRS document, is a self-contained product designed to provide users with an immersive social, educational, and entertainment experience with the help of AR. This product is not part of an existing product family and does not replace any current systems. Instead, it is developed as a standalone system that allows users to create, interact with, and share virtual objects, such as 2D stickers and 3D artifacts, in real-world environments.\\

The app operates independently but may collaborate with third-party content providers, such as educational institutions or businesses, to offer //customized AR experiences. If needed, Realm could integrate with larger systems in the future, such as educational platforms or entertainment networks, by establishing appropriate interfaces for data sharing and interaction.\\

The app is to be compatible with most smartphone devices, as long as they have a camera.


\subsection{Product Functions}

\begin{itemize}
    \item Tutorial
          \begin{itemize}
              \item The user will be able to complete a walkthrough tutorial of all major app features.
          \end{itemize}

    \item Tour Management
          \begin{itemize}
              \item \textit{Organization users} will be able to create/modify/delete AR tours owned by their organization.
          \end{itemize}

    \item Touring
          \begin{itemize}
              \item \textit{General users} will be able to experience AR tours made by approved organizations.
          \end{itemize}

    \item User Profile Management
          \begin{itemize}
              \item Users can create, edit, and manage user profiles as well as manage privacy settings.
          \end{itemize}

    \item AR Object Interaction
          \begin{itemize}
              \item Users will be able to place, edit, and interact with 2D stickers and 3D artifacts within the \hyperref[def:sub_realm]{sub-realm}.
              \item Users can react to and engage with objects placed by other users.
          \end{itemize}

    \item Social Interaction
          \begin{itemize}
              \item Users will be able to create, edit, and add/remove friends to a \hyperref[def:sub_realm]{sub-realm}, where they can each share AR objects within the \hyperref[def:sub_realm]{sub-realm}.
          \end{itemize}

    \item Object Creation
          \begin{itemize}
              \item Allow users to create their own AR objects and share them with others.
              \item Objects can be created via a 'Scan and Upload' option or generate one using a user prompt.
          \end{itemize}

    \item Reporting and Moderation
          \begin{itemize}
              \item Allows users to report inappropriate objects which can go for review and moderation.
          \end{itemize}

    \item Managing Settings
          \begin{itemize}
              \item Manage a variety of settings including accessibility settings, display settings, privacy settings, profile settings, and \hyperref[def:sub_realm]{sub-realm} settings.
          \end{itemize}
\end{itemize}
\subsection{Product Constraints}

\begin{itemize}
    \item \textbf{Interfaces to Users}: The platform must be compatible with both Android and iOS devices, meaning the design must accommodate different screen sizes and operating system guidelines. Additionally, the app requires user-friendly interfaces for interacting with AR objects and managing \hyperref[def:sub_realm]{sub-realms}.

    \item \textbf{Hardware Constraints}: The app depends on mobile devices with AR capabilities (e.g., cameras, GPS, accelerometers). Users with older or less powerful devices might experience limited functionality or performance issues.

    \item \textbf{Quality of Service}: Realm needs to have a smooth interaction, with little to no lag. It must also handle multiple users within a shared AR space (sub-realm).

    \item \textbf{Standards Compliance}: The app has to comply with relevant privacy laws and app store requirements for both Android and iOS.

    \item \textbf{Design and Implementation Constraints}: The app needs to provide cloud storage for AR objects and user data, and its design should allow for easy scalability as the user base grows. Additionally, the AR features and general app use should be intuitive and easy to use for users.

    \item \textbf{Budget Constraints}: The project has a maximum budget of \$750.
\end{itemize}

\subsection{User Characteristics}

There are three types of users. The system has three primary user types: General Users, Organizational Users, and Admins, each with distinct roles and needs. General Users are everyday users who use the app by placing, viewing, and engaging with objects in the app. Organizational Users are businesses, educational institutions, or other groups that create and manage AR tours, requiring tools for content creation, collaboration, and moderation. Admins oversee the platform’s integrity, ensuring compliance with policies through content moderation, user management, and security enforcement. Each user type affects different aspects of the system’s design, ensuring a seamless and functional experience for all stakeholders.\\

\textbf{General Users:}
\begin{itemize}
    \item Frequency of Use: Occasional to frequent.
    \item Functions Used: Creating, interacting with, and sharing AR objects; joining \hyperref[def:sub_realm]{sub-realms}; participating in AR tours.
    \item Technical Expertise: Moderate; basic understanding of mobile apps and social platforms.
    \item Security/Privilege Level: Standard user access with limited permissions (e.g., cannot create tours or manage content).
    \item Importance: Most important user class, as they represent the primary user base for the platform.
\end{itemize}

\textbf{Organizational Users:}
\begin{itemize}
    \item Frequency of Use: Frequent.
    \item Functions Used: Creating and managing AR tours, educational content, and \hyperref[def:sub_realm]{sub-realms} for public or private use.
    \item Technical Expertise: Moderate to advanced; comfortable with content creation tools and AR platforms.
    \item Security/Privilege Level: Higher-level access to manage and publish content for users.
    \item Importance: Highly important, as they provide some of the platform’s key content.
\end{itemize}

\textbf{Administrators/Moderators:}
\begin{itemize}
    \item Frequency of Use: Regular.
    \item Functions Used: Managing user permissions, moderating content, handling reports of inappropriate AR objects.
    \item Technical Expertise: Advanced; familiar with platform management and moderation tools.
    \item Security/Privilege Level: Full access to the platform's management functions, including user and content control.
    \item Importance: Less important for day-to-day use, but crucial for ensuring platform integrity and moderation.
\end{itemize}

\subsection{Assumptions and Dependencies}

\begin{itemize}
    \item User has used a mobile app before.
    \item User has internet access.
    \item User device can provide GPS location services.
    \item User device has a camera.
    \item User device utilizes either Android or iOS operating systems.
\end{itemize}

\subsection{Apportioning of Requirements}

\begin{table}[]
    \caption{Apportioning of Requirements}
    \label{tab:apportioning_table}
    \centering
    \begin{tabular}{lcc}
        \toprule
        \textbf{Function}        & \textbf{Priority} & \textbf{Dates}              \\
        \hline \hline
        Object Creation          & 1                 & \multirow{2}{*}{2024-11-11} \\
        AR Object Interaction    & 1                 &                             \\ \hline
        Tour Management          & 2                 & \multirow{3}{*}{2025-12-01} \\
        Touring                  & 2                 &                             \\
        Social Interaction       & 2                 &                             \\
        Reporting and Moderation & 2                 & \multirow{4}{*}{2025-01-15} \\ \hline
        User Profile Management  & 3                 &                             \\
        Managing Settings        & 3                 &                             \\
        Tutorial                 & 3                 &                             \\
        \hline
    \end{tabular}
\end{table}

\subsection{Out of Scope}

The following features are considered out of scope for this version of the system:

\begin{itemize}
    \item \textbf{AR Glasses Support:} The system will not be optimized for AR headsets or smart glasses in this release.
    \item \textbf{Offline Mode:} The system requires an active internet connection and will not support full offline functionality.
\end{itemize}

\section{Requirements}


\subsection{External Interfaces}

\subsubsection{User interfaces}

\begin{center}
    \includegraphics[scale=0.4]{OtherDiagrams/ui_flow.png}
\end{center}

\textbf{Interface Names}
\begin{itemize}
    \item \textbf{Realm:} the interface in the app that provides a view of AR objects imposed on a feed of the user’s camera.
    \item \textbf{Tours:}
          \begin{itemize}
              \item \textbf{Tour List:} Shows a list of the available tours and an option to start them.
              \item \textbf{Tour Editor:} Allows organization users to edit tours and publish changes.
          \end{itemize}
    \item \textbf{Maps:} Map view displaying objects placed around you
    \item \textbf{Inventory:} A display of available AR objects (personal objects, objects mapped to a \hyperref[def:sub_realm]{sub-realm} and application preset objects)
    \item \textbf{Profile:}
          \begin{itemize}
              \item \textbf{Sub-Realms:} Ability to add, edit and interact with \hyperref[def:sub_realm]{sub-realm} settings
              \item \textbf{Friends:} Managing friends
              \item \textbf{Settings:} Accessibility settings, Display settings, Privacy settings, Profile Settings, \ref{def:sub_realm} settings
              \item \textbf{Help:} FAQ and request org account
          \end{itemize}
    \item \textbf{Placement Editor:} Able to move, rotate, resize the object prior to confirming placement.
\end{itemize}

There must be a navigation menu that can be accessed from any page that allows the user to navigate to any of the four main pages: Realm, User Profile, Tours, and Map

\subsubsection{Look and Feel Requirements}



\begin{enumerate}[align=left, label=\textbf{EI-LF\arabic*:}]
    \item The Realm Interface must provide an immersive view of AR objects imposed on the real world, and must be minimally interrupted by other UI elements. \\
          Fit Criteria: Permanent UI elements besides the camera feed should take no more than 10\% of the space on the screen. \\

\end{enumerate}





\subsubsection{Hardware interfaces}

\begin{center}
    \includegraphics[scale=0.4]{OtherDiagrams/hscd.png}
\end{center}

The above figure shows the hardware inputs of the system and the outputs of the system that directly follow from the hardware inputs.


\subsubsection{Software interfaces}

\begin{center}
    \includegraphics[scale=0.4]{OtherDiagrams/sscd.png}
\end{center}

The above figure shows the external systems our system will interact with, and the outputs of the system that directly follow from the external software inputs

\subsection{Functional}
\label{sub:functional}

The functional requirements of the system broken down by the features they specify

\subsubsection{Maps}
\label{ssub:maps}

\begin{enumerate}[align=left, label=\textbf{MP-FR\arabic*:}]
    \item The map should provide the user's location
    \item The map should display location markers for \ref{def:ar_obj} clusters
    \item The marker should display the count of objects present in a tight cluster
    \item The map should display location markers for \ref{def:ar_obj_cls}
    \item The map should display indications of objects from all \hyperref[def:sub_realm]{sub-realms} that the user is a part of
    \item When a marker is selected, the option to provide directions should be available
    \item The map should give directions to the user for a selected marker
    \item The objects shown on the map should be grouped to avoid clutter
    \item The map should allow the user to terminate navigation. \\
    \item The map should be able to identify \hyperref[def:restricted_area]{restricted areas} and disallow navigation.\\
\end{enumerate}

\subsubsection{Inventory}
\label{ssub:inventory}

\begin{enumerate}[align=left, label=\textbf{IV-FR\arabic*:}]
    \item The user should be able to delete objects from their inventory
    \item The user should be able to add objects to their inventory
    \item The inventory should at least have the application-provided objects
    \item The inventory should contain a maximum of 100 personal objects excluding the application-provided ones
    \item The personal objects present in the inventory should either be generated by the user or shared by other users
    \item The inventory should provide the count of the total objects
    \item The inventory should store both 2D and 3D AR objects
    \item The user should be able to add their objects to a “favourite group”
    \item The user should be able to sort their objects according to the number of uses, favourites and storage size
    \item The inventory should display the 3D objects in a continuous rotating state
    \item The user can change the inventory background color
\end{enumerate}

\subsubsection{Object Scan}
\label{ssub:obj_scan}
\begin{enumerate}[align=left, label=\textbf{OS-FR\arabic*:}]
    \item The user should be able to scan their surroundings to create objects.
    \item The user should select whether a 2D or 3D object is being scanned.
    \item While scanning, a real-time render of the scanned portion will be shown to the user.
    \item The scanning state should not be longer than 120 seconds.
    \item When the user has completed scanning, they should be able to confirm that the scanning process is finished.
\end{enumerate}

\subsubsection{Object Upload to Inventory}
\label{ssub:obj_upload_inv}
\begin{enumerate}[align=left, label=\textbf{OUI-FR\arabic*:}]
    \item The object render provided by the scan would be displayed to the user.
    \item The user should be able to remove unneeded features of the render after the target is done being scanned.
    \item For 2D objects, the user can crop and resize the object in the edit interface.
    \item Once the user is satisfied with the scanned object, they should be able to confirm the creation of the \ref{def:ar_obj}.
    \item The user should be able to name their created object.
    \item The name should only contain ASCII characters.
    \item The user-created name of the object should be stored along with the \ref{def:ar_obj}.
    \item The date and time of creation should be stored with the \ref{def:ar_obj}.
    \item The user information should be stored with the \ref{def:ar_obj}.
    \item The storage size should be stored with the \ref{def:ar_obj}.
    \item The type (2D or 3D) of the object should be stored with the \ref{def:ar_obj}.
    \item The user should be able to select portions of the object and change its color.
\end{enumerate}

\subsubsection{Custom AR Object Generation via prompt}
\label{ssub:prompt_obj_gen}

\begin{enumerate}[align=left, label=\textbf{POG-FR\arabic*:}]
    \item The user should be able to enter a prompt.
    \item The user prompt should not be longer than 200 characters (including spaces and punctuation).
    \item The current character count for the prompt should be displayed to the user.
    \item The user prompt should not contain profanity.
    \item The user should select whether they want a 2D or 3D object.
    \item The user should be able to confirm that they have finished writing their prompt and would like to generate the \ref{def:ar_obj}.
    \item The application should provide multiple AR objects for the entered prompt.
    \item After objects are generated, the user should be able to select an object from one of the provided options.
    \item Once the user has chosen an object, they should be able to add it to their personal inventory.
    \item The objects can be previewed by rotating them.
\end{enumerate}

\subsubsection{Tutorial}
\label{ssub:tutorial}

\begin{enumerate}[align=left, label=\textbf{TU-FR\arabic*:}]
    \item The app shall have a step-by-step interactive guide of how to use all major app features.
    \item The app shall prompt the user to complete the app tutorial after they create an account.
    \item The app shall allow the user to leave the tutorial at any time.
    \item The tutorial shall be available at any time through the app's help page in the \textbf{Settings screen}.
    \item The tutorial will involve user participation to directly use functionality in a sandbox environment.
\end{enumerate}

\begin{enumerate}[align=left, label=\textbf{TU-NFR\arabic*:}]
    \item Each interaction step should not take the user more than 15 seconds to figure out.
    \item The entire tutorial shall not take longer than 5 minutes to complete for 80\% of users.
\end{enumerate}

\subsubsection{Tour Management}
\label{ssub:tour_management}

\begin{enumerate}[align=left, label=\textbf{TM-FR\arabic*:}]
    \item The tour management functionality within the app shall only be available to \ref{def:org_user}.
    \item Tours within the app can be created as a “draft” to make it available to other \ref{def:org_user} but not released to the public.
    \item Tours within the app can be published to the public from a “draft” or from directly after creation.
    \item \ref{def:org_user} will be able to customize the following for each of their tours:
          \begin{enumerate}
              \item Name
              \item Description
              \item Route
                    \begin{enumerate}
                        \item Will be editable on a map of the tour area
                        \item Intended direction of travel can be set
                        \item Estimated time of completion that will be automatically determined by the distance if not set
                    \end{enumerate}
              \item Objects
                    \begin{enumerate}
                        \item \ref{def:ar_obj} can be placed along the route at specific geographic locations along with description text for each of the \ref{def:ar_obj}
                        \item Historical information text popups can be placed along the route
                        \item All text will have an audio playback option with text-to-speech or pre-recorded audio (if available)
                    \end{enumerate}
              \item Price
                    \begin{enumerate}
                        \item Can be free or
                        \item a price under \$10
                    \end{enumerate}
              \item Relevant Web Link(s)
          \end{enumerate}
    \item \ref{def:org_user} will have the ability to preview the tours that belong to their organization.
    \item \ref{def:org_user} will have the ability to edit tours that belong to their organization.
\end{enumerate}

\subsubsection{Touring}
\label{ssub:touring}

\begin{enumerate}[align=left, label=\textbf{TR-FR\arabic*:}]
    \item The touring functionality within the app shall only be available to \ref{def:gen_user}.
    \item The app will have three avenues for a user to find tours:
          \begin{enumerate}[align=left, label=\textbf{TR-FR2.\arabic*:}]
              \item The app will allow \ref{def:gen_user} to see a list of available tours through the \textbf{Tour List Interface}.
                    \begin{itemize}
                        \item The tours can be grouped in this page by organization or by location
                    \end{itemize}
              \item The app will also have tours show up in a push notification (if configured by a user) when in close proximity to a tour area.
              \item Locations can place QR codes at the starting location of the tour which can be scanned by a mobile camera that will open the tour preview in the app.
          \end{enumerate}
    \item Users will be able to preview a tour to see the following information:
          \begin{enumerate}
              \item Name
              \item Description
              \item Relevant web link(s)
              \item Tour distance (auto-calculated from route)
              \item Estimated time of completion
              \item A map that will show the route and locations of \ref{def:ar_obj}s
              \item Price
          \end{enumerate}
    \item Once a user starts a tour they will have two main views they can switch between:
          \begin{enumerate}[align=left, label=\textbf{TR-FR4.\arabic*:}]
              \item One of the tour views is a map
                    \begin{itemize}
                        \item The designated tour area will be outlined
                        \item The user’s current location will be shown
                        \item Intended route and direction will be overlaid
                        \item Location of \ref{def:ar_obj}s will be marked
                    \end{itemize}
              \item One of the tour views is an AR view
                    \begin{itemize}
                        \item Will be very similar to the interface of the \textbf{Realm Interface}
                        \item Have an added indicator of the intended direction
                        \item Popups for historical information
                    \end{itemize}
          \end{enumerate}
\end{enumerate}

\subsubsection{Object Placement}
\label{ssub:object_placement}

\begin{enumerate}[align=left, label=\textbf{OP-FR\arabic*:}]
    \item The system must have the capability to store object instances that specify the following details:
          \begin{itemize}
              \item The AR object that was placed.
              \item The \hyperref[def:sub_realm]{sub-realm}(s) in which the object instance will be shared.
              \item All data required to precisely reproduce the position and orientation of the object in real space.
          \end{itemize}

    \item The system must provide a way for a user to place AR objects in the space around them, creating object instances.
          \begin{enumerate}[align=left, label=\textbf{OP-FR2.\arabic*:}]
              \item The system must provide a way for a user to select an object either from their inventory or by generating a new object.
              \item The system must provide a way for the user to select one or more \hyperref[def:sub_realm]{sub-realms} of which they are a member.
              \item The system must provide a way for the user to position an AR object in real space.
          \end{enumerate}

    \item The system must limit the number of objects that a user can place
          \begin{enumerate}[align=left, label=\textbf{OP-FR3.\arabic*:}]
              \item The user must be limited in how many objects they can place in a given area
              \item The user must be limited in how many objects they can place over a given period
          \end{enumerate}

    \item The system must have an automated mechanism to retry the upload and storage of object instances when an initial attempt fails. \\
\end{enumerate}

\subsubsection{Realm Interface}
\label{ssub:realm_interface}

\begin{enumerate}[align=left, label=\textbf{RI-FR\arabic*:}]
    \item The Realm interface must provide a view of the AR object instances in the area around the user overlaid on their camera feed positioned in real space.
          \begin{enumerate}[align=left, label=\textbf{RI-FR1.\arabic*:}]
              \item AR object instances presented on the Realm interface must always appear to change perspective correctly with respect to the user’s camera position and angle.
              \item AR object instances that overlap or clutter a given area in real space will not be presented simultaneously.
              \item AR object instances reproduced on the Realm interface must be in the correct position and orientation as defined by the object placement.
          \end{enumerate}

    \item The Realm interface must keep track of a selected \hyperref[def:sub_realm]{sub-realm} which determines which object instances are displayed.
          \begin{enumerate}[align=left, label=\textbf{RI-FR2.\arabic*:}]
              \item The Realm interface must provide an indication of what \hyperref[def:sub_realm]{sub-realm} is currently selected.
              \item The Realm interface must provide a way for the user to change the selected \hyperref[def:sub_realm]{sub-realm}.
          \end{enumerate}

    \item The Realm interface must provide a control to allow the user to enter the object placement workflow.
    \item The Realm interface must provide a control to allow the user to enter the object scanning workflow.
    \item When the user is near the starting point of a tour, the Realm interface must provide an indication that there is a tour nearby as well as a link to the tour preview.
    \item The interface must display warnings to alert users of going outside during dangerous weather conditions. \\
    \item The interface must provide ample warning to users if they are close to a real-world hazard and on track to collide with it. \\
    \item The interface must provide an offline view without location syncing for interactive components. \\
\end{enumerate}


\subsubsection{Admin Interface}
\label{ssub:admin_interface}

\begin{enumerate}[align=left, label=\textbf{AI-FR\arabic*:}]
    \item The admin interface within the app shall only be available to \ref{def:admin}.
    \item The app shall provide an interface for \ref{def:admin} to carry-out their two special roles:
          \begin{enumerate}[align=left, label=\textbf{AI-FR2.\arabic*:}]
              \item The app shall permit \ref{def:admin} to act on user reports of an \ref{def:ar_obj} by either keeping or removing the \ref{def:ar_obj}.
              \item The app shall permit \ref{def:admin} to review requests for \ref{def:org_user} accounts and either accept or deny the request.
          \end{enumerate}
\end{enumerate}

\subsubsection{Profile Screen}
\label{ssub:profile_screen}
\begin{enumerate}[align=left, label=\textbf{PS-FR\arabic*:}]
    \item The user must enter valid credentials to be authenticated and logged into their account.
    \item The system shall provide a multi-factor authentication option for user accounts to enhance security. \\
    \item Allow the user to change their password.
    \item The user must be able to view their profile information including username, password, profile picture, and status.
    \item The user must be able to view all \hyperref[def:sub_realm]{sub-realms} that they are part of.
    \item The user shall be able to view a help page with frequently asked questions and additional help.
\end{enumerate}

\subsubsection{Sub-Realms}
\label{ssub:sub-realms}
\begin{enumerate}[align=left, label=\textbf{G-FR\arabic*:}]
    \item The user must be able to create a new \hyperref[def:sub_realm]{sub-realm} with a \hyperref[def:sub_realm]{sub-realm} name, description, and also allow inviting members.
    \item Allow users to add or remove members from a \hyperref[def:sub_realm]{sub-realm}.
    \item The user must be able to edit \hyperref[def:sub_realm]{sub-realm} settings, including \hyperref[def:sub_realm]{sub-realm} name and description.
    \item The user must be able to delete a \hyperref[def:sub_realm]{sub-realm}, and all associated data should be removed.
    \item All users within a \hyperref[def:sub_realm]{sub-realm} must be able to interact with \hyperref[def:sub_realm]{sub-realm}-specific AR content, including shared AR objects and experiences.
\end{enumerate}

\subsubsection{Friends screen}
\label{ssub:friends_screen}
\begin{enumerate}[align=left, label=\textbf{FS-FR\arabic*:}]
    \item The user shall be able to send friend requests to other users within the platform.
    \item The user must be able to accept or reject incoming friend requests.
    \item The user shall be able to view their list of friends.
    \item The user must be able to remove friends.
\end{enumerate}

\subsubsection{Settings}
\label{ssub:settings}
\begin{enumerate}[align=left, label=\textbf{S-FR\arabic*:}]
    \item The user shall be able to access and modify Accessibility Settings, including text size, enabling/disabling viewing of object names, and changing language.
    \item The user shall be able to adjust Display Settings, including light/dark mode, AR object visibility.
    \item The user must be able to manage Privacy Settings, including controlling who can see their profile, friends list, and AR interactions.
    \item Allow the user to manage Profile Settings, such as changing their username, password, profile picture, and status.
    \item The user must be able to access \hyperref[def:sub_realm]{sub-realm} Settings, as mentioned in the \ref{ssub:sub-realms} section of the requirements.
    \item The system must provide a setting to set the app theme to \textbf{Dark} or \textbf{Light}, with the option to automatically match the system theme \\
\end{enumerate}

\subsubsection{Databases}

\begin{enumerate}[align=left, label=\textbf{DB-FR\arabic*:}]
    \item All system databases must have an automated mechanism to periodically backup data \\
    \item The system must encrypt all user data stored using an \hyperref[def:encryption_standard]{encryption standard}. \\
\end{enumerate}

\subsection{Use Cases}
\label{sub:use_cases}

\begin{center}
    \includegraphics[scale=0.5]{use_cases.png}\\
    \textbf{Figure 1: Use Cases}
\end{center}

\begin{enumerate}[label=\textbf{UC\arabic*}]
    \item \label{uc:1} Complete the Tutorial \\
          \textbf{Actor}: \ref{def:user} \\
          \textbf{Pre-condition:} User has opened the app \\

          \textbf{Main Success Scenario:}
          \begin{enumerate}[label=\textbf{\arabic*.}]
              \item User creates an account (\ref{uc:23})
              \item System prompts user to complete the tutorial (\hyperref[ssub:tutorial]{TU-FR2})
              \item User indicates they would like to do the tutorial
              \item System opens the \textbf{Tutorial Interface}
              \item System provides directions to use a feature
              \item User tries the feature (\hyperref[ssub:tutorial]{TU-FR5})
              \item User goes to next feature when satisfied
              \item Repeat steps \textbf{5}-\textbf{7} for all major app features (\hyperref[ssub:tutorial]{TU-FR1})
              \item System prompts user to leave and end tutorial
              \item User ends tutorial (\hyperref[ssub:tutorial]{TU-FR3})
              \item System redirects to \textbf{Home Interface}
          \end{enumerate}

          \textbf{Secondary Scenarios:}
          \begin{itemize}
              \item[{\bf 1.1:}] User already has an account
                    \begin{enumerate}[label=\textbf{\arabic*.}]
                        \item User logs into account (\ref{uc:24})
                        \item User navigates to \textbf{Help Interface} (\hyperref[ssub:tutorial]{TU-FR4})
                        \item Go to step \textbf{2}
                    \end{enumerate}
              \item[{\bf 10.1:}] User chooses to stay in sandbox environment
                    \begin{enumerate}[label=\textbf{\arabic*.}]
                        \item System closes prompt and allows user to use sandbox environment
                        \item User indicates they want to leave tutorial
                        \item Go to step \textbf{11}
                    \end{enumerate}
          \end{itemize}
          \textbf{Success Postcondition:} The user has a general idea of how all the app’s major features work.

    \item \label{uc:2} Create a Tour \\
          \textbf{Actor}: \ref{def:org_user} \\
          \textbf{Pre-condition:} User is logged into the app \textbf{AND} User has navigated to the \textbf{Tour Management Interface} \\

          \textbf{Main Success Scenario:}
          \begin{enumerate}[label=\textbf{\arabic*.}]
              \item User selects the option to create a new tour
              \item System opens up a form for the user to add the following information about the tour (\hyperref[ssub:tour_management]{TM-FR4[1,2,5,6]}):
                    \begin{enumerate}
                        \item Name
                        \item Description
                        \item Price
                        \item Relevant web link(s)
                    \end{enumerate}
              \item User fills out the information
              \item System identifies no blank fields
              \item System gives the option to move on to the route configuration
              \item User goes to the route configuration
              \item System presents tour route configuration
              \item User configures tour area, route, and direction (\hyperref[ssub:tour_management]{TM-FR4[3]})
              \item System gives the option to move on to the inventory setup
              \item User goes to the inventory setup (\hyperref[ssub:tour_management]{TM-FR4[4]})
              \item System opens the \textbf{Inventory Interface}
              \item User scans and uploads a new \ref{def:ar_obj} (\ref{uc:5})
              \item User repeats step \textbf{12} until all desired \ref{def:ar_obj}s are uploaded
              \item User indicates they would like to move on to object placement
              \item System opens the \textbf{Realm Interface} and allows the user to place \ref{def:ar_obj}s and historical information popups in an isolated environment.
              \item User places down \ref{def:ar_obj} using the \textbf{Realm Interface} (\ref{uc:7})
              \item User repeats step \textbf{16} until all desired \ref{def:ar_obj}s are placed
              \item User indicates they would like to finish tour creation
              \item System gives the user the option create the tour as a draft or directly publish it (\hyperref[ssub:tour_management]{TM-FR2/TM-FR3})
              \item User selects direct publish option
              \item System uploads the tour data
              \item System makes the tour available to \ref{def:gen_user}
          \end{enumerate}

          \textbf{Secondary Scenarios:}
          \begin{itemize}
              \item[{\bf 4.1:}] System identifies some blank fields
                    \begin{enumerate}[label=\textbf{\arabic*.}]
                        \item System validates input to ensure all required fields are not empty
                        \item System determines input passes validation
                        \item Go to step \textbf{5}
                    \end{enumerate}
              \item[{\bf 4.1.2:}] System determines input fails validation
                    \begin{enumerate}[label=\textbf{\arabic*.}]
                        \item Go to step \textbf{3}
                    \end{enumerate}
              \item[{\bf 12.1:}] User does not have any \ref{def:ar_obj}s to scan
                    \begin{enumerate}[label=\textbf{\arabic*.}]
                        \item Go to step \textbf{14}
                    \end{enumerate}
              \item[{\bf 20.1:}] User selects draft option
                    \begin{enumerate}[label=\textbf{\arabic*.}]
                        \item System uploads the tour data but does not make it available to \ref{def:gen_user}
                        \item User decides at a later time to publish the draft tour
                        \item Go to step \textbf{22}
                    \end{enumerate}
          \end{itemize}
          \textbf{Success Postcondition:} One of the \ref{def:org_user} has a new tour linked to their organization

    \item \label{uc:3} Edit an existing Tour \\
          \textbf{Actor}: \ref{def:org_user} \\
          \textbf{Pre-condition:} User is logged into the app \textbf{AND} User has at least one tour connected to their organization \textbf{AND} User has navigated to the \textbf{Tour Management Interface} \\

          \textbf{Main Success Scenario:}
          \begin{enumerate}[label=\textbf{\arabic*.}]
              \item System provides a list of all tours connected to the user’s organization
              \item User selects one of the tours
              \item System shows the preview of the tour (\hyperref[ssub:tour_management]{TM-FR5})
              \item System provides the option to edit the tour (\hyperref[ssub:tour_management]{TM-FR6})
              \item User decides to edit the tour
              \item System opens up a form for the user to with the following information about the tour that may or not be populated (\hyperref[ssub:tour_management]{TM-FR4[1,2,5,6]}):
                    \begin{enumerate}
                        \item Name
                        \item Description
                        \item Price
                        \item Relevant web link(s)
                    \end{enumerate}
              \item User edits information as desired
              \item System identifies no blank fields
              \item System gives the option to move on to the route configuration
              \item User goes to the route configuration
              \item System presents tour route configuration
              \item User edits configuration of tour area, route, and direction as desired (\hyperref[ssub:tour_management]{TM-FR4[3]})
              \item System gives the option to move on to the inventory setup
              \item User goes to the inventory setup (\hyperref[ssub:tour_management]{TM-FR4[4]})
              \item System opens the \textbf{Inventory Interface}
              \item User scans and uploads a new \ref{def:ar_obj} (\ref{uc:5})
              \item User repeats step \textbf{12} until all desired \ref{def:ar_obj}s are uploaded
              \item User indicates they would like to move on to object placement
              \item System opens the \textbf{Realm Interface} and allows the user to place/edit \ref{def:ar_obj}s and historical information popups in an isolated environment.
              \item User edits \ref{def:ar_obj} using the \textbf{Realm Interface} (\ref{uc:7})
              \item User repeats step \textbf{16} until all desired \ref{def:ar_obj}s are edited
              \item User indicates they would like to finish tour editing
              \item System gives the user the option to save the edited tour as a draft or directly publish it (\hyperref[ssub:tour_management]{TM-FR2/TM-FR3})
              \item User selects direct publish option
              \item System uploads the updated tour data
              \item System makes the updated tour available to \ref{def:gen_user}
          \end{enumerate}

          \textbf{Secondary Scenarios:}
          \begin{itemize}
              \item[{\bf 5.1:}] User decides not to edit the tour
                    \begin{enumerate}[label=\textbf{\arabic*.}]
                        \item System goes back to the \textbf{Tour Management Interface}
                    \end{enumerate}
              \item[{\bf 8.1:}] System identifies some blank fields
                    \begin{enumerate}[label=\textbf{\arabic*.}]
                        \item System validates input to ensure all required fields are not empty
                        \item System determines input passes validation
                        \item Go to step \textbf{9}
                    \end{enumerate}
              \item[{\bf 16.1:}] User decides not to scan and upload any new \ref{def:ar_obj}s
                    \begin{enumerate}[label=\textbf{\arabic*.}]
                        \item Go to step \textbf{18}
                    \end{enumerate}
              \item[{\bf 20.1:}] User places down new \ref{def:ar_obj} using the \textbf{Realm Interface}
                    \begin{enumerate}[label=\textbf{\arabic*.}]
                        \item User repeats step \textbf{20.1} until all desired \ref{def:ar_obj}s are placed
                    \end{enumerate}
              \item[{\bf 24.1:}] User selects draft option
                    \begin{enumerate}[label=\textbf{\arabic*.}]
                        \item System uploads the updated tour data but does not make it available to \ref{def:gen_user}
                        \item User decides at a later time to publish the draft of the updated tour
                        \item Go to step \textbf{26}
                    \end{enumerate}
          \end{itemize}
          \textbf{Success Postcondition:} One of the \ref{def:org_user} has edited an existing tour linked to their organization

    \item \label{uc:4} Go on a tour \\
          \textbf{Actor}: \ref{def:gen_user} \\
          \textbf{Pre-condition:} User is logged into the app \\

          \textbf{Main Success Scenario:}
          \begin{enumerate}[label=\textbf{\arabic*.}]
              \item User navigates to the \textbf{Tour Interface}
              \item System displays a list of all tours
              \item User selects a tour (\hyperref[ssub:touring]{TR-FR2.1})
              \item System opens up the preview of the selected tour (\hyperref[ssub:touring]{TR-FR3})
              \item System also provides the option to take the tour
              \item User reviews details of the tour
              \item User decides to go on the tour
              \item System opens a modified version of the \textbf{Realm Interface} in an isolated environment that does not allow them to place any objects or modify existing objects (\hyperref[ssub:touring]{TR-FR4.2})
              \item System shows the intended direction of travel
              \item User walks to the highlighted \ref{def:ar_obj} of interest
              \item System acknowledges when the user makes it to the \ref{def:ar_obj}
              \item System provides historical information regarding \ref{def:ar_obj}
              \item System updates the target to be the next \ref{def:ar_obj}
              \item Repeat steps \textbf{10}-\textbf{13} for every \ref{def:ar_obj}
              \item System indicates that the tour has been completed
              \item System provides metrics for time taken and distance traveled
              \item System prompts user to leave tour
              \item User leaves tour
              \item System goes back to the tour list
          \end{enumerate}

          \textbf{Secondary Scenarios:}
          \begin{itemize}
              \item[{\bf 1.1:}] System pushes a notification to the user when they are in the proximity of a tour (\hyperref[ssub:touring]{TR-FR2.2})
                    \begin{enumerate}[label=\textbf{\arabic*.}]
                        \item System opens up the \textbf{Tour Interface}
                        \item Go to step \textbf{4}
                    \end{enumerate}
              \item[{\bf 1.2:}] User scans a tour QR code in camera app (\hyperref[ssub:touring]{TR-FR2.3})
                    \begin{enumerate}[label=\textbf{\arabic*.}]
                        \item System opens up the \textbf{Tour Interface}
                        \item Go to step \textbf{4}
                    \end{enumerate}
              \item[{\bf 7.1:}] User decides not to go on the tour
                    \begin{enumerate}[label=\textbf{\arabic*.}]
                        \item Go to step \textbf{19}
                    \end{enumerate}
              \item[{\bf 10.1:}] User indicates they want to open the map
                    \begin{enumerate}[label=\textbf{\arabic*.}]
                        \item System opens up the \textbf{Map} view of the tour (\hyperref[ssub:touring]{TR-FR4.2})
                        \item System shows the user’s current location
                        \item System shows the indented route, direction and boundaries of the tour
                        \item System marks the location of \ref{def:ar_obj}s
                        \item User views the map
                        \item User closes the map
                        \item Go to step \textbf{9}
                    \end{enumerate}
              \item[{\bf 14.1:}] User does not want to go to all \ref{def:ar_obj}s
                    \begin{enumerate}[label=\textbf{\arabic*.}]
                        \item User indicates they wish to exit the tour
                        \item Go to step \textbf{19}
                    \end{enumerate}
              \item[{\bf 18.1:}] User does not leave tour
                    \begin{enumerate}[label=\textbf{\arabic*.}]
                        \item System removes target object and allows user to explore \ref{def:ar_obj}s in the isolated environment
                        \item User eventually decides to exit the tour
                        \item Go to step \textbf{19}
                    \end{enumerate}
          \end{itemize}
          \textbf{Success Postcondition:} One of the \ref{def:gen_user} has completed a tour

    \item \label{uc:5} Scan and upload \ref{def:ar_obj} to the user inventory \\
          \textbf{Actor}: \ref{def:gen_user} \\
          \textbf{Pre-condition:} User has an account on the application \textbf{AND} is logged into the application \textbf{AND} navigates to the Realm display \\

          \textbf{Main Success Scenario:}
          \begin{enumerate}[label=\textbf{\arabic*.}]
              \item System displays Realm screen to user
              \item User selects control to scan and create an \ref{def:ar_obj}
              \item System presents user with the option to scan a 2D or 3D \ref{def:ar_obj}
              \item User selects one of the provided options
                    \begin{enumerate}[label=(\alph*)]
                        \item User selects 2D \ref{def:ar_obj} option
                        \item User selects 3D \ref{def:ar_obj} option
                    \end{enumerate}
              \item System displays a scanning interface
              \item User performs motions to scan the targeted object
              \item System provides a render for the scanned portion (happens concurrently with step \textbf{6})
              \item User selects the option to confirm that the scanning process is complete
              \item System navigates user to an editor with the complete render of the target entity
              \item User performs actions to edit the render to meet their needs
                    \begin{enumerate}[label=(\alph*)]
                        \item User removes unneeded parts of the render
                        \item User changes the colour of selected portions
                    \end{enumerate}
              \item System applies the changes to the \ref{def:ar_obj} intended by the user
              \item User confirms that the editing process is completed
              \item System receives the user confirmation and provides a preview of the final \ref{def:ar_obj} with the options to discard or add it to their inventory
              \item User selects what they would want to do with the new \ref{def:ar_obj}
                    \begin{enumerate}[label=(\alph*)]
                        \item User returns to the editing interface for further customization (Go to step \textbf{10})
                        \item User selects control to add the object to their inventory
                        \item User wants to discard the object (Go to Secondary Scenario 2)
                    \end{enumerate}
              \item System adds the new \ref{def:ar_obj} to the User's inventory and notifies the User about adding the object to their inventory
              \item System returns to the Realm screen
          \end{enumerate}

          \textbf{Secondary Scenarios:}
          \begin{itemize}
              \item[\textbf{\textbf{(3.1, 5.1)}}] User cancels scanning process
                    \begin{enumerate}[label=\textbf{\arabic*.}]
                        \item Main scenario steps \textit{1-3 or 1-5}
                        \item User selects option to cancel the scanning process
                        \item System receives User's request to cancel scanning process and returns to the Realm screen
                    \end{enumerate}

              \item[9.1] User discards the scanned \ref{def:ar_obj}
                    \begin{enumerate}[label=\textbf{\arabic*.}]
                        \item Main scenario steps \textit{1-9}
                        \item User selects option to discard the scanned object
                        \item System receives User's request and presents confirmation prompt for discarding object
                        \item User interacts with confirmation prompt
                              \begin{enumerate}[label=(\alph*)]
                                  \item User selects to discard object
                                  \item User selects to not discard object (Go back to Main scenario step \textbf{13})
                              \end{enumerate}
                        \item System receives User response and deletes object and returns to the Realm screen
                    \end{enumerate}

          \end{itemize}
          \textbf{Success Postcondition:} The User can view the new object in their inventory

    \item \label{uc:6} Generate a custom \ref{def:ar_obj} via a user prompt \\
          \textbf{Actor}: \ref{def:gen_user} \\
          \textbf{Pre-condition:} User has an account on the application \textbf{AND} is logged in to the app \textbf{AND} has navigated to the \textbf{Realm Interface} \\

          \textbf{Main Success Scenario:}
          \begin{enumerate}[label=\textbf{\arabic*.}]
              \item User is present in the Realm Interface and performs actions to navigate to the object prompt generation interface
              \item System presents the object prompt generation interface
              \item User enters their desired prompt and confirms that the prompt is complete
              \item System checks the prompt for profanity
                    \begin{enumerate}[label=(\alph*)]
                        \item If the prompt contains profanity, system notifies user that the prompt is invalid due to profanity and requests the user to reenter their prompt (Go to step \textbf{2})
                        \item If the prompt does not contain profanity, system accepts the prompt and generates objects
                    \end{enumerate}
              \item User selects from an array of objects provided by the system
              \item System provides the options to \textbf{preview the object}, \textbf{add it to user inventory} and \textbf{place object in the Realm Interface}
              \item User selects one of the options
                    \begin{enumerate}[label=(\alph*)]
                        \item User selects the options to preview the object
                        \item User selects to add object to their inventory
                        \item User selects option to place object in the Realm Interface
                    \end{enumerate}
              \item System performs the action selected by the user
                    \begin{enumerate}[label=(\alph*)]
                        \item User selects option to preview object
                              \begin{enumerate}
                                  \item System opens the \ref{def:ar_obj} in the Preview Interface
                                  \item User performs actions on the \ref{def:ar_obj}
                                        \begin{itemize}
                                            \item User rotates the object to view different angles
                                            \item User zooms in and out to view object details
                                        \end{itemize}
                                  \item System provides the view of the \ref{def:ar_obj} according to the User's performed actions
                              \end{enumerate}
                        \item User selects option to add object to inventory
                              \begin{enumerate}
                                  \item System adds object to user inventory
                              \end{enumerate}
                        \item User selects option to place object in the Realm space (Go to \ref{uc:7} Main Success Scenario)
                    \end{enumerate}
          \end{enumerate}

          \textbf{Secondary Scenarios:}
          \begin{itemize}
              \item[\textbf{(2.1, 4(b).1)}] User cancels the \ref{def:ar_obj} generation
                    \begin{enumerate}[label=\textbf{\arabic*.}]
                        \item Main scenario steps \textit{1-2 or 1-4(b)}
                        \item User selects control to cancel object generation via prompt
                        \item System receives user request and returns to the Realm interface
                    \end{enumerate}

              \item[\textbf{(2.1, 4(a).1)}] User reenters prompt due to presence of profanity
                    \begin{enumerate}[label=\textbf{\arabic*.}]
                        \item Main scenario steps \textit{1-2 or 1-4(a)}
                        \item User is notified of profanity in the prompt and reenters their prompt
                        \item Go to Main scenario step 4
                        \item Continues Main Scenario from step 5
                    \end{enumerate}

              \item[\textbf{(2.1, 4(b).1)}] User reenters prompt for a different variation
                    \begin{enumerate}[label=\textbf{\arabic*.}]
                        \item Main scenario steps \textit{1-2 or 1-4(b)}
                        \item User reenters their prompt
                        \item System validates user prompt (Go to Main scenario step \textbf{4})
                        \item Continues Main Scenario from step \textbf{5}
                    \end{enumerate}

          \end{itemize}
          \textbf{Success Postcondition:} The User can view the new object in their inventory
    \item \label{uc:7} Place an AR object from the \textbf{Realm Interface} (\hyperref[ssub:realm_interface]{RI-FR1}) \\
          \textbf{Actor}: General User \\
          \textbf{Pre-condition:} User is logged in to the app \textbf{AND} has navigated to the \textbf{Realm Interface} (\hyperref[ssub:realm_interface]{RI-FR1}) \\

          \textbf{Main Success Scenario:}
          \begin{enumerate}[label=\textbf{\arabic*.}]
              \item System presents \textbf{Realm interface}.
              \item User performs control to initiate object placement.
              \item System presents the \textbf{Object Selection Menu} (\hyperref[ssub:object_placement]{OP-FR2.1}).
              \item User selects an existing AR object for placement.
              \item System presents the \textbf{Sub-Realm Selection Menu} (\hyperref[ssub:object_placement]{OP-FR2.2}).
              \item User selects the \hyperref[def:sub_realm]{sub-realm}(s) in which they wish for their object instance to be shared.
              \item System presents the \textbf{Object Positioning Interface} (\hyperref[ssub:object_placement]{OP-FR2.3}).
              \item User positions the AR object using given controls and confirms when done.
              \item System returns to Realm screen.
          \end{enumerate}

          \textbf{Secondary Scenarios:}
          \begin{itemize}
              \item[{\bf 3.1:}] User creates new object with a prompt:
                    \begin{enumerate}[label=\textbf{\arabic*.}]
                        \item Main scenario steps 1-3.
                        \item User selects the option to generate an object via prompt.
                        \item System presents the \textbf{Prompt Object Generation Interface}.
                        \item User completes the prompt object generation workflow.
                        \item Main scenario steps 5-9.
                    \end{enumerate}

              \item[{\bf 3.2:}] User creates a new object with object scanning:
                    \begin{enumerate}[label=\textbf{\arabic*.}]
                        \item Main scenario steps 1-3.
                        \item User selects the option to generate an object via object scan.
                        \item System presents the \textbf{Object Scanning Interface}.
                        \item User completes object scanning workflow.
                        \item Main scenario steps 5-9.
                    \end{enumerate}

              \item[{\bf 3.3, 5.1, 7.1:}] User cancels object placement:
                    \begin{enumerate}[label=\textbf{\arabic*.}]
                        \item Main scenario steps 1-3, or 1-5, or 1-7.
                        \item User selects option to cancel object placement.
                        \item System returns to the \textbf{Realm Interface}.
                    \end{enumerate}

              \item[{\bf 5.1:}] User reselects object:
                    \begin{enumerate}[label=\textbf{\arabic*.}]
                        \item Main scenario steps 1-5.
                        \item User selects option to return to \textbf{Object Selection Menu} (\hyperref[ssub:object_placement]{OP-FR2.1}).
                        \item Main scenario resumes from step 3.
                    \end{enumerate}

              \item[{\bf 7.2:}] User reselects \hyperref[def:sub_realm]{sub-realms}:
                    \begin{enumerate}[label=\textbf{\arabic*.}]
                        \item Main scenario steps 1-7.
                        \item User selects option to return to \textbf{Sub-Realm Selection Menu} (\hyperref[ssub:object_placement]{OP-FR2.2}).
                        \item Main scenario resumes from step 5.
                    \end{enumerate}
          \end{itemize}

          \textbf{Success Postcondition:} Users that are members of the \hyperref[def:sub_realm]{sub-realm} in which the object instance has been shared can see the object instance from the \textbf{Realm interface}.

    \item \label{uc:8} Place an AR object from the Inventory \\
          \textbf{Actor}: General User \\
          \textbf{Pre-condition:} User is logged in to the app \textbf{AND} has navigated to the \textbf{Inventory Interface} \\

          \textbf{Main Success Scenario:}
          \begin{enumerate}[label=\textbf{\arabic*.}]
              \item System presents Inventory interface.
              \item User selects an object they wish to place.
              \item System presents object details.
              \item User selects option to initiate object placement.
              \item System presents the \textbf{Sub-Realm Selection Menu} (\hyperref[ssub:object_placement]{OP-FR2.1}).
              \item User selects the \hyperref[def:sub_realm]{sub-realm}(s) in which they wish for their object instance to be shared.
              \item System presents the \textbf{Object Positioning Interface} (\hyperref[ssub:object_placement]{OP-FR2.3}).
              \item User positions the AR object using given controls and confirms when done.
              \item System returns to Realm screen.
          \end{enumerate}

          \textbf{Secondary Scenarios:}
          \begin{itemize}
              \item[{\bf 5.1, 7.1:}] User cancels object placement:
                    \begin{enumerate}[label=\textbf{\arabic*.}]
                        \item Main scenario steps 1-5, or 1-7.
                        \item User selects option to cancel object placement.
                        \item System returns to the Inventory screen.
                    \end{enumerate}

              \item[{\bf 7.2:}] User reselects \hyperref[def:sub_realm]{sub-realm}:
                    \begin{enumerate}[label=\textbf{\arabic*.}]
                        \item Main scenario steps 1-7.
                        \item User selects option to return to \textbf{Sub-Realm Selection} (\hyperref[ssub:object_placement]{OP-FR2.2}).
                        \item Main scenario resumes from step 5.
                    \end{enumerate}
          \end{itemize}

          \textbf{Success Postcondition:} Users that are members of the \hyperref[def:sub_realm]{sub-realm} in which the object instance has been shared can see the object instance from the \textbf{Realm Interface}.

    \item \label{uc:9} Preview AR objects from the Inventory \\
          \textbf{Actor}: \ref{def:gen_user} \\
          \textbf{Pre-condition:} User has an account on the system \textbf{AND} is logged into the system \\

          \textbf{Main Success Scenario:}
          \begin{enumerate}[label=\textbf{\arabic*.}]
              \item User navigates to their profile view
              \item System presents the User's profile and inventory
              \item User selects controls to view their entire inventory
              \item System provides complete view of the inventory
              \item User selects an \ref{def:ar_obj} they would like to view
              \item System provides the option to preview the \ref{def:ar_obj}
              \item User selects the option to preview the \ref{def:ar_obj}
              \item System opens the \ref{def:ar_obj} in the Preview Interface
              \item User performs actions on the \ref{def:ar_obj}
                    \begin{enumerate}[label=(\alph*)]
                        \item User rotates the object to view different angles
                        \item User zooms in and out to view object details
                    \end{enumerate}
              \item System provides the view of the \ref{def:ar_obj} according to the User's performed actions
          \end{enumerate}

          \textbf{Secondary Scenarios:}
          N/A

          \textbf{Success Postcondition:} User is able to view an \ref{def:ar_obj} in their inventory from different angles


    \item \label{uc:10} View object instances that are shared in a \hyperref[def:sub_realm]{sub-realm} in the \textbf{Realm interface} \\
          \textbf{Actor}: General User \\
          \textbf{Pre-condition:} User is logged in to the app \textbf{AND} has navigated to the \textbf{Realm Interface} (\hyperref[ssub:realm_interface]{RI-FR1}) \\

          \textbf{Main Success Scenario:}
          \begin{enumerate}[label=\textbf{\arabic*.}]
              \item System presents the \textbf{Realm Interface}.
              \item User selects a \hyperref[def:sub_realm]{sub-realm} to see objects from.
              \item System only presents object instances in the selected \hyperref[def:sub_realm]{sub-realm}.
              \item User moves their device to see any angle of the object instance.
              \item System continuously presents the object instance with the correct perspective relative to the user’s camera.
          \end{enumerate}

          \textbf{Secondary Scenarios:}
          \begin{itemize}
              \item[{\bf 10.1:}] User is not a member of any \hyperref[def:sub_realm]{sub-realms}:
                    \begin{enumerate}[label=\textbf{\arabic*.}]
                        \item System presents \textbf{Realm Interface} (\hyperref[ssub:realm_interface]{RI-FR1}).
                        \item System presents only public object instances.
                        \item User moves their device to see any angle of the object instance.
                        \item System continuously presents the object instance with the correct perspective relative to the user’s camera.
                    \end{enumerate}
          \end{itemize}

          \textbf{Success Postcondition:} The user can see any AR object instances that are in any \hyperref[def:sub_realm]{sub-realm} that they are a part of correctly positioned in real space.

    \item \label{uc:11} Check the map to see where objects are placed in the world \\
          \textbf{Actor}: \ref{def:gen_user} \\
          \textbf{Pre-condition:} User has an account on the application \textbf{AND} is logged in to the app \\

          \textbf{Main Success Scenario:}
          \begin{enumerate}[label=\textbf{\arabic*.}]
              \item User performs actions to navigate to the Map view
              \item System presents the Maps Interface
              \item User observes location markers with the count of the objects and selects a marker
              \item System shows information and options for the marker
                    \begin{enumerate}[label=(\alph*)]
                        \item System displays number of visits to the marker, the number of objects available and the area covered by the marker
                        \item System provides the options to receive directions for the selected marker (Go to \ref{uc:12})
                    \end{enumerate}
          \end{enumerate}

          \textbf{Secondary Scenarios:}
          N/A

          \textbf{Success Postcondition:} User is able to view cluster of objects in a confined area on the Maps Interface

    \item \label{uc:12}  Provide directions to a selected group of objects \\
          \textbf{Actor}: \ref{def:gen_user} \\
          \textbf{Pre-condition:} User has an account on the application \textbf{AND} is logged in to the app \textbf{AND} is present on the Maps Interface \\

          \textbf{Main Success Scenario:}
          \begin{enumerate}[label=\textbf{\arabic*.}]
              \item User selects a location marker on the Map Interface
              \item System shows information about the marker (see \ref{uc:11} Main scenario step \textit{4(a)}) and the option to provide directions for the marker
              \item User selects the option to get directions for the location markers
              \item System provides a set of instructions to direct the user to the marker
              \item User views the instructions and follows the directions provided
              \item System tracks User location to check if they have arrived at the marker location (happens concurrently with step 5)
              \item User arrives at the selected marker
              \item System is informed that the user has arrived at the marker location (based on device location) and notifies the User of their arrival
          \end{enumerate}

          \textbf{Secondary Scenarios:}
          \begin{itemize}
              \item[{\textbf{6.1}}] User ends navigation to marker before reaching the marker
                    \begin{enumerate}[label=\textbf{\arabic*.}]
                        \item Main Sceanrio step \textit{1-6}
                        \item User selects option to end navigation to the marker
                        \item System provides a confirmation prompt for ending the navigation
                        \item User responds to the confirmation prompt
                              \begin{enumerate}[label=(\alph*)]
                                  \item User selects option to confirm terminating navigation
                                  \item User selects option to deny terminating the navigation and continues navigation (Go to Main scenario step \textit{5})
                              \end{enumerate}
                        \item System ends navigation and returns to the Map Interface
                    \end{enumerate}
          \end{itemize}
          \textbf{Success Postcondition:} The User is provided the directions to their chosen marker and have arrived at the marker location

    \item \label{uc:13} \textbf{Change App Settings} \\
          \textbf{Actor}: \ref{def:gen_user} \\
          \textbf{Precondition:} User is logged into the app.

          \textbf{Main Success Scenario:}
          \begin{enumerate}[label=\textbf{\arabic*.}]
              \item User navigates to the settings menu from the app’s main interface.
              \item System presents the Settings Menu with various categories (e.g., Accessibility, Display, Privacy).
              \item User selects a specific settings category to modify.
              \item System presents the options within the selected category.
              \item User modifies the desired settings (e.g., adjusts brightness, changes privacy preferences).
              \item User confirms the changes.
              \item System saves the changes locally and within the cloud, and updates the settings.
              \item System returns to the main settings screen.
          \end{enumerate}

          \textbf{Secondary Scenarios:}
          \begin{itemize}
              \item[{}] \textbf{User cancels changes}
                    \begin{enumerate}[label=\textbf{\arabic*.}]
                        \item Main scenario 1-5
                        \item User selects the option to cancel changes before confirming.
                        \item System discards the changes and returns to the main settings screen.
                    \end{enumerate}

              \item[{}] \textbf{User reselects settings category:}
                    \begin{enumerate}[label=\textbf{\arabic*.}]
                        \item Main scenario 1-4
                        \item User selects the option to return to the Settings Menu and choose another category.
                        \item Main scenario resumes from step 3.
                    \end{enumerate}

              \item[{}] \textbf{User has no network connection:}
                    \begin{enumerate}[label=\textbf{\arabic*.}]
                        \item Main scenario 1-7
                        \item System detects that the network is unavailable.
                        \item System queues the changes for synchronization once the network is reconnected.
                        \item Main scenario resumes from step 8 when the network is restored.
                    \end{enumerate}
          \end{itemize}
          \textbf{Success Postcondition:} The user’s settings are successfully updated locally and, when connected, synchronized to the cloud.

    \item \label{uc:14} Report an existing AR object \\
          \textbf{Actor}: General User \\
          \textbf{Pre-condition:} User is logged in to the app \textbf{AND} has navigated to the \textbf{Realm Interface} (\hyperref[ssub:realm_interface]{RI-FR1}) \textbf{AND} there is another user’s AR object placed in the vicinity of the user \\

          \textbf{Main Success Scenario:}
          \begin{enumerate}[label=\textbf{\arabic*.}]
              \item System presents \textbf{Realm Interface} with another user’s object instance present on screen.
              \item User selects the object they wish to report from the screen.
              \item System presents \textbf{Object Selection Context Action Menu}.
              \item User selects option to report object.
              \item System presents \textbf{Object Reporting Interface}.
              \item User provides required details for report and submits report.
              \item System presents message thanking user for report.
          \end{enumerate}

          \textbf{Secondary Scenarios:}
          \begin{itemize}
              \item[{\bf 5.1:}] User cancels report:
                    \begin{enumerate}[label=\textbf{\arabic*.}]
                        \item User closes \textbf{Object Reporting Interface}.
                        \item System returns to \textbf{Realm Interface}.
                    \end{enumerate}

              \item[{\bf 3.1:}] User cancels object selection:
                    \begin{enumerate}[label=\textbf{\arabic*.}]
                        \item User closes \textbf{Object Selection Context Action Menu}.
                        \item Main scenario resumes from step 2.
                    \end{enumerate}
          \end{itemize}

          \textbf{Success Postcondition:} The user is informed that their report will be investigated. The system initiates a report review process.

    \item \label{uc:15} Add reaction to an existing AR object instance \\
          \textbf{Actor}: General User \\
          \textbf{Pre-condition:} User is logged in to the app \textbf{AND} has navigated to the \textbf{Realm Interface} (\hyperref[ssub:realm_interface]{RI-FR1}) \textbf{AND} there is an AR object placed in the vicinity of the user \\

          \textbf{Main Success Scenario:}
          \begin{enumerate}[label=\textbf{\arabic*.}]
              \item System presents \textbf{Realm Interface} with an object instance present on screen.
              \item User selects the object they wish to react to from the screen.
              \item System presents \textbf{Object Selection Context Action Menu}.
              \item User selects option to react to object instance.
              \item System presents \textbf{Reaction Selection Menu}.
              \item User selects desired reaction.
              \item System returns to \textbf{Object Selection Context Action Menu} with user reaction displayed.
          \end{enumerate}

          \textbf{Secondary Scenarios:}
          \begin{itemize}
              \item[{\bf 3.1:}] User cancels object selection:
                    \begin{enumerate}[label=\textbf{\arabic*.}]
                        \item User closes \textbf{Object Selection Context Action Menu}.
                        \item Main scenario resumes from step 2.
                    \end{enumerate}
          \end{itemize}

          \textbf{Success Postcondition:} The user’s reaction is visible to any user who selects the AR object instance.

    \item \label{uc:16} Add other user’s placed AR object to Inventory \\
          \textbf{Actor}: General User \\
          \textbf{Pre-condition:} User is logged in to the app \textbf{AND} has navigated to the \textbf{Realm Interface} (\hyperref[ssub:realm_interface]{RI-FR1}) \textbf{AND} there are other users’ AR objects placed in the vicinity of the user \\

          \textbf{Main Success Scenario:}
          \begin{enumerate}[label=\textbf{\arabic*.}]
              \item System presents \textbf{Realm Interface} with another user’s object instance present on screen.
              \item User selects the object they wish to add to their inventory from the screen.
              \item System presents \textbf{Object Selection Context Action Menu}.
              \item User selects option to add object to inventory.
              \item System returns to \textbf{Realm Interface} with indication that object has been added to Inventory.
          \end{enumerate}

          \textbf{Secondary Scenarios:}
          \begin{itemize}
              \item[{\bf 4.1:}] User inventory is full:
                    \begin{enumerate}[label=\textbf{\arabic*.}]
                        \item System returns to \textbf{Realm Interface} with indication that Inventory is full.
                    \end{enumerate}

              \item[{\bf 3.1:}] User cancels object selection:
                    \begin{enumerate}[label=\textbf{\arabic*.}]
                        \item User exits \textbf{Object Selection Context Action Menu}.
                        \item System returns to \textbf{Realm Interface}.
                    \end{enumerate}
          \end{itemize}

          \textbf{Success Postcondition:} The user’s desired object is available in the user’s inventory for preview and placement.


    \item \label{uc:17} \textbf{Create a \ref{def:sub_realm}} \\
          \textbf{Actor}: \ref{def:gen_user} \\
          \textbf{Precondition:} User is logged into the app.

          \textbf{Main Success Scenario:}
          \begin{enumerate}[label=\textbf{\arabic*.}]
              \item User navigates to the "Create \ref{def:sub_realm}" page from the app’s main interface.
              \item System presents the \ref{def:sub_realm} Creation Page.
              \item User provides a name and description for the \hyperref[def:sub_realm]{sub-realm}.
              \item System presents the option to “Add users to \hyperref[def:sub_realm]{sub-realm}.”
              \item User selects the option to add users.
              \item System displays the user’s friend list.
              \item User selects the friends they want to add to the \hyperref[def:sub_realm]{sub-realm}.
              \item User confirms that they are done adding friends.
              \item User confirms creation of the \hyperref[def:sub_realm]{sub-realm}.
              \item System saves the \hyperref[def:sub_realm]{sub-realm} with the provided information and friends list.
              \item System returns to the Realm interface, displaying the newly created \hyperref[def:sub_realm]{sub-realm}.
          \end{enumerate}

          \textbf{Secondary Scenarios:}
          \begin{itemize}
              \item[{}] \textbf{User cancels the creation before confirming:}
                    \begin{enumerate}[label=\textbf{\arabic*.}]
                        \item Main scenario 1-8
                        \item User selects the option to cancel \hyperref[def:sub_realm]{sub-realm} creation.
                        \item System discards the \hyperref[def:sub_realm]{sub-realm} creation process and returns to the Realm interface.
                    \end{enumerate}

              \item[{}] \textbf{User reselects friends:}
                    \begin{enumerate}[label=\textbf{\arabic*.}]
                        \item Main scenario 1-7
                        \item User selects the option to go back to the friends list.
                        \item Main scenario resumes from step 6.
                    \end{enumerate}

              \item[{}] \textbf{User does not add any friends:}
                    \begin{enumerate}[label=\textbf{\arabic*.}]
                        \item Main scenario 1-5
                        \item User selects the option to skip adding friends.
                        \item Main scenario resumes from step 9.
                    \end{enumerate}
          \end{itemize}

          \textbf{Success Postcondition:} A new \hyperref[def:sub_realm]{sub-realm} is created and linked to the user’s account with the chosen settings and added friends.

    \item \label{uc:18} \textbf{Leave a \ref{def:sub_realm}} \\
          \textbf{Actor}: \ref{def:gen_user} \\
          \textbf{Precondition:} User is logged into the app AND is a member of the \hyperref[def:sub_realm]{sub-realm}.

          \textbf{Main Success Scenario:}
          \begin{enumerate}[label=\textbf{\arabic*.}]
              \item User navigates to the \hyperref[def:sub_realm]{sub-realm} and opens the \ref{def:sub_realm} Settings.
              \item System presents the \ref{def:sub_realm} Settings Menu.
              \item User selects the "Leave \ref{def:sub_realm}" option.
              \item System prompts the user to confirm the action.
              \item User confirms that they wish to leave the \hyperref[def:sub_realm]{sub-realm}.
              \item System removes the user from the \hyperref[def:sub_realm]{sub-realm}.
              \item System returns the user to the Realm Interface, with the \hyperref[def:sub_realm]{sub-realm} no longer visible.
          \end{enumerate}

          \textbf{Secondary Scenarios:}
          \begin{itemize}
              \item[{}] \textbf{User cancels leaving the \hyperref[def:sub_realm]{sub-realm}:}
                    \begin{enumerate}[label=\textbf{\arabic*.}]
                        \item Main scenario 1-4
                        \item User selects the option to cancel leaving the \hyperref[def:sub_realm]{sub-realm}.
                        \item System returns to the \ref{def:sub_realm} Settings without making any changes.
                    \end{enumerate}
          \end{itemize}

          \textbf{Success Postcondition:} The user is no longer part of the \hyperref[def:sub_realm]{sub-realm}, and the \hyperref[def:sub_realm]{sub-realm} is no longer accessible or visible to the user.

    \item \label{uc:19} \textbf{Join an Existing\ref{def:sub_realm}} \\
          \textbf{Actor}: \ref{def:gen_user} \\
          \textbf{Precondition:} User is logged into the app AND has received an invite to the \hyperref[def:sub_realm]{sub-realm}.

          \textbf{Main Success Scenario:}
          \begin{enumerate}[label=\textbf{\arabic*.}]
              \item User receives a notification of a \hyperref[def:sub_realm]{sub-realm} invitation.
              \item System presents the \hyperref[def:sub_realm]{sub-realm} invitation details.
              \item User selects the \hyperref[def:sub_realm]{sub-realm} they wish to join.
              \item System presents a confirmation prompt.
              \item User confirms their request to join the \hyperref[def:sub_realm]{sub-realm}.
              \item System adds the user to the selected \hyperref[def:sub_realm]{sub-realm}.
              \item System returns to the Realm Interface, now displaying the newly joined \hyperref[def:sub_realm]{sub-realm}.
          \end{enumerate}

          \textbf{Secondary Scenarios:}
          \begin{itemize}
              \item[{}] \textbf{Sub-realm is invite-only, and approval is required:}
                    \begin{enumerate}[label=\textbf{\arabic*.}]
                        \item Main scenario 1-5
                        \item System notifies the user that their request to join is pending approval.
                        \item User must wait for an admin’s approval before being added to the \hyperref[def:sub_realm]{sub-realm}.
                    \end{enumerate}

              \item[{}] \textbf{User cancels the request before joining:}
                    \begin{enumerate}[label=\textbf{\arabic*.}]
                        \item Main scenario 1-4
                        \item User selects the option to cancel the request to join.
                        \item System returns to the main interface, and no further action is taken.
                    \end{enumerate}
          \end{itemize}

          \textbf{Success Postcondition:} The user is successfully added to the \hyperref[def:sub_realm]{sub-realm} and can now access it from within the Realm Interface.

    \item \label{uc:20} \textbf{Add a Friend} \\
          \textbf{Actor}: \ref{def:gen_user} \\
          \textbf{Precondition:} User is logged into the app AND the friend has created an account on the app.

          \textbf{Main Success Scenario:}
          \begin{enumerate}[label=\textbf{\arabic*.}]
              \item User navigates to the "Add Friend" option in the app's interface.
              \item System presents the "Add Friend" search interface.
              \item User searches for the friend’s username.
              \item System displays matching search results.
              \item User selects the friend's username from the results.
              \item User sends a friend request.
              \item System sends the friend request and notifies the friend.
              \item System returns the user to the friend management screen.
          \end{enumerate}

          \textbf{Secondary Scenarios:}
          \begin{itemize}
              \item[{}] \textbf{Friend accepts or rejects the request:}
                    \begin{enumerate}[label=\textbf{\arabic*.}]
                        \item Main scenario 6-7
                        \item Friend receives the request and either accepts or rejects it.
                        \item System notifies the user if their request was accepted or rejected.
                    \end{enumerate}

              \item[{}] \textbf{User cancels the friend request before sending:}
                    \begin{enumerate}[label=\textbf{\arabic*.}]
                        \item Main scenario 1-5
                        \item User cancels the request, and no action is taken.
                        \item System returns the user to the previous screen.
                    \end{enumerate}

              \item[{}] \textbf{User cancels the friend request after sending:}
                    \begin{enumerate}[label=\textbf{\arabic*.}]
                        \item Main scenario 6-7
                        \item User cancels the sent request before the friend responds.
                        \item System withdraws the request.
                    \end{enumerate}
          \end{itemize}

          \textbf{Success Postcondition:} The user has successfully added the friend as a connection, and the friend now appears in their friends list.

    \item \label{uc:21} \textbf{Add a Friend to a\ref{def:sub_realm}} \\
          \textbf{Actor}: \ref{def:gen_user} \\
          \textbf{Precondition:} User is logged into the app AND is a member of the \hyperref[def:sub_realm]{sub-realm} AND has the friend already added to their friend list.

          \textbf{Main Success Scenario:}
          \begin{enumerate}[label=\textbf{\arabic*.}]
              \item User navigates to the desired \hyperref[def:sub_realm]{sub-realm} from the Realm Interface.
              \item System presents the \hyperref[def:sub_realm]{sub-realm} view.
              \item User opens the\ref{def:sub_realm} Settings.
              \item System presents the\ref{def:sub_realm} Settings Menu.
              \item User selects the option to “Add a Friend to\ref{def:sub_realm}.”
              \item System displays the user's friend list.
              \item User selects the friend they wish to add to the \hyperref[def:sub_realm]{sub-realm}.
              \item System presents a confirmation prompt.
              \item User confirms adding the friend.
              \item System adds the friend to the \hyperref[def:sub_realm]{sub-realm}.
              \item System returns to the\ref{def:sub_realm} Interface, now displaying the updated member list.
          \end{enumerate}

          \textbf{Secondary Scenarios:}
          \begin{itemize}
              \item[{}] \textbf{User cancels adding the friend:}
                    \begin{enumerate}[label=\textbf{\arabic*.}]
                        \item Main scenario 1-8
                        \item User selects the option to cancel adding the friend.
                        \item System discards the action and returns to the\ref{def:sub_realm} Settings Menu.
                    \end{enumerate}

              \item[{}] \textbf{User selects a friend who is already a \hyperref[def:sub_realm]{sub-realm} member:}
                    \begin{enumerate}[label=\textbf{\arabic*.}]
                        \item Main scenario 1-7
                        \item System notifies the user that the selected friend is already a member of the \hyperref[def:sub_realm]{sub-realm}.
                        \item User can return to the friend list and select another friend.
                    \end{enumerate}

              \item[{}] \textbf{User reselects a friend:}
                    \begin{enumerate}[label=\textbf{\arabic*.}]
                        \item Main scenario 1-7
                        \item User decides to reselect a different friend from the list.
                        \item System returns to the friend selection interface.
                        \item Main scenario resumes from step 7.
                    \end{enumerate}
          \end{itemize}

          \textbf{Success Postcondition:} The user has successfully added the friend to the \hyperref[def:sub_realm]{sub-realm}, and the updated member list is displayed in the\ref{def:sub_realm} Interface.

    \item \label{uc:22} \textbf{Remove a Friend from a\ref{def:sub_realm}} \\
          \textbf{Actor}: \ref{def:gen_user} \\
          \textbf{Precondition:} User is logged into the app AND is a member of the \hyperref[def:sub_realm]{sub-realm} AND has the friend added to the \hyperref[def:sub_realm]{sub-realm}.

          \textbf{Main Success Scenario:}
          \begin{enumerate}[label=\textbf{\arabic*.}]
              \item User navigates to the desired \hyperref[def:sub_realm]{sub-realm} from the Realm Interface.
              \item System presents the \hyperref[def:sub_realm]{sub-realm} view.
              \item User opens the\ref{def:sub_realm} Settings.
              \item System presents the\ref{def:sub_realm} Settings Menu.
              \item User selects the option to “Remove a Friend from\ref{def:sub_realm}.”
              \item System displays the current list of members in the \hyperref[def:sub_realm]{sub-realm}.
              \item User selects the friend they wish to remove.
              \item System presents a confirmation prompt.
              \item User confirms the removal.
              \item System removes the friend from the \hyperref[def:sub_realm]{sub-realm}.
              \item System returns to the\ref{def:sub_realm} Interface, displaying the updated member list without the removed friend.
          \end{enumerate}

          \textbf{Secondary Scenarios:}
          \begin{itemize}
              \item[{}] \textbf{User removes the friend from settings:}
                    \begin{enumerate}[label=\textbf{\arabic*.}]
                        \item User navigates to the settings page in the app.
                        \item System presents the settings display.
                        \item User navigates to the \hyperref[def:sub_realm]{sub-realm} settings.
                        \item System displays a list of \hyperref[def:sub_realm]{sub-realms}.
                        \item User selects the desired \hyperref[def:sub_realm]{sub-realm}.
                        \item Main scenario 4-11
                    \end{enumerate}

              \item[{}] \textbf{User cancels removing the friend:}
                    \begin{enumerate}[label=\textbf{\arabic*.}]
                        \item Main scenario 1-8
                        \item User selects the option to cancel the removal.
                        \item System discards the action and returns to the\ref{def:sub_realm} Settings Menu.
                    \end{enumerate}

              \item[{}] \textbf{User attempts to remove a friend who is not in the \hyperref[def:sub_realm]{sub-realm}:}
                    \begin{enumerate}[label=\textbf{\arabic*.}]
                        \item Main scenario 1-7
                        \item System notifies the user that the selected friend is not a member of the \hyperref[def:sub_realm]{sub-realm}.
                        \item User returns to the member list to reselect another friend.
                    \end{enumerate}

              \item[{}] \textbf{User reselects a friend to remove:}
                    \begin{enumerate}[label=\textbf{\arabic*.}]
                        \item Main scenario 1-7
                        \item User decides to reselect a different friend from the list.
                        \item System returns to the member selection interface.
                        \item Main scenario resumes from step 7.
                    \end{enumerate}
          \end{itemize}

          \textbf{Success Postcondition:} The selected friend is successfully removed from the \hyperref[def:sub_realm]{sub-realm} and no longer has access to the \hyperref[def:sub_realm]{sub-realm} content.

    \item \label{uc:23} \textbf{Create an Account} \\
          \textbf{Actor}: \ref{def:gen_user} \\
          \textbf{Precondition:} User has installed the app AND is on the login screen without an existing account.

          \textbf{Main Success Scenario:}
          \begin{enumerate}[label=\textbf{\arabic*.}]
              \item User selects the "Sign Up" option from the login screen.
              \item System presents the account creation form, asking for details such as email, username, and password.
              \item User fills in the required information.
              \item System performs validation to ensure the information is complete and valid (e.g., unique username, valid email format).
              \item User submits the form.
              \item System creates the account and stores the user’s information securely.
              \item System logs the user into the app and navigates to the home screen.
          \end{enumerate}

          \textbf{Secondary Scenarios:}
          \begin{itemize}
              \item[{}] \textbf{User cancels account creation:}
                    \begin{enumerate}[label=\textbf{\arabic*.}]
                        \item Main scenario 1-4
                        \item User cancels the sign-up process before submitting.
                        \item System discards the entered information and returns the user to the login screen.
                    \end{enumerate}

              \item[{}] \textbf{Account creation fails due to existing email/username:}
                    \begin{enumerate}[label=\textbf{\arabic*.}]
                        \item Main scenario 1-5
                        \item System detects that the email or username is already in use.
                        \item System presents an error message, and the user must provide new information.
                        \item Main scenario resumes from step 2-7.
                    \end{enumerate}
          \end{itemize}

          \textbf{Success Postcondition:} A new account is created, and the user is logged into the app with their new credentials.

    \item \label{uc:24} \textbf{Login to an Account} \\
          \textbf{Actor}: \ref{def:gen_user} \\
          \textbf{Precondition:} User has an existing account on the app.

          \textbf{Main Success Scenario:}
          \begin{enumerate}[label=\textbf{\arabic*.}]
              \item User navigates to the "Login" page.
              \item System presents the login form, asking for email or username, and password.
              \item User enters their email or username, and password.
              \item System performs validation to check if the credentials match an existing account.
              \item User submits the form.
              \item System logs the user into the app and navigates to the home screen.
          \end{enumerate}

          \textbf{Secondary Scenarios:}
          \begin{itemize}
              \item[{}] \textbf{Invalid credentials entered:}
                    \begin{enumerate}[label=\textbf{\arabic*.}]
                        \item Main scenario 1-5
                        \item System detects that the credentials are invalid (e.g., incorrect password).
                        \item System presents an error message, allowing the user to try again.
                    \end{enumerate}

              \item[{}] \textbf{User resets password:}
                    \begin{enumerate}[label=\textbf{\arabic*.}]
                        \item Main scenario 1-4
                        \item User selects the "Forgot Password" option.
                        \item System initiates the password recovery workflow, and the user must follow the steps to reset their password.
                    \end{enumerate}

              \item[{}] \textbf{User cancels login process:}
                    \begin{enumerate}[label=\textbf{\arabic*.}]
                        \item Main scenario 1-4
                        \item User cancels the login process before submitting.
                        \item System returns to the initial screen without logging in.
                    \end{enumerate}
          \end{itemize}

          \textbf{Success Postcondition:} The user is successfully logged into the app and gains access to their account and personalized settings.

    \item \label{uc:25} \textbf{Edit Profile} \\
          \textbf{Actor}: \ref{def:gen_user} \\
          \textbf{Precondition:} User is logged into the app AND has an existing profile.

          \textbf{Main Success Scenario:}
          \begin{enumerate}[label=\textbf{\arabic*.}]
              \item User navigates to the "Settings" page from the app’s main interface.
              \item System presents the Settings Menu with the list of settings categories, including "Profile Settings."
              \item User selects the "Profile Settings" option.
              \item System displays editable fields, such as username, bio, password, and other personal details.
              \item User modifies the desired fields.
              \item User confirms the changes.
              \item System saves the updated profile information.
              \item System returns the user to the Settings page with a confirmation that the profile has been updated.
          \end{enumerate}

          \textbf{Secondary Scenarios:}
          \begin{itemize}
              \item[{}] \textbf{User cancels profile editing:}
                    \begin{enumerate}[label=\textbf{\arabic*.}]
                        \item Main scenario 1-5
                        \item User selects the option to cancel editing before confirming.
                        \item System discards the changes and returns the user to the Settings page without applying any updates.
                    \end{enumerate}

              \item[{}] \textbf{Invalid information provided:}
                    \begin{enumerate}[label=\textbf{\arabic*.}]
                        \item Main scenario 1-6
                        \item System detects that the updated information (e.g., username) is invalid or already taken.
                        \item System presents an error message and prompts the user to enter valid information.
                    \end{enumerate}
          \end{itemize}
          \textbf{Success Postcondition:} The user’s profile is successfully updated and reflects the new information.


    \item \label{uc:26}  Act on reported \ref{def:ar_obj} \\
          \textbf{Actor}: \ref{def:admin} \\
          \textbf{Pre-condition:} Admin is logged into the app \textbf{AND} Admin has navigated to the \textbf{Admin Interface} \\

          \textbf{Main Success Scenario:}
          \begin{enumerate}[label=\textbf{\arabic*.}]
              \item Admin navigates to the “AR Object Reports” panel (\hyperref[ssub:admin_interface]{AI-FR2.1})
              \item System shows a list of app \ref{def:ar_obj}s that have been reported
              \item Admin selects an object
              \item System gives more detailed information of \ref{def:ar_obj} and allows Admin to open an \ref{def:ar_obj} inventory preview (\hyperref[ssub:inventory]{IV-FR6})
              \item System also provides decision options on whether to keep the object or remove it
              \item Admin opens \ref{def:ar_obj} preview
              \item System shows preview of \ref{def:ar_obj}
              \item Admin views \ref{def:ar_obj} preview
              \item Admin closes \ref{def:ar_obj} preview
              \item System once again shows the object detailed information and decision options
              \item Admin reviews the information and decides to keep the \ref{def:ar_obj}
              \item System keeps \ref{def:ar_obj}
              \item System removes user report of \ref{def:ar_obj}
              \item System brings admin back to “AR Object Reports” panel
          \end{enumerate}

          \textbf{Secondary Scenarios:}
          \begin{itemize}
              \item[{\bf 5.1:}] Admin does not open \ref{def:ar_obj} preview
                    \begin{enumerate}[label=\textbf{\arabic*.}]
                        \item Go to step \textbf{11}
                    \end{enumerate}
              \item[{\bf 11.1:}] Admin reviews the information and decides to remove the object
                    \begin{enumerate}[label=\textbf{\arabic*.}]
                        \item System removes object
                        \item Go to step \textbf{13}
                    \end{enumerate}
          \end{itemize}
          \textbf{Success Postcondition:} An \ref{def:ar_obj} report is resolved (\ref{def:ar_obj} is kept or removed)

    \item \label{uc:27}  Act on reported \ref{def:ar_obj} \\
          \textbf{Actor}: \ref{def:admin} \\
          \textbf{Pre-condition:} Admin is logged into the app \textbf{AND} Admin has navigated to the \textbf{Admin Interface} \\

          \textbf{Main Success Scenario:}
          \begin{enumerate}[label=\textbf{\arabic*.}]
              \item Admin navigates to the “Organization User Requests” panel (\hyperref[ssub:admin_interface]{AI-FR2.2})
              \item System shows a list of \ref{def:org_user} requests
              \item Admin selects a request
              \item System gives more details about a request
              \item System also provides decision options on whether to accept or decline a request
              \item Admin reviews information and approves the request
              \item System activates the user as one of the \ref{def:org_user}
              \item System notifies the user that their request has been approved
              \item System removes organization user request
              \item System brings admin back to “Organization User Requests” panel
          \end{enumerate}

          \textbf{Secondary Scenarios:}
          \begin{itemize}
              \item[{\bf 6.1:}] Admin reviews information and denies the request
                    \begin{enumerate}[label=\textbf{\arabic*.}]
                        \item System notifies the user that their request has been denied and gives a reason why
                        \item Go to step \textbf{9}
                    \end{enumerate}
          \end{itemize}
          \textbf{Success Postcondition:} An \ref{def:org_user} account request is resolved (approved or denied).

    \item \label{uc:28} \textbf{Unfriend a Friend} \\
          \textbf{Actor}: \ref{def:gen_user} \\
          \textbf{Precondition:} User is logged into the app AND the friend is currently on the user's friend list.

          \textbf{Main Success Scenario:}
          \begin{enumerate}[label=\textbf{\arabic*.}]
              \item User navigates to friend list from the app’s main interface.
              \item System presents the list of friends.
              \item User selects a friend.
              \item System shows list of actions with the friend.
              \item User selects the "Unfriend" option.
              \item System presents confirmation prompt.
              \item User confirms to unfriend the person.
              \item System removes the friend from the user’s friend list.
              \item System returns the user to the friend list.
          \end{enumerate}

          \textbf{Secondary Scenarios:}
          \begin{itemize}
              \item[{}] \textbf{User cancels unfriending the friend:}
                    \begin{enumerate}[label=\textbf{\arabic*.}]
                        \item Main scenario 1-7
                        \item User selects the option to cancel.
                        \item System discards action and returns to the friend list.
                    \end{enumerate}

              \item[{}] \textbf{User reselects a friend to unfriend:}
                    \begin{enumerate}[label=\textbf{\arabic*.}]
                        \item Main scenario 1-6
                        \item User decides to reselect a different friend from the list.
                        \item System returns to the friend list interface.
                        \item Main scenario resumes from step 3.
                    \end{enumerate}
          \end{itemize}

          \textbf{Success Postcondition:} The selected friend is successfully unfriended and removed from the user's friend list.

\end{enumerate}

\subsection{Quality of Service}

Non-functional requirements pertaining to the quality of the product


\subsubsection{Performance}
\label{sub:performance}

\begin{enumerate}[align=left, label=\textbf{QS-P\arabic*.}]
    \item The map rendering should be completed quickly
    \item The inventory should load quickly when viewing the user’s profile
    \item The real-time render should be delayed from the actual scan by at most ~1 second.
    \item The object generation should not take longer than 30 seconds.
    \item The system must provide fallback modes for rendering AR objects on low-performance devices to ensure accessibility for all users. \\
          {\bf Rationale:} This allows for a wider user base as hardware is less of a restriction. \\
          {\bf Fit Criteria:} \textit{All} tested devices with the required sensors report no performance issues.

\end{enumerate}

\subsubsection{Usability}
\label{sub:usability}

\begin{enumerate}[align=left, label=\textbf{QS-U\arabic*.}]

    \item The system should be localized in most common languages. \\
          {\bf Rationale:} The app should be usable by people in non-english speaking countries. \\
          {\bf Fit Criteria:} The app can be set to display all text in the 5 most spoken languages worldwide. \\
    \item The interfaces and workflows in the app should be intuitive, easy to learn, and satisfying to use. \\
          {\bf Rationale:} The satisfaction the user feels when using the app is a large component of their experience. \\
          {\bf Fit Criteria:} Survey results show most users rate the app as highly intuitive and highly satisfying.
\end{enumerate}

\subsubsection{Security}
\label{sub:security}

\begin{enumerate}[align=left, label=\textbf{QS-SC\arabic*.}]

    \item The system should follow an \hyperref[def:encryption_standard]{encryption standard} for all communications \\
          {\bf Rationale:} This measure eliminates many security vulnerabilities \\
          {\bf Fit Criteria:} It is not possible to extract information from a man in the middle attack
    \item The system should avoid directly or indirectly leaking private user data to other users. \\
          {\bf Rationale:} Giving other users indication of the user's email, name, or location without the user's consent can be harmful to the user \\
          {\bf Fit Criteria:} All interfaces presenting user data including AR object instances present only consentually shared data
\end{enumerate}

\subsubsection{Safety}

\begin{enumerate}[align=left, label=\textbf{QS-SA\arabic*.}]
    \item The system should not be excessively distracting. \\
          {\bf Rationale:} Distracting users from their surroundings may cause lead to collisions or cause users to inadvertently enter \hyperref[def:restricted_area]{restricted areas}.\\
          {\bf Fit Criteria:} Survey results show users rate the app as not distracting.
    \item No system interfaces should present bright flashes or loud noises \\
          {\bf Rationale:} Bright flashes and loud noises can cause harm to users with epilepsy \\
          {\bf Fit Criteria:} No interfaces in the system present bright flashes or loud noises
\end{enumerate}

\subsubsection{Reliability}
\label{sub:reliability}

\begin{enumerate}[align=left, label=\textbf{QS-R\arabic*.}]
    \item The system databases must be resilient to failure/corruption, able to recover, and return to normal operation easily. \\
          {\bf Rationale:} unrecoverable database failure would be fatal to the application. \\
          {\bf Fit Criteria:} fewer than 2\% of users report data loss in the case of database failure and recovery.
\end{enumerate}


\subsubsection{Availability}
\label{sub:availability}

\begin{enumerate}[align=left, label=\textbf{QS-A\arabic*.}]

    \item The server of the system should be highly available. \\
          {\bf Rationale:} Downtime prevents users from using the app when they want to, creating a poor experience. \\
          {\bf Fit Criteria:} System unavailability does not generate complaints for the duration of user testing.
\end{enumerate}


\subsection{Compliance}
\label{sub:compliance}


\begin{enumerate}[align=left, label=\textbf{CO\arabic*.}]
    \item The project shall comply with the \emph{Personal Information and Electronic Documents Act} (PIPEDA).\\
          {\bf Rationale:} The Government of Canada requires all companies to follow certain rules regarding the collection, use, and dissemination of personal user information \cite{PIPEDA}.
    \item The project shall keep records of all in-app purchases and ad revenue for the purposes of yearly tax filing for the period of six years.\\
          {\bf Rationale:} Corporate taxes must be filed every year and both streams of app income will need to be reported. Businesses must keep records going back six years in the event of an audit \cite{6Year}.
    \item The app shall comply with the \emph{Google Play} developer policy.\\
          {\bf Rationale:} All apps published through \emph{Google Play} must first be reviewed by Google for compliance with the developer policy published on their website \cite{GooglePlay}.
    \item The app shall comply with \emph{App Store} review guidelines.\\
          {\bf Rationale:} For an app to be approved for dissemination on the \emph{App Store}, the app must be reviewed and approved by Apple in accordance with the acceptance criteria published on their website \cite{AppStore}.
     \item The system shall comply with the \emph{Web Content Accessibility Guidelines (WCAG) 2.1 Level AA} and the \emph{Accessibility for Ontarians with Disabilities Act (AODA)}. \\
        {\bf Rationale:} Ensuring compliance with these accessibility standards allows users with disabilities to fully engage with the system and meets legal requirements for accessibility in Ontario. \\

\end{enumerate}

\subsubsection{Accessibility Compliance}

\begin{enumerate}[align=left, label=\textbf{QS-AC\arabic*.}]
    \item The system shall comply with established accessibility standards, including \textbf{WCAG 2.1 Level AA} and the \textbf{Accessibility for Ontarians with Disabilities Act (AODA)}. \\
          {\bf Rationale:} Ensuring compliance with accessibility standards allows users with disabilities to fully engage with the system. \\
          {\bf Fit Criteria:} The app shall pass accessibility testing for compliance with WCAG 2.1 Level AA and AODA regulations.

    \item The system shall offer a high-contrast mode and customizable text size for users with visual impairments. \\
          {\bf Rationale:} High contrast and adjustable text improve readability for users with low vision. \\
          {\bf Fit Criteria:} The app shall provide an accessibility setting to adjust contrast and font size.

    \item The system shall avoid flashing visuals and rapid animations that could trigger seizures, as specified in \textbf{QS-SA2}. \\
          {\bf Rationale:} Flashing content can cause seizures in users with photosensitive epilepsy. \\
          {\bf Fit Criteria:} The app shall pass seizure safety tests and restrict flashing effects.

    \item The system shall allow users to modify UI settings, including color themes, font scaling, and motion reduction. \\
          {\bf Rationale:} Users with cognitive disabilities or visual impairments may require a more adaptable interface. \\
          {\bf Fit Criteria:} The app shall provide an accessibility menu with UI customization options.
\end{enumerate}


\subsection{Design and Implementation}

Non-functional requirements pertaining to the design and implementation of the product

\subsubsection{Installation}
\label{ssub:installation}


\begin{enumerate}[align=left, label=\textbf{DI-I\arabic*.}]
    \item The app shall be installable on \emph{Android} and \emph{iOS} devices from their respective app stores.\\
          {\bf Rationale:} Users are accustomed to downloading their apps from their device app store and should not be required to navigate to a 3rd party app store. They also should not have to change OS settings in order to download the app.
    \item The app shall not require any additional installation steps beyond those required from within the target device app store.\\
          {\bf Rationale:} Users may be dissuaded from downloading the app if the installation process is too cumbersome compared to other apps.
\end{enumerate}

\subsubsection{Distribution}
\label{ssub:distribution}


\begin{enumerate}[align=left, label=\textbf{DI-D\arabic*.}]
    \item The app shall be distributed on any mobile devices running iOS 16.0+ or Android 12+\\
          {\bf Rationale:} The app should be available to as many people as possible while at the same time making the development easier by not having to keep old operating system versions supported. Versions should be supported for at least a couple years.
    \item The app shall be available in Canada and the USA.\\
          {\bf Rationale:} Due to legal considerations in different countries, the focus for this app should be the country this project is based out of, Canada, and the USA since they have a larger population with similar laws.
    \item The app shall have a recommended age requirement of 16+.\\
          {\bf Rationale:} Users may be exposed to content not suitable for really young kids so an age requirement should be recommended.
    \item The system shall store all user data within North America.\\
          {\bf Rationale:} Data should be located in a jurisdiction close to home and in a reputable country to reduce privacy concerns of foreign state actors viewing user data.
\end{enumerate}

\subsubsection{Maintainability}
\label{ssub:maintainability}

\begin{enumerate}[align=left, label=\textbf{DI-M\arabic*.}]
    \item All internal APIs of the system must provide useful error messages in the case of system failures. \\
          {\bf Rationale:} Helps with finding and fixing bugs in the system. \\
          {\bf Fit Criteria:} API error messages help identify error source in 90\% of cases.
\end{enumerate}

\subsubsection{Reusability}
\label{ssub:reusability}

\begin{enumerate}[align=left, label=\textbf{DI-R\arabic*.}]
    \item Common elements in the user interfaces of the system will be created as reusable components.  \\
          {\bf Rationale:} Keeps the look of the app consistent across interfaces, and speeds up development. \\
          {\bf Fit Criteria:} All UI elements appearing in more than one place are derived from common components.
\end{enumerate}

\subsubsection{Portability}
\label{ssub:portability}


\begin{enumerate}[align=left, label=\textbf{DI-P\arabic*.}]
    \item The app shall be developed using a cross-platform mobile platform that can build iOS and Android applications. \\
          {\bf Rationale:} To reach as many potential users as possible, the app should be available on the two major mobile operating systems.
    \item The app shall have a common codebase that only differs in configuration files for different target operating systems.\\
          {\bf Rationale:} For ease of development on the different mobile operating systems, there should be no extra consideration for the differences between native app implementations that are not handled by the cross-platform framework.
\end{enumerate}

\subsubsection{Cost}

The Project is constrained to a budget of \$700.

\subsubsection{Deadline}

Refer to Figure \ref{tab:apportioning_table}, the apportioning table, for the planned deadlines for the requirements outlined in this document. Overall project deadlines are outlined in section eight of the \href{https://github.com/russellrd/realm/blob/main/docs/DevelopmentPlan/DevelopmentPlan.pdf}{Development Plan}.

\subsubsection{Proof of Concept}

As part of the proof of concept for this product, the following functional requirements will be implemented:

\begin{itemize}
    \item \hyperref[ssub:realm_interface]{RI-FR1}
    \item \hyperref[ssub:realm_interface]{RI-FR3}
    \item \hyperref[ssub:realm_interface]{RI-FR4}
    \item \hyperref[ssub:object_placement]{OP-FR1}
    \item \hyperref[ssub:object_placement]{OP-FR2}
    \item \hyperref[ssub:obj_scan]{OS-FR1}
    \item \hyperref[ssub:obj_scan]{OS-FR5}
\end{itemize}

\section{Verification}

See the \href{https://github.com/russellrd/realm/blob/main/docs/VnVPlan/VnVPlan.pdf}{Verification and Validation Plan}.

\section{Appendixes}

Supplemental documentation

\subsection{Activity Diagrams}
\label{sub:activity_diagrams}

\begin{center}
    \includegraphics[scale=0.4]{SequenceDiagrams/UC4.png}\\
    \textbf{Figure 2: UC4}
\end{center}

\begin{center}
    \includegraphics[scale=0.4]{SequenceDiagrams/uc7.png}\\
    \textbf{Figure 3: UC7}
\end{center}


\end{document}
