\documentclass[12pt, titlepage]{article}

\usepackage{booktabs}
\usepackage{tabularx}
\usepackage{hyperref}
\hypersetup{
    colorlinks,
    citecolor=black,
    filecolor=black,
    linkcolor=red,
    urlcolor=blue
}
\usepackage[round]{natbib}
\usepackage{geometry}

\usepackage{adjustbox}
\usepackage[dvipsnames]{xcolor}
\usepackage{float}
\usepackage{changepage}
\usepackage{pdflscape}

\input{../Comments}
%% Common Parts

\newcommand{\progname}{Software Engineering} % PUT YOUR PROGRAM NAME HERE
\newcommand{\authname}{Team \#13, ARC
    \\ Ahluwalia, Avanish
    \\ Davidson, Russell
    \\ Malik, Rafey
    \\ Zulfiqar, Abdul} % AUTHOR NAMES                  

\usepackage{hyperref}
    \hypersetup{colorlinks=true, linkcolor=blue, citecolor=blue, filecolor=blue,
                urlcolor=blue, unicode=false}
    \urlstyle{same}
                                


\begin{document}

\title{Verification and Validation Report: \progname} 
\author{\authname}
\date{\today}
	
\maketitle

\pagenumbering{roman}

\section{Revision History}

\begin{tabularx}{\textwidth}{p{3cm}p{2cm}X}
\toprule {\bf Date} & {\bf Version} & {\bf Notes}\\
\midrule
Date 1 & 1.0 & Notes\\
Date 2 & 1.1 & Notes\\
\bottomrule
\end{tabularx}

~\newpage

\section{Symbols, Abbreviations and Acronyms}

\renewcommand{\arraystretch}{1.2}
\begin{tabular}{l l} 
  \toprule		
  \textbf{symbol} & \textbf{description}\\
  \midrule 
  T & Test\\
  \bottomrule
\end{tabular}\\

\wss{symbols, abbreviations or acronyms -- you can reference the SRS tables if needed}

\newpage

\tableofcontents

\listoftables %if appropriate

\listoffigures %if appropriate

\newpage

\pagenumbering{arabic}

This document ...

\section{Functional Requirements Evaluation}

\subsection{Realm Testing}
The following section presents the results of our testing of the realm interface.

\begin{table}[H]
  \caption{\bf Functional Requirements Evaluation Results for the Realm Interface}
  \resizebox{6in}{!}{\begin{tabular}{|l|p{0.15\linewidth}|p{0.3\linewidth}|p{0.3\linewidth}|p{0.3\linewidth}|p{0.1\linewidth}|}
      \hline
      \multicolumn{1}{|l|}{\bfseries Id} &
      \multicolumn{1}{|l|}{\bfseries Control} &
      \multicolumn{1}{l|}{\bfseries Inputs} &
      \multicolumn{1}{l|}{\bfseries Expected Result} &
      \multicolumn{1}{l|}{\bfseries Actual Result} &
      \multicolumn{1}{l|}{\bfseries Result} \\
      \hline
      Test-RI1 & 
      Manual & 
      Tester changes their position and angle in relation to an AR object. &
      The AR object adjusts perspective appropriately, reflecting the new camera position and angle. & 
      Same as expected & 
      \textcolor{Green}{Pass} \\
      \hline
      Test-RI2 & 
      Manual & 
      Tester moves camera over a crowded area where multiple AR objects are present. &
      The interface selectively displays a manageable number of AR objects without overwhelming the user’s view. &
      Same as expected & 
      \textcolor{Green}{Pass} \\
      \hline
      Test-RI3 & 
      Manual & 
      Test AR object instance is placed with a known alignment in the real world, and reference screenshots. &
      Test AR object appears in correct position and orientation as expected, matches stored object instance data. & 
      Same as expected & 
      \textcolor{Green}{Pass} \\
      \hline
      Test-RI6 & 
      Manual & 
      Tester attempts to access the object placement workflow via the provided control. &
      Tester is successfully redirected to the object placement workflow. & 
      Same as expected & 
      \textcolor{Green}{Pass} \\
      \hline
      Test-RI8 & 
      Manual & 
      Tester moves within range of the tour start point. &
      The interface displays a clear indication of the nearby tour and a link to the tour preview. & 
      Same as expected & 
      \textcolor{Green}{Pass} \\
      \hline
      Test-RI9 & 
      Manual & 
      Tester moves closer to a hazard in real space. &
      Interface displays a clear warning when the user approaches the hazard. & 
      Same as expected & 
      \textcolor{Green}{Pass} \\
      \hline
    \end{tabular}}
  \label{table:Realm_Interface_Tests}
\end{table}

\subsection{Object Placement Testing}
The following section presents the results of our object placement testing.

\begin{table}[H]
  \caption{\bf Functional Requirements Evaluation Results for Object Placement Features}
  \resizebox{6in}{!}{\begin{tabular}{|l|p{0.15\linewidth}|p{0.3\linewidth}|p{0.3\linewidth}|p{0.3\linewidth}|p{0.1\linewidth}|}
      \hline
      \multicolumn{1}{|l|}{\bfseries Id} &
      \multicolumn{1}{|l|}{\bfseries Control} &
      \multicolumn{1}{l|}{\bfseries Inputs} &
      \multicolumn{1}{l|}{\bfseries Expected Result} &
      \multicolumn{1}{l|}{\bfseries Actual Result} &
      \multicolumn{1}{l|}{\bfseries Result} \\
      \hline
      Test-OP1 & 
      Manual & 
      Tester selects object from inventory or prompt generation. &
      Interface successfully proceeds to the placement interface with the selected object. & 
      Same as expected & 
      \textcolor{Green}{Pass} \\
      \hline
      Test-OP3 & 
      Manual & 
      Tester rotates, resizes, and translates the object in real space. &
      Object is placed accurately in real space with correct orientation. & 
      Same as expected & 
      \textcolor{Green}{Pass} \\
      \hline
      Test-OP4 & 
      Manual & 
      Tester checks the AR object instance database. &
      Object instance is present with correct details (type, position, orientation). & 
      Same as expected & 
      \textcolor{Green}{Pass} \\
      \hline
      Test-OP5 & 
      Automated and Manual & 
      Tester attempts to place another object in an area with placement limit reached. &
      System prevents additional placements, displaying a warning. & 
      Same as expected & 
      \textcolor{Green}{Pass} \\
      \hline
      Test-OP6 & 
      Automated and Manual & 
      Tester attempts to place another object within a short period after the time-based limit is reached. &
      System restricts further placements, displaying a warning. & 
      Same as expected & 
      \textcolor{Green}{Pass} \\
      \hline
      Test-OP7 & 
      Automated and Manual& 
      Tester places an object, but the initial storage attempt fails. &
      System automatically retries storage until success or retry limit is reached. & 
      Same as expected & 
      \textcolor{Green}{Pass} \\
      \hline
    \end{tabular}}
  \label{table:Object_Placement_Tests}
\end{table}


\subsection{Interactions with User Inventory}
The following section presents the results of our testing of interactions with the user inventory.

\begin{table}[H]
  \caption{\bf Functional Requirements Evaluation Results for Inventory Features}
  \resizebox{6in}{!}{\begin{tabular}{|l|p{0.15\linewidth}|p{0.3\linewidth}|p{0.3\linewidth}|p{0.3\linewidth}|p{0.1\linewidth}|}
      \hline
      \multicolumn{1}{|l|}{\bfseries Id} &
      \multicolumn{1}{|l|}{\bfseries Control} &
      \multicolumn{1}{l|}{\bfseries Inputs} &
      \multicolumn{1}{l|}{\bfseries Expected Result} &
      \multicolumn{1}{l|}{\bfseries Actual Result} &
      \multicolumn{1}{l|}{\bfseries Result} \\
      \hline
      Test-IV1 & 
      Manual & 
      Tester selects an object and chooses the delete option. &
      The selected object is removed from the inventory. & 
      Same as expected & 
      \textcolor{Green}{Pass} \\
      \hline
      Test-IV2 & 
      Manual & 
      Tester adds a new object to the inventory. &
      The new object appears in the inventory. & 
      Same as expected & 
      \textcolor{Green}{Pass} \\
      \hline
      Test-IV3 & 
      Automatic & 
      Tester opens the inventory. &
      Inventory contains the preloaded application-provided objects. & 
      Same as expected & 
      \textcolor{Green}{Pass} \\
      \hline
      Test-IV4 & 
      Automatic & 
      Tester attempts to add an additional object. &
      The object is successfully added, but adding another would be prevented. & 
      Same as expected & 
      \textcolor{Green}{Pass} \\
      \hline
      Test-IV5 & 
      Manual & 
      Tester opens the inventory and inspects object origins. &
      Each personal object is present. & 
      Same as expected & 
      \textcolor{Green}{Pass} \\
      \hline
      Test-IV6 & 
      Automatic & 
      Tester views the total count of objects. &
      The app displays the correct total number of objects. & 
      Same as expected & 
      \textcolor{Green}{Pass} \\
      \hline
      Test-IV7 & 
      Manual & 
      Tester adds both 2D and 3D AR objects to their inventory. &
      Both 2D and 3D objects are correctly stored in inventory. & 
      Same as expected & 
      \textcolor{Green}{Pass} \\
      \hline
      Test-IV9 & 
      Manual & 
      Tester sorts objects by usage or size. &
      Objects are sorted as per user selection. & 
      Same as expected & 
      \textcolor{Green}{Pass} \\
      \hline
      Test-IV10 & 
      Automatic & 
      Tester selects option to view a 3D AR object. &
      3D objects are displayed in a continuous rotating state. & 
      Same as expected & 
      \textcolor{Green}{Pass} \\
      \hline
    \end{tabular}}
  \label{table:Inventory_Tests}
\end{table}

\section{Nonfunctional Requirements Evaluation}

\subsection{Maintainability Testing}
The following section presents the results of our maintainability testing.

\begin{table}[H]
  \caption{\bf Maintainability Testing Evaluation Results}
  \resizebox{6in}{!}{\begin{tabular}{|l|p{0.15\linewidth}|p{0.3\linewidth}|p{0.3\linewidth}|p{0.3\linewidth}|p{0.1\linewidth}|}
      \hline
      \multicolumn{1}{|l|}{\bfseries Id} &
      \multicolumn{1}{|l|}{\bfseries Control} &
      \multicolumn{1}{l|}{\bfseries Inputs} &
      \multicolumn{1}{l|}{\bfseries Expected Result} &
      \multicolumn{1}{l|}{\bfseries Actual Result} &
      \multicolumn{1}{l|}{\bfseries Result} \\
      \hline
      Test-DI-M1 & 
      Manual and Automated & 
      Simulate common errors like database connection failure, invalid input data, service timeout in internal APIs. &
      Error messages clearly indicate the source and nature of the error (90\% of the cases). & 
      Same as expected & 
      \textcolor{Green}{Pass} \\
      \hline
  \end{tabular}}
  \label{table:Maintainability_Tests}
\end{table}

\subsection{Compliance Testing}
The following section presents the results of our compliance testing.

\begin{table}[H]
  \caption{\bf Compliance Testing Evaluation Results}
  \resizebox{6in}{!}{\begin{tabular}{|l|p{0.15\linewidth}|p{0.3\linewidth}|p{0.3\linewidth}|p{0.3\linewidth}|p{0.1\linewidth}|}
      \hline
      \multicolumn{1}{|l|}{\bfseries Id} &
      \multicolumn{1}{|l|}{\bfseries Control} &
      \multicolumn{1}{l|}{\bfseries Inputs} &
      \multicolumn{1}{l|}{\bfseries Expected Result} &
      \multicolumn{1}{l|}{\bfseries Actual Result} &
      \multicolumn{1}{l|}{\bfseries Result} \\
      \hline
      Test-CO1 & 
      Manual & 
      App is checked against the Personal Information and Electronic Documents Act (PIPEDA). &
      The app complies with all sections of PIPEDA. & 
      Same as expected & 
      \textcolor{Green}{Pass} \\
      \hline
      Test-CO2 & 
      Manual & 
      The app's revenue records are checked for purchases and ad-revenue spanning at least 6 years. &
      The records go back at least 6 years. & 
      N/A & 
       \\
      \hline
      Test-CO3 & 
      Manual & 
      App is checked against the Google Play Developer Policy. &
      The app complies with all sections of the Google Play Developer Policy. & 
      Same as expected & 
      \textcolor{Green}{Pass} \\
      \hline
      Test-CO4 & 
      Manual & 
      App is checked against the App Store Review Guidelines. &
      The app complies with all sections of the App Store Review Guidelines. & 
      Same as expected & 
      \textcolor{Green}{Pass} \\
      \hline
  \end{tabular}}
  \label{table:Compliance_Tests}
\end{table}

\subsection{Reusability Testing}
The following section presents the results of our reusability testing.

\begin{table}[H]
  \caption{\bf Reusability Testing Evaluation Results}
  \resizebox{6in}{!}{\begin{tabular}{|l|p{0.15\linewidth}|p{0.3\linewidth}|p{0.3\linewidth}|p{0.3\linewidth}|p{0.1\linewidth}|}
      \hline
      \multicolumn{1}{|l|}{\bfseries Id} &
      \multicolumn{1}{|l|}{\bfseries Control} &
      \multicolumn{1}{l|}{\bfseries Inputs} &
      \multicolumn{1}{l|}{\bfseries Expected Result} &
      \multicolumn{1}{l|}{\bfseries Actual Result} &
      \multicolumn{1}{l|}{\bfseries Result} \\
      \hline
      Test-DI-R1 & 
      Static & 
      All code is sent to a static analyzer that detects code duplication. &
      The analysis shows metrics related to code sections with a high amount of duplication, suggesting areas for refactoring. & 
      Some duplicate code was found. Refactoring to fix this issue. & 
      \textcolor{Red}{Fail} \\
      \hline
  \end{tabular}}
  \label{table:Reusability_Tests}
\end{table}
	
\section{Comparison to Existing Implementation}	

This section will not be appropriate for every project.

\section{Unit Testing}

\subsection{Access Hardware Testing}

\begin{table}[H]
  \caption{\bf Access Hardware Module Unit Test Results}
  \resizebox{6in}{!}{\begin{tabular}{|l|p{0.15\linewidth}|p{0.3\linewidth}|p{0.3\linewidth}|p{0.3\linewidth}|p{0.1\linewidth}|}
      \hline
      \multicolumn{1}{|l|}{\bfseries Id} &
      \multicolumn{1}{|l|}{\bfseries Type} &
      \multicolumn{1}{l|}{\bfseries Inputs} &
      \multicolumn{1}{l|}{\bfseries Expected Result} &
      \multicolumn{1}{l|}{\bfseries Actual Result} &
      \multicolumn{1}{l|}{\bfseries Result} \\
      \hline
      Test-AHM1 & 
      Automated & 
      Known simulator height value compared with simulator height. & 
      The known height value should match the simulator's height. & 
      Same as expected & 
      \textcolor{Green}{Pass} \\
      \hline
      Test-AHM2 & 
      Automated & 
      Known simulator width value compared with simulator width. & 
      The known width value should match the simulator's width. & 
      Same as expected & 
      \textcolor{Green}{Pass} \\
      \hline
  \end{tabular}}
  \label{table:Access_Hardware_Module_Unit_Tests}
\end{table}

\subsection{Inventory Module Testing}

\begin{table}[H]
  \caption{\bf Inventory Module Unit Test Results}
  \resizebox{6in}{!}{\begin{tabular}{|l|p{0.15\linewidth}|p{0.3\linewidth}|p{0.3\linewidth}|p{0.3\linewidth}|p{0.1\linewidth}|}
      \hline
      \multicolumn{1}{|l|}{\bfseries Id} &
      \multicolumn{1}{|l|}{\bfseries Type} &
      \multicolumn{1}{l|}{\bfseries Inputs} &
      \multicolumn{1}{l|}{\bfseries Expected Result} &
      \multicolumn{1}{l|}{\bfseries Actual Result} &
      \multicolumn{1}{l|}{\bfseries Result} \\
      \hline
      Test-IM1 & 
      Automated & 
      Ensure the object count is less than or equal to the maximum allowed. & 
      The object count should be less than or equal to the maximum object count (MAX\_OBJ\_COUNT). & TOTAL\_OBJ\_COUNT 
      Same as expected & 
      \textcolor{Green}{Pass} \\
      \hline
      Test-IM2 & 
      Automated & 
      Add an object to the inventory. & 
      The TOTAL\_OBJ\_COUNT should increase by one, and the object should be added to the objects list. & 
      Same as expected & 
      \textcolor{Green}{Pass} \\
      \hline
      Test-IM3 & 
      Automated & 
      Retrieve an object from the inventory. & 
      The object should be returned with its properties intact. & 
      Same as expected & 
      \textcolor{Green}{Pass} \\
      \hline
      Test-IM4 & 
      Automated & 
      Delete an object from the inventory. & 
      The TOTAL\_OBJ\_COUNT should decrease by one, and the object should be removed from the objects list. & 
      Same as expected & 
      \textcolor{Green}{Pass} \\
      \hline
      Test-IM5 & 
      Automated & 
      Retrieve the list of all objects in the inventory. & 
      The list should contain exactly the number of objects corresponding to the TOTAL\_OBJ\_COUNT. & 
      Same as expected & 
      \textcolor{Green}{Pass} \\
      \hline
  \end{tabular}}
  \label{table:Inventory_Module_Unit_Tests}
\end{table}

\subsection{Object Placement Testing}
N/A

\subsection{Restricted Area Detect Testing}

\begin{table}[H]
  \caption{\bf Restricted Area Detect Module Unit Test Results}
  \resizebox{6in}{!}{\begin{tabular}{|l|p{0.15\linewidth}|p{0.3\linewidth}|p{0.3\linewidth}|p{0.3\linewidth}|p{0.1\linewidth}|}
      \hline
      \multicolumn{1}{|l|}{\bfseries Id} &
      \multicolumn{1}{|l|}{\bfseries Type} &
      \multicolumn{1}{l|}{\bfseries Inputs} &
      \multicolumn{1}{l|}{\bfseries Expected Result} &
      \multicolumn{1}{l|}{\bfseries Actual Result} &
      \multicolumn{1}{l|}{\bfseries Result} \\
      \hline
      Test-RADM1 & 
      Automated & 
      GPS coordinates of a known restricted area. & 
      The module should detect that the area is restricted. & 
      Same as expected & 
      \textcolor{Green}{Pass} \\
      \hline
      Test-RADM2 & 
      Automated & 
      GPS coordinates of a known unrestricted area. & 
      The module should detect that the area is unrestricted. & 
      Same as expected & 
      \textcolor{Green}{Pass} \\
      \hline
  \end{tabular}}
  \label{table:Restricted_Area_Detect_Module_Unit_Tests}
\end{table}

\subsection{Weather Hazard Detect Testing}

\begin{table}[H]
  \caption{\bf Weather Hazard Detect Module Unit Test Results}
  \resizebox{6in}{!}{\begin{tabular}{|l|p{0.15\linewidth}|p{0.3\linewidth}|p{0.3\linewidth}|p{0.3\linewidth}|p{0.1\linewidth}|}
      \hline
      \multicolumn{1}{|l|}{\bfseries Id} &
      \multicolumn{1}{|l|}{\bfseries Type} &
      \multicolumn{1}{l|}{\bfseries Inputs} &
      \multicolumn{1}{l|}{\bfseries Expected Result} &
      \multicolumn{1}{l|}{\bfseries Actual Result} &
      \multicolumn{1}{l|}{\bfseries Result} \\
      \hline
      Test-WHDM1 & 
      Automated & 
      Make an external call to the weather API for the Toronto area. & 
      The weather data returned by the external API request should match the data returned by the module. & 
      Same as expected & 
      \textcolor{Green}{Pass} \\
      \hline
  \end{tabular}}
  \label{table:Weather_Hazard_Detect_Module_Unit_Tests}
\end{table}


\section{Changes Due to Testing}

\wss{This section should highlight how feedback from the users and from 
the supervisor (when one exists) shaped the final product.  In particular 
the feedback from the Rev 0 demo to the supervisor (or to potential users) 
should be highlighted.}

The following Tests for Functional Requirements (3.1 of VnVPlan) subsections where removed due to focus the project on tours instead of social media as advised in the Rev 0 demo:
\begin{itemize}
    \item 3.1.1
    \item 3.1.2
        \begin{itemize}
            \item Test-RI4
            \item Test-RI5
            \item Test-RI7
            \item Test-RI10
        \end{itemize}
    \item 3.1.3
        \begin{itemize}
            \item Test-OP2
        \end{itemize}
    \item 3.1.7
    \item 3.1.8
    \item 3.1.9
    \item 3.1.11
    \item 3.1.12
        \begin{itemize}
            \item Test-IV8
        \end{itemize}
\end{itemize}

\section{Automated Testing}
		
\section{Trace to Requirements}
		
\section{Trace to Modules}		

\section{Code Coverage Metrics}

\bibliographystyle{plainnat}
\bibliography{../../refs/References}

\newpage{}
\section*{Appendix --- Reflection}

The information in this section will be used to evaluate the team members on the
graduate attribute of Reflection.

\input{../Reflection.tex}

\begin{enumerate}
  \item What went well while writing this deliverable? 
  \item What pain points did you experience during this deliverable, and how
    did you resolve them?
  \item Which parts of this document stemmed from speaking to your client(s) or
  a proxy (e.g. your peers)? Which ones were not, and why?
  \item In what ways was the Verification and Validation (VnV) Plan different
  from the activities that were actually conducted for VnV?  If there were
  differences, what changes required the modification in the plan?  Why did
  these changes occur?  Would you be able to anticipate these changes in future
  projects?  If there weren't any differences, how was your team able to clearly
  predict a feasible amount of effort and the right tasks needed to build the
  evidence that demonstrates the required quality?  (It is expected that most
  teams will have had to deviate from their original VnV Plan.)
\end{enumerate}

\end{document}