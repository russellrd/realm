\documentclass{article}

\usepackage{float}
\restylefloat{table}

\usepackage{booktabs}

\title{Team Contributions: POC\\\progname}

\author{\authname}

\date{}

\input{../Comments}
%% Common Parts

\newcommand{\progname}{Software Engineering} % PUT YOUR PROGRAM NAME HERE
\newcommand{\authname}{Team \#13, ARC
    \\ Ahluwalia, Avanish
    \\ Davidson, Russell
    \\ Malik, Rafey
    \\ Zulfiqar, Abdul} % AUTHOR NAMES                  

\usepackage{hyperref}
    \hypersetup{colorlinks=true, linkcolor=blue, citecolor=blue, filecolor=blue,
                urlcolor=blue, unicode=false}
    \urlstyle{same}
                                


\begin{document}

\maketitle

This document summarizes the contributions of each team member up to the POC
Demo.  The time period of interest is the time between the beginning of the term
and the POC demo.

\section{Demo Plans}

We will be demonstrating two main features in our AR app: object scanning to obtain a 3D object which can be placed, and object placement on a surface.

\section{Team Meeting Attendance}

\begin{table}[H]
\centering
\begin{tabular}{ll}
\toprule
\textbf{Student} & \textbf{Meetings}\\
\midrule
Total & 20\\
Avanish Ahluwalia & 20\\
Russell Davidson & 20\\
Rafey Malik & 20\\
Abdul Zulfiqar & 20\\
\bottomrule
\end{tabular}
\end{table}

\section{Supervisor/Stakeholder Meeting Attendance}

\begin{table}[H]
\centering
\begin{tabular}{ll}
\toprule
\textbf{Student} & \textbf{Meetings}\\
\midrule
Total & 0\\
Avanish Ahluwalia & 0\\
Russell Davidson & 0\\
Rafey Malik & 0\\
Abdul Zulfiqar & 0\\
\bottomrule
\end{tabular}
\end{table}

Our team does not have any supervisor or stakeholder.

\section{Lecture Attendance}

\begin{table}[H]
\centering
\begin{tabular}{ll}
\toprule
\textbf{Student} & \textbf{Lectures}\\
\midrule
Total & 5\\
Avanish Ahluwalia & 5\\
Russell Davidson & 5\\
Rafey Malik & 3\\
Abdul Zulfiqar & 4\\
\bottomrule
\end{tabular}
\end{table}

\section{TA Document Discussion Attendance}

\begin{table}[H]
\centering
\begin{tabular}{ll}
\toprule
\textbf{Student} & \textbf{Lectures}\\
\midrule
Total & 4\\
Avanish Ahluwalia & 3\\
Russell Davidson & 4\\
Rafey Malik & 3\\
Abdul Zulfiqar & 4\\
\bottomrule
\end{tabular}
\end{table}

For one of the meetings, we had initially decided not to come, but Abdul and Russell were on campus, so they just decided to go anyways. Rafey and Avanish supplied questions to be asked during the meeting and were briefed afterwards.

\section{Commits}

\begin{table}[H]
\centering
\begin{tabular}{lll}
\toprule
\textbf{Student} & \textbf{Commits} & \textbf{Percent}\\
\midrule
Total & Num & 100\% \\
Avanish Ahluwalia & Num & 8.93\% \\
Russell Davidson & Num & 28.57\% \\
Rafey Malik & Num & 33.93\% \\
Abdul Zulfiqar & Num & 28.57\% \\
\bottomrule
\end{tabular}
\end{table}

Some team members chose to do smaller commits, which resulting in more commits being counted for them, whilst others did one large commit in one PR, resulting in fewer overall commits. This does not necessitate that an individual did less work than others, just that they had larger and fewer commits.

\section{Issue Tracker}

\begin{table}[H]
\centering
\begin{tabular}{lll}
\toprule
\textbf{Student} & \textbf{Authored (O+C)} & \textbf{Assigned (C only)}\\
\midrule
Avanish Ahluwalia & 8 & 0\\
Russell Davidson & 10 & 0\\
Rafey Malik & 6 & 0\\
Abdul Zulfiqar & 10 & 0\\
\bottomrule
\end{tabular}
\end{table}

We did not assign issues formally on github. Perhaps we will begin doing so on github for counting.

\section{CICD}

We will use CICD to automatically test and deploy our code every time we make updates, ensuring everything works smoothly without breaking. This will help to catch bugs early and see how changes perform in a staging environment before going live. CICD will streamline our workflow, allowing us to focus more on coding and teamwork while preparing us with tools used in real software projects.

\end{document}