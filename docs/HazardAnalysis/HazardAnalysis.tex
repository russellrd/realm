\documentclass{article}

\usepackage{booktabs}
\usepackage{tabularx}
\usepackage{hyperref}
\usepackage{makecell}   % For multi-line cells
\usepackage{array}     % for custom column formatting
\usepackage{boldline}  % for thicker lines
\usepackage{graphicx}
\usepackage{geometry}
\usepackage{pdflscape}
\usepackage{enumitem}  % For custom numbering of lists
\geometry{a4paper, margin=1in}
\renewcommand{\arraystretch}{1.3} % Adjust row height for better fit

\hypersetup{
    colorlinks=true,       % false: boxed links; true: colored links
    linkcolor=red,          % color of internal links (change box color with linkbordercolor)
    citecolor=green,        % color of links to bibliography
    filecolor=magenta,      % color of file links
    urlcolor=cyan           % color of external links
}

\title{Hazard Analysis\\\progname}

\author{\authname}

\date{}

\input{../Comments}
%% Common Parts

\newcommand{\progname}{Software Engineering} % PUT YOUR PROGRAM NAME HERE
\newcommand{\authname}{Team \#13, ARC
    \\ Ahluwalia, Avanish
    \\ Davidson, Russell
    \\ Malik, Rafey
    \\ Zulfiqar, Abdul} % AUTHOR NAMES                  

\usepackage{hyperref}
    \hypersetup{colorlinks=true, linkcolor=blue, citecolor=blue, filecolor=blue,
                urlcolor=blue, unicode=false}
    \urlstyle{same}
                                


\begin{document}

\maketitle
\thispagestyle{empty}

~\newpage

\pagenumbering{roman}

\begin{table}[hp]
    \caption{Revision History} \label{rev_history_table}
    \begin{tabularx}{\textwidth}{p{3cm}p{2cm}p{3cm}X}
        \toprule {\textbf{Date}} & {\textbf{Version}} & {\textbf{Author(s)}} & {\textbf{Notes}} \\
        \midrule
        2024-10-18               & 1.0         & All      & Initial Hazard Analysis      \\
        \bottomrule
    \end{tabularx}
\end{table}

~\newpage

\tableofcontents

~\newpage

\pagenumbering{arabic}

\wss{You are free to modify this template.}

\section{Introduction}

\wss{You can include your definition of what a hazard is here.}

\section{Scope and Purpose of Hazard Analysis}

\wss{You should say what \textbf{loss} could be incurred because of the
hazards.}

\section{System Boundaries and Components}

\wss{Dividing the system into components will help you brainstorm the hazards.
You shouldn't do a full design of the components, just get a feel for the major
ones.  For projects that involve hardware, the components will typically include
each individual piece of hardware.  If your software will have a database, or an
important library, these are also potential components.}

\section{Critical Assumptions}

\wss{These assumptions that are made about the software or system.  You should
minimize the number of assumptions that remove potential hazards.  For instance,
you could assume a part will never fail, but it is generally better to include
this potential failure mode.}

\begin{itemize}
    \item The software system is only used in the intended software environments (unmodified iOS and Android Versions 16.0+ and 12.0+ respectively as per distribution requirement DI-D1)
    \item The software system is only used on devices that meet the minimum hardware requirements (GPS, camera, and all required sensors present)
\end{itemize}

\section{Failure Mode and Effect Analysis}

\wss{Include your FMEA table here. This is the most important part of this document.}
\wss{The safety requirements in the table do not have to have the prefix SR.
The most important thing is to show traceability to your SRS. You might trace to
requirements you have already written, or you might need to add new
requirements.}
\wss{If no safety requirement can be devised, other mitigation strategies can be
entered in the table, including strategies involving providing additional
documentation, and/or test cases.}

\subsection{Hazards Out of Scope}

Hazards resulting from the failure of the following components will not be considered in this analysis as the software system cannot mitigate hazards in these external systems 
\begin{itemize}
    \item User device hardware
    \item Back-end server hardware
\end{itemize}


\subsection{Failure Mode and Effects Analysis Table}

\newgeometry{left=0.2in,right=0.2in,top=0.2in,bottom=0.2in}
\begin{landscape}
\begin{table}[hp]
    \caption{FMEA Table} \label{FMEA}
    \centering
    \begin{footnotesize}
    \begin{tabular}{|p{1in}|p{1in}|p{1.5in}|p{1.5in}|p{1.5in}|p{2in}|p{0.4in}|p{0.4in}|}
        \hline
        \multicolumn{1}{|c|}{\textbf{Design Function}} & \multicolumn{1}{c|}{\textbf{Failure Modes}} & \multicolumn{1}{c|}{\textbf{Effects of Failure}} & \multicolumn{1}{c|}{\textbf{Causes of Failure}} & \multicolumn{1}{|c|}{\textbf{Detection}} & \multicolumn{1}{c|}{\textbf{Recommended Action}} & \multicolumn{1}{c|}{\textbf{Req}} & \multicolumn{1}{c|}{\textbf{Ref.}} \\
        \hline
        Object Placement & System fails to store AR object instance in database & User has to redo the object placement workflow, wasting their time & Database failure, Back-end overwhelmed with traffic & Provide useful error messages from back-end to app client & Implement automatic retry mechanism for AR object instance storage in the case of storage failure & RR-1, RR-2 & H1-1\\
        \hline
        Object Instance Storage & Database becomes corrupted & Users lose access to their (and other's) AR object instances & Faulty storage devices on server, Bugs in database management software & Automated periodic database testing & Implement a mechanism to restore the database from a backup if necessary, based on automated database testing & RR-3, RR-4 & H2-1 \\
        \hline
    \end{tabular}
    \end{footnotesize}
\end{table}
\end{landscape}
\restoregeometry

\section{Safety and Security Requirements}

\wss{Newly discovered requirements.  These should also be added to the SRS.  (A rationale design process how and why to fake it.)}


\subsection{Robustness Requirements}

\begin{enumerate}[label=\textbf{RR-\arabic*},ref=SR-\arabic*]
    \item \label{RR-1} The system must have an automated mechanism to retry the upload and storage of object instances when an initial attempt fails
    \item \label{RR-2} All internal APIs of the system must provide useful error messages in the case of system failures
    \item \label{RR-3} The system must automatically back up databases daily
    \item \label{RR-4} The system must have a mechanism to restore a database from a backup in the case of unrecoverable failure / corruption
\end{enumerate}


\section{Roadmap}

\wss{Which safety requirements will be implemented as part of the capstone timeline?
Which requirements will be implemented in the future?}

\newpage{}

\section*{Appendix --- Reflection}

\wss{Not required for CAS 741}

\input{../Reflection.tex}

\begin{enumerate}
    \item What went well while writing this deliverable? 
    \item What pain points did you experience during this deliverable, and how
    did you resolve them?
    \item Which of your listed risks had your team thought of before this
    deliverable, and which did you think of while doing this deliverable? For
    the latter ones (ones you thought of while doing the Hazard Analysis), how
    did they come about?
    \item Other than the risk of physical harm (some projects may not have any
    appreciable risks of this form), list at least 2 other types of risk in
    software products. Why are they important to consider?
\end{enumerate}

\end{document}