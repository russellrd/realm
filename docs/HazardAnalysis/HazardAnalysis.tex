\documentclass{article}

\usepackage{booktabs}
\usepackage{tabularx}
\usepackage{hyperref}
\usepackage{makecell}   % For multi-line cells
\usepackage{array}     % for custom column formatting
\usepackage{boldline}  % for thicker lines
\usepackage{graphicx}
\usepackage{geometry}
\usepackage{pdflscape}
\usepackage{enumitem}  % For custom numbering of lists
\geometry{a4paper, margin=1in}
\renewcommand{\arraystretch}{1.3} % Adjust row height for better fit

\hypersetup{
    colorlinks=true,       % false: boxed links; true: colored links
    linkcolor=red,          % color of internal links (change box color with linkbordercolor)
    citecolor=green,        % color of links to bibliography
    filecolor=magenta,      % color of file links
    urlcolor=cyan           % color of external links
}

\title{Hazard Analysis\\\progname}

\author{\authname}

\date{}

%% Comments

\usepackage{color}

\newif\ifcomments\commentstrue %displays comments
%\newif\ifcomments\commentsfalse %so that comments do not display

\ifcomments
\newcommand{\authornote}[3]{\textcolor{#1}{[#3 ---#2]}}
\newcommand{\todo}[1]{\textcolor{red}{[TODO: #1]}}
\else
\newcommand{\authornote}[3]{}
\newcommand{\todo}[1]{}
\fi

\newcommand{\wss}[1]{\authornote{blue}{SS}{#1}} 
\newcommand{\plt}[1]{\authornote{magenta}{TPLT}{#1}} %For explanation of the template
\newcommand{\an}[1]{\authornote{cyan}{Author}{#1}}

%% Common Parts

\newcommand{\progname}{Software Engineering} % PUT YOUR PROGRAM NAME HERE
\newcommand{\authname}{Team \#13, ARC
    \\ Avanish, Ahluwalia
    \\ Russell, Davidson
    \\ Rafey, Malik
    \\ Abdul, Zulfiqar} % AUTHOR NAMES                  

\usepackage{hyperref}
    \hypersetup{colorlinks=true, linkcolor=blue, citecolor=blue, filecolor=blue,
                urlcolor=blue, unicode=false}
    \urlstyle{same}
                                


\begin{document}

\maketitle
\thispagestyle{empty}

~\newpage

\pagenumbering{roman}

\begin{table}[hp]
    \caption{Revision History} \label{rev_history_table}
    \begin{tabularx}{\textwidth}{p{3cm}p{2cm}p{3cm}X}
        \toprule {\textbf{Date}} & {\textbf{Version}} & {\textbf{Author(s)}} & {\textbf{Notes}} \\
        \midrule
        2024-10-18               & 1.0         & All      & Initial Hazard Analysis      \\
        \bottomrule
    \end{tabularx}
\end{table}

~\newpage

\tableofcontents

~\newpage

\pagenumbering{arabic}

\section{Introduction}

Hazard Analysis is a key step in the engineering process, which is used to identify potential risks and dangers in a system or process. It helps us to ensure the safety and risk management of a system. By systematically analyzing potential risks of the system, we can work to mitigate these potential harms and any consequences that may arise. This document is a key part of the overall safety of the Realm app. It aims to help our stakeholders understand the possible risks of the app and all precautions we have in place to prevent such risks.

\section{Scope and Purpose of Hazard Analysis}

The scope of this Hazard Analysis covers the identification, evaluation, and mitigation of hazards as it relates to the entire development process of the Realm project. Hazards which are considered in this document include features within the app, and external hazards through the environment.\\

Certain losses that could be incurred because of hazards are loss of privacy, including unauthorized tracking of users location and unauthorized sharing of personal data, such as email or password. Another loss is health risks from bright flashes within the app, which can trigger seizures in users that may be suffering from photosensitive epilepsy. Furthermore, human injury may occur from accidents because of users being distracted from AR content.

\section{System Boundaries and Components}

\wss{Dividing the system into components will help you brainstorm the hazards.
You shouldn't do a full design of the components, just get a feel for the major
ones.  For projects that involve hardware, the components will typically include
each individual piece of hardware.  If your software will have a database, or an
important library, these are also potential components.}

\section{Critical Assumptions}

\wss{These assumptions that are made about the software or system.  You should
minimize the number of assumptions that remove potential hazards.  For instance,
you could assume a part will never fail, but it is generally better to include
this potential failure mode.}

\begin{itemize}
    \item The software system is only used in the intended software environments (unmodified iOS and Android Versions 16.0+ and 12.0+ respectively as per distribution requirement DI-D1)
    \item The software system is only used on devices that meet the minimum hardware requirements (GPS, camera, and all required sensors present)
\end{itemize}

\section{Failure Mode and Effect Analysis}

\wss{Include your FMEA table here. This is the most important part of this document.}
\wss{The safety requirements in the table do not have to have the prefix SR.
The most important thing is to show traceability to your SRS. You might trace to
requirements you have already written, or you might need to add new
requirements.}
\wss{If no safety requirement can be devised, other mitigation strategies can be
entered in the table, including strategies involving providing additional
documentation, and/or test cases.}

\subsection{Hazards Out of Scope}

Hazards resulting from the failure of the following components will not be considered in this analysis as the software system cannot mitigate hazards in these external systems 
\begin{itemize}
    \item User device hardware
    \item Back-end server hardware
\end{itemize}


\subsection{Failure Mode and Effects Analysis Table}

\newgeometry{left=0.2in,right=0.2in,top=0.2in,bottom=0.2in}
\begin{landscape}
\begin{table}[hp]
    \caption{FMEA Table} \label{FMEA}
    \centering
    \begin{footnotesize}
    \begin{tabular}{|p{1in}|p{1in}|p{1.5in}|p{1.5in}|p{1.5in}|p{2in}|p{0.4in}|p{0.4in}|}
        \hline
        \multicolumn{1}{|c|}{\textbf{Design Function}} & \multicolumn{1}{c|}{\textbf{Failure Modes}} & \multicolumn{1}{c|}{\textbf{Effects of Failure}} & \multicolumn{1}{c|}{\textbf{Causes of Failure}} & \multicolumn{1}{|c|}{\textbf{Detection}} & \multicolumn{1}{c|}{\textbf{Recommended Action}} & \multicolumn{1}{c|}{\textbf{Req}} & \multicolumn{1}{c|}{\textbf{Ref.}} \\
        \hline
        Object Placement & System fails to store AR object instance in database & User has to redo the object placement workflow, wasting their time & Database failure, Back-end overwhelmed with traffic & Provide useful error messages from back-end to app client & Implement automatic retry mechanism for AR object instance storage in the case of storage failure & RR-1, RR-2 & H1-1\\
        \hline
        Object Instance Storage & Database becomes corrupted & Users lose access to their (and other's) AR object instances & Faulty storage devices on server, Bugs in database management software & Automated periodic database testing & Implement a mechanism to restore the database from a backup if necessary, based on automated database testing & RR-3, RR-4 & H2-1 \\
        \hline
        Privacy and Data Protection & User data is exposed to unauthorized users & Loss of user trust, potential legal implications, data breaches & Weak encryption, improper access control policies and other security vulnerabilities & Regular security audits, reports of unauthorized access & Implement strong encryption protocols, two-factor authentication, and regular security updates & SR-5, SR-6 & H3-1 \\
        \hline
        AR Object Rendering &  AR objects fail to render or display incorrectly in the user’s environment & Users are unable to see placed objects or experience visual glitches & Device camera issues, insufficient processing power, software bugs, network issues & User-reported issues, monitoring rendering logs & Optimize rendering algorithms for performance; implement fallback modes for low-performance devices & SR-7 & H4-1 \\
        \hline
    \end{tabular}
    \end{footnotesize}
\end{table}
\end{landscape}
\restoregeometry

\section{Safety and Security Requirements}

\wss{Newly discovered requirements.  These should also be added to the SRS.  (A rationale design process how and why to fake it.)}

\begin{enumerate}[label=\textbf{SR-\arabic*},ref=SR-\arabic*]
    \item \label{SR-5} The system shall implement encryption protocols to protect user data when storing.
    \item \label{SR-6} The system shall provide a multi-factor authentication option for user accounts to enhance security.
    \item \label{SR-7} The system shall provide fallback modes for rendering AR objects on low-performance devices to ensure accessibility for all users.
    \item \label{SR-8} The system shall allow users to customize rendering settings (e.g., brightness, effects) to minimize discomfort or health risks associated with viewing objects.
    \item \label{SR-9} The system shall display indicators to alert users when rendering might cause discomfort (e.g., rapid movement or flashing effects).

\end{enumerate}

\subsection{Robustness Requirements}

\begin{enumerate}[label=\textbf{RR-\arabic*},ref=SR-\arabic*]
    \item \label{RR-1} The system must have an automated mechanism to retry the upload and storage of object instances when an initial attempt fails
    \item \label{RR-2} All internal APIs of the system must provide useful error messages in the case of system failures
    \item \label{RR-3} The system must automatically back up databases daily
    \item \label{RR-4} The system must have a mechanism to restore a database from a backup in the case of unrecoverable failure / corruption
\end{enumerate}

\section{Roadmap}

\wss{Which safety requirements will be implemented as part of the capstone timeline?
Which requirements will be implemented in the future?}

\newpage{}

\section*{Appendix --- Reflection}

\wss{Not required for CAS 741}

The purpose of reflection questions is to give you a chance to assess your own
learning and that of your group as a whole, and to find ways to improve in the
future. Reflection is an important part of the learning process.  Reflection is
also an essential component of a successful software development process.  

Reflections are most interesting and useful when they're honest, even if the
stories they tell are imperfect. You will be marked based on your depth of
thought and analysis, and not based on the content of the reflections
themselves. Thus, for full marks we encourage you to answer openly and honestly
and to avoid simply writing ``what you think the evaluator wants to hear.''

Please answer the following questions.  Some questions can be answered on the
team level, but where appropriate, each team member should write their own
response:


\begin{enumerate}
    \item What went well while writing this deliverable? \\ \\
    \hspace*{-0.97cm}\textbf{Ans.} Answer 1 \\
    
    \item What pain points did you experience during this deliverable, and how did you resolve them? \\ \\
    \hspace*{-0.97cm}\textbf{Ans.} Answer 2 \\
    
    \item Which of your listed risks had your team thought of before this
    deliverable, and which did you think of while doing this deliverable? For
    the latter ones (ones you thought of while doing the Hazard Analysis), how did they come about? \\ \\
    \hspace*{-0.97cm}\textbf{Ans.} Answer 3 \\
    
    \item Other than the risk of physical harm (some projects may not have any appreciable risks of this form), list at least 2 other types of risk in software products. Why are they important to consider? \\ \\
    \hspace*{-0.97cm}\textbf{Ans.} Answer 4 \\
    
\end{enumerate}

\end{document}