\documentclass[12pt, titlepage]{article}

\usepackage{booktabs}
\usepackage{tabularx}
\usepackage{hyperref}
\hypersetup{
    colorlinks,
    citecolor=blue,
    filecolor=black,
    linkcolor=red,
    urlcolor=blue
}
\usepackage[round]{natbib}

\input{../Comments}
%% Common Parts

\newcommand{\progname}{Software Engineering} % PUT YOUR PROGRAM NAME HERE
\newcommand{\authname}{Team \#13, ARC
    \\ Ahluwalia, Avanish
    \\ Davidson, Russell
    \\ Malik, Rafey
    \\ Zulfiqar, Abdul} % AUTHOR NAMES                  

\usepackage{hyperref}
    \hypersetup{colorlinks=true, linkcolor=blue, citecolor=blue, filecolor=blue,
                urlcolor=blue, unicode=false}
    \urlstyle{same}
                                


\begin{document}

\title{System Verification and Validation Plan for \progname{}} 
\author{\authname}
\date{\today}
	
\maketitle

\pagenumbering{roman}

\section*{Revision History}

\begin{tabularx}{\textwidth}{p{3cm}p{2cm}X}
\toprule {\bf Date} & {\bf Version} & {\bf Notes}\\
\midrule
Date 1 & 1.0 & Notes\\
Date 2 & 1.1 & Notes\\
\bottomrule
\end{tabularx}

~\\
\wss{The intention of the VnV plan is to increase confidence in the software.
However, this does not mean listing every verification and validation technique
that has ever been devised.  The VnV plan should also be a \textbf{feasible}
plan. Execution of the plan should be possible with the time and team available.
If the full plan cannot be completed during the time available, it can either be
modified to ``fake it'', or a better solution is to add a section describing
what work has been completed and what work is still planned for the future.}

\wss{The VnV plan is typically started after the requirements stage, but before
the design stage.  This means that the sections related to unit testing cannot
initially be completed.  The sections will be filled in after the design stage
is complete.  the final version of the VnV plan should have all sections filled
in.}

\newpage

\tableofcontents

\listoftables
\wss{Remove this section if it isn't needed}

\listoffigures
\wss{Remove this section if it isn't needed}

\newpage

\section{Symbols, Abbreviations, and Acronyms}

\renewcommand{\arraystretch}{1.2}
\begin{tabular}{l l} 
  \toprule		
  \textbf{symbol} & \textbf{description}\\
  \midrule 
  T & Test\\
  \bottomrule
\end{tabular}\\

\wss{symbols, abbreviations, or acronyms --- you can simply reference the SRS
  \citep{SRS} tables, if appropriate}

\wss{Remove this section if it isn't needed}

\newpage

\pagenumbering{arabic}

The purpose of this Verification and Validation (VnV) document is to outline the testing process we’ll use to make sure our project meets all its requirements and functions correctly. Verification checks that we’re building the product according to our design, while validation ensures the product meets the users’ needs and performs as expected. This document will describe both functional and non-functional tests, which help confirm that the project is reliable, safe, and easy to use. By following this VnV plan, we can identify and fix any issues early, ensuring the final product is high-quality and ready for users.

\section{General Information}

\subsection{Summary}

\wss{Say what software is being tested.  Give its name and a brief overview of
  its general functions.}

\subsection{Objectives}

\wss{State what is intended to be accomplished.  The objective will be around
  the qualities that are most important for your project.  You might have
  something like: ``build confidence in the software correctness,''
  ``demonstrate adequate usability.'' etc.  You won't list all of the qualities,
  just those that are most important.}

\wss{You should also list the objectives that are out of scope.  You don't have 
the resources to do everything, so what will you be leaving out.  For instance, 
if you are not going to verify the quality of usability, state this.  It is also 
worthwhile to justify why the objectives are left out.}

\wss{The objectives are important because they highlight that you are aware of 
limitations in your resources for verification and validation.  You can't do everything, 
so what are you going to prioritize?  As an example, if your system depends on an 
external library, you can explicitly state that you will assume that external library 
has already been verified by its implementation team.}

\subsection{Challenge Level and Extras}

The challenge level for the Realm project is \textbf{general}. The extras that are going to be tackled in this project are \textbf{user documentation} and \textbf{usability testing}.

\subsection{Relevant Documentation}

\wss{Reference relevant documentation.  This will definitely include your SRS
  and your other project documents (design documents, like MG, MIS, etc).  You
  can include these even before they are written, since by the time the project
  is done, they will be written.  You can create BibTeX entries for your
  documents and within those entries include a hyperlink to the documents.}

\citet{SRS}

\wss{Don't just list the other documents.  You should explain why they are relevant and 
how they relate to your VnV efforts.}

\section{Plan}

This section outlines the structured plan for verifying and validating our project at each stage of development. It first lists the roles of the team members involved in verification, the strategies for ensuring the accuracy of our requirements, design, and implementation, as well as the tools we'll use for automated testing. Each part is designed to ensure our project meets quality standards and fulfills its intended purpose.

\subsection{Verification and Validation Team}

\wss{Your teammates.  Maybe your supervisor.
  You should do more than list names.  You should say what each person's role is
  for the project's verification.  A table is a good way to summarize this information.}

\subsection{SRS Verification Plan}

\wss{List any approaches you intend to use for SRS verification.  This may
  include ad hoc feedback from reviewers, like your classmates (like your
  primary reviewer), or you may plan for something more rigorous/systematic.}

\wss{If you have a supervisor for the project, you shouldn't just say they will
read over the SRS.  You should explain your structured approach to the review.
Will you have a meeting?  What will you present?  What questions will you ask?
Will you give them instructions for a task-based inspection?  Will you use your
issue tracker?}

\wss{Maybe create an SRS checklist?}

\subsection{Design Verification Plan}

\wss{Plans for design verification}

\wss{The review will include reviews by your classmates}

\wss{Create a checklists?}

\begin{itemize}
    \item Internal Team Review
    \item Peer Review
    \item TA Review
    \item Requirements Coverage Check
\end{itemize}

\subsection{Verification and Validation Plan Verification Plan}

\wss{The verification and validation plan is an artifact that should also be
verified.  Techniques for this include review and mutation testing.}

\wss{The review will include reviews by your classmates}

\wss{Create a checklists?}



\subsection{Implementation Verification Plan}

\wss{You should at least point to the tests listed in this document and the unit
  testing plan.}

\wss{In this section you would also give any details of any plans for static
  verification of the implementation.  Potential techniques include code
  walkthroughs, code inspection, static analyzers, etc.}

\wss{The final class presentation in CAS 741 could be used as a code
walkthrough.  There is also a possibility of using the final presentation (in
CAS741) for a partial usability survey.}

\subsection{Automated Testing and Verification Tools}

Refer to \href{https://github.com/russellrd/realm/blob/main/docs/DevelopmentPlan/DevelopmentPlan.pdf}{section 10 of the development plan} for the list of tools. We will use the NUnit-based Unity Testing framework for unit testing and code coverage metrics. Continuous testing will be done by GitHub actions using Unity Builder and Unity Test Runner from the GameCI open source project.

\subsection{Software Validation Plan}

\wss{If there is any external data that can be used for validation, you should
  point to it here.  If there are no plans for validation, you should state that
  here.}

\wss{You might want to use review sessions with the stakeholder to check that
the requirements document captures the right requirements.  Maybe task based
inspection?}

\wss{For those capstone teams with an external supervisor, the Rev 0 demo should 
be used as an opportunity to validate the requirements.  You should plan on 
demonstrating your project to your supervisor shortly after the scheduled Rev 0 demo.  
The feedback from your supervisor will be very useful for improving your project.}

\wss{For teams without an external supervisor, user testing can serve the same purpose 
as a Rev 0 demo for the supervisor.}

\wss{This section might reference back to the SRS verification section.}

\section{System Tests}

\wss{There should be text between all headings, even if it is just a roadmap of
the contents of the subsections.}

\subsection{Tests for Functional Requirements}

\wss{Subsets of the tests may be in related, so this section is divided into
  different areas.  If there are no identifiable subsets for the tests, this
  level of document structure can be removed.}

\wss{Include a blurb here to explain why the subsections below
  cover the requirements.  References to the SRS would be good here.}

\subsubsection{Area of Testing1}

\wss{It would be nice to have a blurb here to explain why the subsections below
  cover the requirements.  References to the SRS would be good here.  If a section
  covers tests for input constraints, you should reference the data constraints
  table in the SRS.}
		
\paragraph{Title for Test}

\begin{enumerate}

\item{test-id1\\}

Control: Manual versus Automatic
					
Initial State: 
					
Input: 
					
Output: \wss{The expected result for the given inputs.  Output is not how you
are going to return the results of the test.  The output is the expected
result.}

Test Case Derivation: \wss{Justify the expected value given in the Output field}
					
How test will be performed: 
					
\item{test-id2\\}

Control: Manual versus Automatic
					
Initial State: 
					
Input: 
					
Output: \wss{The expected result for the given inputs}

Test Case Derivation: \wss{Justify the expected value given in the Output field}

How test will be performed: 

\end{enumerate}

\subsubsection{Area of Testing2}

...

\subsubsection{Profile Testing}
\label{sec:profile_screen_testing}

The Profile Screen Testing focuses on verifying user interactions related to profile management, password changes, and viewing of profile data. These tests ensure that users can efficiently manage their profile settings and view relevant information, which is critical for maintaining user engagement and security.

\begin{enumerate}
    \item \textbf{Name:} {Validate User Authentication} \label{itm:Test-PS1}\\
    \textbf{Test ID:} Test-PS1\\
    \textbf{Control:} Automated \\
    \textbf{Initial State:} App launched, login screen displayed. \\
    \textbf{Input:} User enters valid credentials. \\
    \textbf{Output:} The expected result is that the user successfully logs in and is redirected to their profile page. \\
    \textbf{Test Case Derivation:} This test is to ensure the system authenticates users with valid credentials.\\
    \textbf{How test will be performed:} Through an automated script that inputs valid user credentials and checks if redirection to the profile page is successful.

    \item \textbf{Name:} {Password Change Functionality} \label{itm:Test-PS2}\\
    \textbf{Test ID:} Test-PS2\\
    \textbf{Control:} Manual \\
    \textbf{Initial State:} User logged in, on profile settings page. \\
    \textbf{Input:} User inputs new password and confirms. \\
    \textbf{Output:} The expected result is that the system updates the password and provides a confirmation message. \\
    \textbf{Test Case Derivation:} To verify that the system allows users to change their password securely. \\
    \textbf{How test will be performed:} Tester manually changes the password and checks for confirmation of the change.

    \item \textbf{Name:} {View Profile Information} \label{itm:Test-PS3}\\
    \textbf{Test ID:} Test-PS3\\
    \textbf{Control:} Automated \\
    \textbf{Initial State:} User logged in, on profile page. \\
    \textbf{Input:} None. \\
    \textbf{Output:} The expected result is that profile information (username, password, profile picture, status) is displayed correctly. \\
    \textbf{Test Case Derivation:} Confirm that all user profile information is retrievable and displayed correctly. \\
    \textbf{How test will be performed:} Automated test that logs in as a user and verifies that all profile information is displayed as expected.

    \item \textbf{Name:} {Access Help Page} \label{itm:Test-PS4}\\
    \textbf{Test ID:} Test-PS4\\
    \textbf{Control:} Manual \\
    \textbf{Initial State:} User logged in, on profile page. \\
    \textbf{Input:} User navigates to help page. \\
    \textbf{Output:} The expected result is that a help page with FAQs and additional help information is displayed. \\
    \textbf{Test Case Derivation:} Ensure the help page is accessible and provides useful information. \\
    \textbf{How test will be performed:} Tester navigates to the help page and verifies the presence and accuracy of the information.
\end{enumerate}

\subsubsection{Sub-Realm Testing}
\label{sec:sub-realms_testing}

The Sub-Realms Testing focuses on verifying user interactions related to creating, managing, and interacting within sub-realms. These tests ensure that users can efficiently create sub-realms, manage sub-realm members, and interact with sub-realm-specific content, which is essential for fostering collaboration and community engagement within the app.

\begin{enumerate}

    \item \textbf{Name:} {Create New Sub-Realm} \label{itm:Test-G1}\\
    \textbf{Test ID:} Test-G1\\
    \textbf{Control:} Manual \\
    \textbf{Initial State:} User logged in, on sub-realms page. \\
    \textbf{Input:} User provides a sub-realm name, description, and invites members. \\
    \textbf{Output:} The expected result is that a new sub-realm is created with the specified details, and invited members are notified. \\
    \textbf{Test Case Derivation:} To verify that users can create new sub-realms with relevant details and invite members. \\
    \textbf{How test will be performed:} Tester manually creates a sub-realm and verifies that the sub-realm is created with the correct details and members are notified.

    \item \textbf{Name:} {Add or Remove Sub-Realm Members} \label{itm:Test-G2}\\
    \textbf{Test ID:} Test-G2\\
    \textbf{Control:} Manual \\
    \textbf{Initial State:} User logged in, viewing a sub-realm they manage. \\
    \textbf{Input:} User adds or removes specific members from the sub-realm. \\
    \textbf{Output:} The expected result is that sub-realm membership updates accordingly, reflecting added or removed members. \\
    \textbf{Test Case Derivation:} Ensure the sub-realm manager can manage sub-realm membership. \\
    \textbf{How test will be performed:} Tester adds and removes members from the sub-realm and verifies that the changes are reflected.

    \item \textbf{Name:} {Edit Sub-Realm Settings} \label{itm:Test-G3}\\
    \textbf{Test ID:} Test-G3\\
    \textbf{Control:} Manual \\
    \textbf{Initial State:} User logged in, viewing a sub-realm they manage. \\
    \textbf{Input:} User changes sub-realm name or description. \\
    \textbf{Output:} The expected result is that sub-realm settings update with the new name or description. \\
    \textbf{Test Case Derivation:} Verify that sub-realm settings can be modified by the sub-realm manager. \\
    \textbf{How test will be performed:} Tester edits the sub-realm settings and checks that the changes are applied.

    \item \textbf{Name:} {Delete Sub-Realm} \label{itm:Test-G4}\\
    \textbf{Test ID:} Test-G4\\
    \textbf{Control:} Manual \\
    \textbf{Initial State:} User logged in, viewing a sub-realm they manage. \\
    \textbf{Input:} User selects the option to delete the sub-realm. \\
    \textbf{Output:} The expected result is that the sub-realm and all associated data are removed from the system. \\
    \textbf{Test Case Derivation:} Ensure that sub-realm managers can delete sub-realms and remove associated data. \\
    \textbf{How test will be performed:} Tester deletes a sub-realm and verifies that it no longer exists in the system.

    \item \textbf{Name:} {Interact with Sub-Realm AR Content} \label{itm:Test-G5}\\
    \textbf{Test ID:} Test-G5\\
    \textbf{Control:} Manual \\
    \textbf{Initial State:} User logged in, viewing a sub-realm. \\
    \textbf{Input:} User interacts with sub-realm-specific AR content. \\
    \textbf{Output:} The expected result is that sub-realm-specific AR content responds to user interactions. \\
    \textbf{Test Case Derivation:} Verify that users can interact with shared AR content within the sub-realm. \\
    \textbf{How test will be performed:} Tester interacts with sub-realm-specific AR content and observes responses.

\end{enumerate}

\subsubsection{Friends Testing}
\label{sec:friends_screen_testing}

The Friends Screen Testing focuses on verifying user interactions related to managing friends, including sending requests, accepting or rejecting requests, and viewing or removing friends. These tests ensure that users can effectively manage their friend connections within the app, which is important for social interaction and network building.

\begin{enumerate}

    \item \textbf{Name:} {Send Friend Request} \label{itm:Test-FS1}\\
    \textbf{Test ID:} Test-FS1\\
    \textbf{Control:} Manual \\
    \textbf{Initial State:} User logged in, viewing friends screen. \\
    \textbf{Input:} User sends a friend request to another user. \\
    \textbf{Output:} The expected result is that the friend request is sent, and the recipient receives a notification. \\
    \textbf{Test Case Derivation:} To ensure users can initiate friend requests. \\
    \textbf{How test will be performed:} Tester sends a friend request and verifies that it reaches the recipient.

    \item \textbf{Name:} {Accept or Reject Friend Request} \label{itm:Test-FS2}\\
    \textbf{Test ID:} Test-FS2\\
    \textbf{Control:} Manual \\
    \textbf{Initial State:} User logged in, viewing pending friend requests. \\
    \textbf{Input:} User accepts or rejects a friend request. \\
    \textbf{Output:} The expected result is that the friend list updates with the accepted friend, or the request is removed if rejected. \\
    \textbf{Test Case Derivation:} Confirm that users can manage incoming friend requests. \\
    \textbf{How test will be performed:} Tester accepts and rejects friend requests, verifying updates to the friend list.

    \item \textbf{Name:} {View Friend List} \label{itm:Test-FS3}\\
    \textbf{Test ID:} Test-FS3\\
    \textbf{Control:} Manual \\
    \textbf{Initial State:} User logged in, viewing friends screen. \\
    \textbf{Input:} None \\
    \textbf{Output:} The expected result is that a list of current friends is displayed. \\
    \textbf{Test Case Derivation:} Ensure users can view their list of friends. \\
    \textbf{How test will be performed:} Tester views the friends screen to check for an accurate friend list.

    \item \textbf{Name:} {Remove Friend} \label{itm:Test-FS4}\\
    \textbf{Test ID:} Test-FS4\\
    \textbf{Control:} Manual \\
    \textbf{Initial State:} User logged in, viewing friends screen. \\
    \textbf{Input:} User selects a friend to remove. \\
    \textbf{Output:} The expected result is that the friend is removed from the user’s friend list. \\
    \textbf{Test Case Derivation:} To verify users can remove friends from their friend list. \\
    \textbf{How test will be performed:} Tester removes a friend and verifies the friend list updates.

\end{enumerate}

\subsubsection{Settings Testing}
\label{sec:settings_testing}

The Settings Testing focuses on verifying user interactions related to modifying various settings, including accessibility, display, privacy, profile, and sub-realm settings. These tests ensure that users can customize their app experience according to their preferences and privacy requirements, enhancing usability and personalization.

\begin{enumerate}

    \item \textbf{Name:} {Modify Accessibility Settings} \label{itm:Test-S1}\\
    \textbf{Test ID:} Test-S1\\
    \textbf{Control:} Manual \\
    \textbf{Initial State:} User logged in, viewing settings page. \\
    \textbf{Input:} User adjusts text size, enables/disables viewing of object names, or changes language. \\
    \textbf{Output:} The expected result is that accessibility settings apply as configured by the user. \\
    \textbf{Test Case Derivation:} Confirm that accessibility settings are configurable. \\
    \textbf{How test will be performed:} Tester changes each accessibility setting and verifies the changes apply.

    \item \textbf{Name:} {Adjust Display Settings} \label{itm:Test-S2}\\
    \textbf{Test ID:} Test-S2\\
    \textbf{Control:} Manual \\
    \textbf{Initial State:} User logged in, viewing settings page. \\
    \textbf{Input:} User changes display settings such as light/dark mode or AR object visibility. \\
    \textbf{Output:} The expected result is that display settings reflect user preferences. \\
    \textbf{Test Case Derivation:} To ensure display settings are customizable by the user. \\
    \textbf{How test will be performed:} Tester modifies display settings and observes changes.

    \item \textbf{Name:} {Manage Privacy Settings} \label{itm:Test-S3}\\
    \textbf{Test ID:} Test-S3\\
    \textbf{Control:} Manual \\
    \textbf{Initial State:} User logged in, viewing settings page. \\
    \textbf{Input:} User modifies privacy settings to control visibility of profile, friends list, and AR interactions. \\
    \textbf{Output:} The expected result is that privacy settings update based on user preferences. \\
    \textbf{Test Case Derivation:} Verify users can control privacy settings for their profiles and interactions. \\
    \textbf{How test will be performed:} Tester changes privacy settings and checks for corresponding updates.

    \item \textbf{Name:} {Update Profile Settings} \label{itm:Test-S4}\\
    \textbf{Test ID:} Test-S4\\
    \textbf{Control:} Manual \\
    \textbf{Initial State:} User logged in, viewing settings page. \\
    \textbf{Input:} User changes username, password, profile picture, or status. \\
    \textbf{Output:} The expected result is that profile settings are updated and saved. \\
    \textbf{Test Case Derivation:} Ensure users can update personal profile information. \\
    \textbf{How test will be performed:} Tester updates profile settings and verifies changes.

    \item \textbf{Name:} {Access Sub-Realm Settings} \label{itm:Test-S5}\\
    \textbf{Test ID:} Test-S5\\
    \textbf{Control:} Manual \\
    \textbf{Initial State:} User logged in, viewing settings page. \\
    \textbf{Input:} User navigates to sub-realm settings to modify sub-realm options. \\
    \textbf{Output:} The expected result is that sub-realm settings are accessible and configurable. \\
    \textbf{Test Case Derivation:} To confirm users can manage settings for sub-realms they belong to. \\
    \textbf{How test will be performed:} Tester accesses and modifies sub-realm settings, verifying updates.

\end{enumerate}

\subsection{Tests for Nonfunctional Requirements}

\wss{The nonfunctional requirements for accuracy will likely just reference the
  appropriate functional tests from above.  The test cases should mention
  reporting the relative error for these tests.  Not all projects will
  necessarily have nonfunctional requirements related to accuracy.}

\wss{For some nonfunctional tests, you won't be setting a target threshold for
passing the test, but rather describing the experiment you will do to measure
the quality for different inputs.  For instance, you could measure speed versus
the problem size.  The output of the test isn't pass/fail, but rather a summary
table or graph.}

\wss{Tests related to usability could include conducting a usability test and
  survey.  The survey will be in the Appendix.}

\wss{Static tests, review, inspections, and walkthroughs, will not follow the
format for the tests given below.}

\wss{If you introduce static tests in your plan, you need to provide details.
How will they be done?  In cases like code (or document) walkthroughs, who will
be involved? Be specific.}

\subsubsection{Area of Testing1}
		
\paragraph{Title for Test}

\begin{enumerate}

\item{test-id1\\}

Type: Functional, Dynamic, Manual, Static etc.
					
Initial State: 
					
Input/Condition: 
					
Output/Result: 
					
How test will be performed: 
					
\item{test-id2\\}

Type: Functional, Dynamic, Manual, Static etc.
					
Initial State: 
					
Input: 
					
Output: 
					
How test will be performed: 

\end{enumerate}

\subsubsection{Area of Testing2}

...

\subsubsection{Portability Testing}

\begin{enumerate}

    \item \textbf{Name:} Validate Cross-Platform Compatibility \label{itm:Test-PT1} \\
    \textbf{Test ID:} Test-PT1 \\
    \textbf{Type:} Non-Functional, Manual \\
    \textbf{Initial State:} Application is built for both iOS and Android platforms. \\
    \textbf{Input/Condition:} Run the app on iOS and Android devices. \\
    \textbf{Output/Result:} The app is functional and displays correctly on both platforms. \\
    \textbf{How test will be performed:} Tester will install the app on both an iOS and an Android device, verifying consistent functionality and UI.

    \item \textbf{Name:} Common Codebase Validation \label{itm:Test-PT2} \\
    \textbf{Test ID:} Test-PT2 \\
    \textbf{Type:} Non-Functional, Code Review \\
    \textbf{Initial State:} The app’s codebase is ready for review. \\
    \textbf{Input/Condition:} Inspect the codebase to ensure shared files are correctly configured with minimal platform-specific files. \\
    \textbf{Output/Result:} Codebase only differs in configuration files for platform-specific settings. \\
    \textbf{How test will be performed:} Developer will conduct a code walkthrough, focusing on configuration files to confirm minimal platform-specific variations.

    \item \textbf{Name:} Build Verification on iOS and Android \label{itm:Test-PT3} \\
    \textbf{Test ID:} Test-PT3 \\
    \textbf{Type:} Non-Functional, Automated \\
    \textbf{Initial State:} The cross-platform codebase is ready for automated build testing. \\
    \textbf{Input/Condition:} Initiate automated builds for both iOS and Android. \\
    \textbf{Output/Result:} Both builds succeed without errors. \\
    \textbf{How test will be performed:} An automated CI/CD pipeline will attempt to build the app for both platforms, confirming compatibility.

\end{enumerate}


\subsubsection{Safety Testing}

\begin{enumerate}

    \item \textbf{Name:} Distraction to Surroundings Assessment \label{itm:Test-SA1} \\
    \textbf{Test ID:} Test-SA1 \\
    \textbf{Type:} Non-Functional, Survey-Based \\
    \textbf{Initial State:} App is functional and ready for user testing. \\
    \textbf{Input/Condition:} Conduct a user survey after users engage with the app in a controlled environment. \\
    \textbf{Output/Result:} Survey results show that users do not find the app dangerously distracting them from their surroundings while using it. \\
    \textbf{How test will be performed:} A group of users will be observed using the app, followed by a survey asking them to rate their distraction levels from surrounding objects. Results will be analyzed to confirm minimal distraction.

    \item \textbf{Name:} No Bright Flashes or Loud Noises \label{itm:Test-SA2} \\
    \textbf{Test ID:} Test-SA2 \\
    \textbf{Type:} Non-Functional, Manual Inspection \\
    \textbf{Initial State:} The app is fully developed with all interfaces available for review. \\
    \textbf{Input/Condition:} Navigate through all screens and interactions within the app. \\
    \textbf{Output/Result:} No bright flashes or loud noises are present in any of the app interfaces. \\
    \textbf{How test will be performed:} Tester will manually explore the app, paying special attention to visual and audio elements, ensuring that no features could trigger discomfort or seizures in sensitive users.
\end{enumerate}

\subsubsection{Installation Testing}

\begin{enumerate}
    \item \textbf{Name:} Verify App Store Availability \label{itm:Test-I1} \\
    \textbf{Test ID:} Test-I1 \\
    \textbf{Type:} Non-Functional, Manual \\
    \textbf{Initial State:} App has been submitted and approved on both iOS and Android app stores. \\
    \textbf{Input/Condition:} Search for the app on the Apple App Store and Google Play Store. \\
    \textbf{Output/Result:} The app is available for download on both app stores. \\
    \textbf{How test will be performed:} Tester will verify the presence of the app by searching for it on the respective app stores and confirming it is listed and downloadable.

    \item \textbf{Name:} Simple Installation Process \label{itm:Test-I2} \\
    \textbf{Test ID:} Test-I2 \\
    \textbf{Type:} Non-Functional, Manual \\
    \textbf{Initial State:} App is available on both app stores. \\
    \textbf{Input/Condition:} Attempt to install the app on a device from both the Apple App Store and Google Play Store. \\
    \textbf{Output/Result:} The app installs directly without any additional steps or configurations. \\
    \textbf{How test will be performed:} Tester will initiate the installation from each app store, ensuring the app installs seamlessly without requiring extra configurations or settings adjustments.
\end{enumerate}


\subsection{Traceability Between Test Cases and Requirements}

\wss{Provide a table that shows which test cases are supporting which
  requirements.}

\begin{table}[h!]
    \centering
    \begin{tabular}{|l|l|l|}
        \hline
        \textbf{Test-ID} & \textbf{Test Name} & \textbf{Requirements} \\
        \hline
        \hyperref[itm:Test-PS1]{Test-PS1} & Validate User Authentication & PS-FR1 \\
        \hline
        \hyperref[itm:Test-PS2]{Test-PS2} & Password Change Functionality & PS-FR3 \\
        \hline
        \hyperref[itm:Test-PS3]{Test-PS3} & View Profile Information & PS-FR4 \\
        \hline
        \hyperref[itm:Test-PS4]{Test-PS4} & Access Help Page & PS-FR6 \\
        \hline
        \hyperref[itm:Test-G1]{Test-G1} & Create New Group & G-FR1 \\
        \hline
        \hyperref[itm:Test-G2]{Test-G2} & Add or Remove Group Members & G-FR2 \\
        \hline
        \hyperref[itm:Test-G3]{Test-G3} & Edit Group Settings & G-FR3 \\
        \hline
        \hyperref[itm:Test-G4]{Test-G4} & Delete Group & G-FR4 \\
        \hline
        \hyperref[itm:Test-G5]{Test-G5} & Interact with Group AR Content & G-FR5 \\
        \hline
        \hyperref[itm:Test-FS1]{Test-FS1} & Send Friend Request & FS-FR1 \\
        \hline
        \hyperref[itm:Test-FS2]{Test-FS2} & Accept or Reject Friend Request & FS-FR2 \\
        \hline
        \hyperref[itm:Test-FS3]{Test-FS3} & View Friend List & FS-FR3 \\
        \hline
        \hyperref[itm:Test-FS4]{Test-FS4} & Remove Friend & FS-FR4 \\
        \hline
        \hyperref[itm:Test-S1]{Test-S1} & Modify Accessibility Settings & S-FR1 \\
        \hline
        \hyperref[itm:Test-S2]{Test-S2} & Adjust Display Settings & S-FR2 \\
        \hline
        \hyperref[itm:Test-S3]{Test-S3} & Manage Privacy Settings & S-FR3 \\
        \hline
        \hyperref[itm:Test-S4]{Test-S4} & Update Profile Settings & S-FR4 \\
        \hline
        \hyperref[itm:Test-S5]{Test-S5} & Access Group Settings & S-FR5 \\
        \hline
        \hyperref[itm:Test-PT1]{Test-PT1} & Validate Cross-Platform Compatibility & DI-P1 \\
        \hline
        \hyperref[itm:Test-PT2]{Test-PT2} & Common Codebase Validation & DI-P2 \\
        \hline
        \hyperref[itm:Test-PT3]{Test-PT3} & Build Verification on iOS and Android & DI-P1\\
        \hline
        \hyperref[itm:Test-SA1]{Test-SA1} & Distraction Level Assessment & QS-SA1 \\
        \hline
        \hyperref[itm:Test-SA2]{Test-SA2} & No Bright Flashes or Loud Noises & QS-SA2 \\
        \hline
        \hyperref[itm:Test-I1]{Test-I1} & Verify App Store Availability & DI-I1 \\
        \hline
        \hyperref[itm:Test-I2]{Test-I2} & Simple Installation Process & DI-I2 \\
        \hline
    \end{tabular}
    \caption{Mapping of Tests to Requirements}
    \label{tab:test_requirements}
\end{table}


\pagebreak

\section{Unit Test Description}

\wss{This section should not be filled in until after the MIS (detailed design
  document) has been completed.}

\wss{Reference your MIS (detailed design document) and explain your overall
philosophy for test case selection.}  

\wss{To save space and time, it may be an option to provide less detail in this section.  
For the unit tests you can potentially layout your testing strategy here.  That is, you 
can explain how tests will be selected for each module.  For instance, your test building 
approach could be test cases for each access program, including one test for normal behaviour 
and as many tests as needed for edge cases.  Rather than create the details of the input 
and output here, you could point to the unit testing code.  For this to work, you code 
needs to be well-documented, with meaningful names for all of the tests.}

\subsection{Unit Testing Scope}

\wss{What modules are outside of the scope.  If there are modules that are
  developed by someone else, then you would say here if you aren't planning on
  verifying them.  There may also be modules that are part of your software, but
  have a lower priority for verification than others.  If this is the case,
  explain your rationale for the ranking of module importance.}

\subsection{Tests for Functional Requirements}

\wss{Most of the verification will be through automated unit testing.  If
  appropriate specific modules can be verified by a non-testing based
  technique.  That can also be documented in this section.}

\subsubsection{Module 1}

\wss{Include a blurb here to explain why the subsections below cover the module.
  References to the MIS would be good.  You will want tests from a black box
  perspective and from a white box perspective.  Explain to the reader how the
  tests were selected.}

\begin{enumerate}

\item{test-id1\\}

Type: \wss{Functional, Dynamic, Manual, Automatic, Static etc. Most will
  be automatic}
					
Initial State: 
					
Input: 
					
Output: \wss{The expected result for the given inputs}

Test Case Derivation: \wss{Justify the expected value given in the Output field}

How test will be performed: 
					
\item{test-id2\\}

Type: \wss{Functional, Dynamic, Manual, Automatic, Static etc. Most will
  be automatic}
					
Initial State: 
					
Input: 
					
Output: \wss{The expected result for the given inputs}

Test Case Derivation: \wss{Justify the expected value given in the Output field}

How test will be performed: 

\item{...\\}
    
\end{enumerate}

\subsubsection{Module 2}

...

\subsection{Tests for Nonfunctional Requirements}

\wss{If there is a module that needs to be independently assessed for
  performance, those test cases can go here.  In some projects, planning for
  nonfunctional tests of units will not be that relevant.}

\wss{These tests may involve collecting performance data from previously
  mentioned functional tests.}

\subsubsection{Module ?}
		
\begin{enumerate}

\item{test-id1\\}

Type: \wss{Functional, Dynamic, Manual, Automatic, Static etc. Most will
  be automatic}
					
Initial State: 
					
Input/Condition: 
					
Output/Result: 
					
How test will be performed: 
					
\item{test-id2\\}

Type: Functional, Dynamic, Manual, Static etc.
					
Initial State: 
					
Input: 
					
Output: 
					
How test will be performed: 

\end{enumerate}

\subsubsection{Module ?}

...

\subsection{Traceability Between Test Cases and Modules}

\wss{Provide evidence that all of the modules have been considered.}
				
\bibliographystyle{plainnat}

\bibliography{../../refs/References}

\newpage

\section{Appendix}

This is where you can place additional information.

\subsection{Symbolic Parameters}

The definition of the test cases will call for SYMBOLIC\_CONSTANTS.
Their values are defined in this section for easy maintenance.

\subsection{Usability Survey Questions?}

\wss{This is a section that would be appropriate for some projects.}

\newpage{}
\section*{Appendix --- Reflection}

\wss{This section is not required for CAS 741}

The information in this section will be used to evaluate the team members on the
graduate attribute of Lifelong Learning.

\input{../Reflection.tex}

\begin{enumerate}
  \item What went well while writing this deliverable? \\
  
  Making the tests were very easy after we had the initial template down and distributed the work effectively amongst ourselves based on the functional and non-functional requirements.
  
  \item What pain points did you experience during this deliverable, and how
    did you resolve them?\\

    One pain point we experienced was that we had a lot to do for this deliverable that we couldn't work on the next one, which is POC. Ultimately, what we decided to do was to focus mainly on this, because the deadline is earlier, and focus mainly on getting our project set up for POC. That way, we at least have started on the next deliverable.

    Another thing was deciding which tests to automate and which to make manual. We resolved this by seeing which test would be easiest or quickest to implement. If a manual test was easiest, we would choose that, but if a manual test would take long, we'd make it automated.\\
    
  \item What knowledge and skills will the team collectively need to acquire to
  successfully complete the verification and validation of your project?
  Examples of possible knowledge and skills include dynamic testing knowledge,
  static testing knowledge, specific tool usage, Valgrind etc.  You should look to
  identify at least one item for each team member.\\

    
  
  \item For each of the knowledge areas and skills identified in the previous
  question, what are at least two approaches to acquiring the knowledge or
  mastering the skill?  Of the identified approaches, which will each team
  member pursue, and why did they make this choice?
\end{enumerate}

\end{document}