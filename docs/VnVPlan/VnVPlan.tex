\documentclass[12pt, titlepage]{article}

\usepackage{booktabs}
\usepackage{tabularx}
\usepackage{hyperref}
\usepackage{comment}
\hypersetup{
    colorlinks,
    citecolor=blue,
    filecolor=black,
    linkcolor=red,
    urlcolor=blue
}
\usepackage[round]{natbib}

\input{../Comments}
%% Common Parts

\newcommand{\progname}{Software Engineering} % PUT YOUR PROGRAM NAME HERE
\newcommand{\authname}{Team \#13, ARC
    \\ Ahluwalia, Avanish
    \\ Davidson, Russell
    \\ Malik, Rafey
    \\ Zulfiqar, Abdul} % AUTHOR NAMES                  

\usepackage{hyperref}
    \hypersetup{colorlinks=true, linkcolor=blue, citecolor=blue, filecolor=blue,
                urlcolor=blue, unicode=false}
    \urlstyle{same}
                                


\begin{document}

\title{System Verification and Validation Plan for \progname{}} 
\author{\authname}
\date{\today}
	
\maketitle

\pagenumbering{roman}

\section*{Revision History}

\begin{tabularx}{\textwidth}{p{3cm}p{2cm}X}
\toprule {\bf Date} & {\bf Version} & {\bf Notes}\\
\midrule
Date 1 & 1.0 & Notes\\
Date 2 & 1.1 & Notes\\
\bottomrule
\end{tabularx}

~\\
\wss{The intention of the VnV plan is to increase confidence in the software.
However, this does not mean listing every verification and validation technique
that has ever been devised.  The VnV plan should also be a \textbf{feasible}
plan. Execution of the plan should be possible with the time and team available.
If the full plan cannot be completed during the time available, it can either be
modified to ``fake it'', or a better solution is to add a section describing
what work has been completed and what work is still planned for the future.}

\wss{The VnV plan is typically started after the requirements stage, but before
the design stage.  This means that the sections related to unit testing cannot
initially be completed.  The sections will be filled in after the design stage
is complete.  the final version of the VnV plan should have all sections filled
in.}

\newpage

\tableofcontents

\listoftables
\wss{Remove this section if it isn't needed}

\listoffigures
\wss{Remove this section if it isn't needed}

\newpage

\section{Symbols, Abbreviations, and Acronyms}

\renewcommand{\arraystretch}{1.2}
\begin{tabular}{l l} 
  \toprule		
  \textbf{symbol} & \textbf{description}\\
  \midrule 
  T & Test\\
  \bottomrule
\end{tabular}\\

\wss{symbols, abbreviations, or acronyms --- you can simply reference the SRS
  \citep{SRS} tables, if appropriate}

\wss{Remove this section if it isn't needed}

\newpage

\pagenumbering{arabic}

This document ... \wss{provide an introductory blurb and roadmap of the
  Verification and Validation plan}

\section{General Information}

\subsection{Summary}

\wss{Say what software is being tested.  Give its name and a brief overview of
  its general functions.}

The software being tested is "Realm", an AR social platform. The platform allows users to place AR objects in the real world for others to see, and allows organizations to create AR tours to provide users high quality experiences.

\subsection{Objectives}

\wss{State what is intended to be accomplished.  The objective will be around
  the qualities that are most important for your project.  You might have
  something like: ``build confidence in the software correctness,''
  ``demonstrate adequate usability.'' etc.  You won't list all of the qualities,
  just those that are most important.}

\wss{You should also list the objectives that are out of scope.  You don't have 
the resources to do everything, so what will you be leaving out.  For instance, 
if you are not going to verify the quality of usability, state this.  It is also 
worthwhile to justify why the objectives are left out.}

\wss{The objectives are important because they highlight that you are aware of 
limitations in your resources for verification and validation.  You can't do everything, 
so what are you going to prioritize?  As an example, if your system depends on an 
external library, you can explicitly state that you will assume that external library 
has already been verified by its implementation team.}

\subsection{Challenge Level and Extras}

\wss{State the challenge level (advanced, general, basic) for your project.
Your challenge level should exactly match what is included in your problem
statement.  This should be the challenge level agreed on between you and the
course instructor.  You can use a pull request to update your challenge level
(in TeamComposition.csv or Repos.csv) if your plan changes as a result of the
VnV planning exercise.}

\wss{Summarize the extras (if any) that were tackled by this project.  Extras
can include usability testing, code walkthroughs, user documentation, formal
proof, GenderMag personas, Design Thinking, etc.  Extras should have already
been approved by the course instructor as included in your problem statement.
You can use a pull request to update your extras (in TeamComposition.csv or
Repos.csv) if your plan changes as a result of the VnV planning exercise.}

\subsection{Relevant Documentation}

\wss{Reference relevant documentation.  This will definitely include your SRS
  and your other project documents (design documents, like MG, MIS, etc).  You
  can include these even before they are written, since by the time the project
  is done, they will be written.  You can create BibTeX entries for your
  documents and within those entries include a hyperlink to the documents.}

\citet{SRS}

\wss{Don't just list the other documents.  You should explain why they are relevant and 
how they relate to your VnV efforts.}

\section{Plan}

\wss{Introduce this section.  You can provide a roadmap of the sections to
  come.}

\subsection{Verification and Validation Team}

\wss{Your teammates.  Maybe your supervisor.
  You should do more than list names.  You should say what each person's role is
  for the project's verification.  A table is a good way to summarize this information.}

\subsection{SRS Verification Plan}

\wss{List any approaches you intend to use for SRS verification.  This may
  include ad hoc feedback from reviewers, like your classmates (like your
  primary reviewer), or you may plan for something more rigorous/systematic.}

\wss{If you have a supervisor for the project, you shouldn't just say they will
read over the SRS.  You should explain your structured approach to the review.
Will you have a meeting?  What will you present?  What questions will you ask?
Will you give them instructions for a task-based inspection?  Will you use your
issue tracker?}

\wss{Maybe create an SRS checklist?}

\subsection{Design Verification Plan}

\wss{Plans for design verification}

\wss{The review will include reviews by your classmates}

\wss{Create a checklists?}

\begin{itemize}
    \item Internal Team Review
    \item Peer Review
    \item TA Review
    \item Requirements Coverage Check
\end{itemize}

\subsection{Verification and Validation Plan Verification Plan}

\wss{The verification and validation plan is an artifact that should also be
verified.  Techniques for this include review and mutation testing.}

\wss{The review will include reviews by your classmates}

\wss{Create a checklists?}

\subsection{Implementation Verification Plan}

\wss{You should at least point to the tests listed in this document and the unit
  testing plan.}

\wss{In this section you would also give any details of any plans for static
  verification of the implementation.  Potential techniques include code
  walkthroughs, code inspection, static analyzers, etc.}

\wss{The final class presentation in CAS 741 could be used as a code
walkthrough.  There is also a possibility of using the final presentation (in
CAS741) for a partial usability survey.}

The primary method of validation for the implementation of the system will be the functional, non-functional, and unit tests described in this document. In addition to the static testing and code review specified in the non-functional tests, we will have a code walkthrough for each module to verify adherence to the design, and the completeness of the implementation. In these walkthroughs we will use our MIS as a checklist and go through module implementations checking off correctly implemented elements.

\subsection{Automated Testing and Verification Tools}

\wss{What tools are you using for automated testing.  Likely a unit testing
  framework and maybe a profiling tool, like ValGrind.  Other possible tools
  include a static analyzer, make, continuous integration tools, test coverage
  tools, etc.  Explain your plans for summarizing code coverage metrics.
  Linters are another important class of tools.  For the programming language
  you select, you should look at the available linters.  There may also be tools
  that verify that coding standards have been respected, like flake9 for
  Python.}

\wss{If you have already done this in the development plan, you can point to
that document.}

\wss{The details of this section will likely evolve as you get closer to the
  implementation.}

\subsection{Software Validation Plan}

\wss{If there is any external data that can be used for validation, you should
  point to it here.  If there are no plans for validation, you should state that
  here.}

\wss{You might want to use review sessions with the stakeholder to check that
the requirements document captures the right requirements.  Maybe task based
inspection?}

\wss{For those capstone teams with an external supervisor, the Rev 0 demo should 
be used as an opportunity to validate the requirements.  You should plan on 
demonstrating your project to your supervisor shortly after the scheduled Rev 0 demo.  
The feedback from your supervisor will be very useful for improving your project.}

\wss{For teams without an external supervisor, user testing can serve the same purpose 
as a Rev 0 demo for the supervisor.}

\wss{This section might reference back to the SRS verification section.}

\section{System Tests}

\wss{There should be text between all headings, even if it is just a roadmap of
the contents of the subsections.}

\subsection{Tests for Functional Requirements}

\wss{Subsets of the tests may be in related, so this section is divided into
  different areas.  If there are no identifiable subsets for the tests, this
  level of document structure can be removed.}

\wss{Include a blurb here to explain why the subsections below
  cover the requirements.  References to the SRS would be good here.}

\subsubsection{Area of Testing1}

\wss{It would be nice to have a blurb here to explain why the subsections below
  cover the requirements.  References to the SRS would be good here.  If a section
  covers tests for input constraints, you should reference the data constraints
  table in the SRS.}
		
\paragraph{Title for Test}

\begin{enumerate}

\item{test-id1\\}

Control: Manual versus Automatic
					
Initial State: 
					
Input: 
					
Output: \wss{The expected result for the given inputs.  Output is not how you
are going to return the results of the test.  The output is the expected
result.}

Test Case Derivation: \wss{Justify the expected value given in the Output field}
					
How test will be performed: 
					
\item{test-id2\\}

Control: Manual versus Automatic
					
Initial State: 
					
Input: 
					
Output: \wss{The expected result for the given inputs}

Test Case Derivation: \wss{Justify the expected value given in the Output field}

How test will be performed: 

\end{enumerate}

\subsubsection{Realm Testing}

\begin{enumerate}


\item \textbf{Name:} Validate AR Object Perspective Adjustment \label{itm:Test-RI1} \\
\textbf{Test ID:} Test-RI1 \\
\textbf{Type:} Functional, Manual \\
\textbf{Initial State:} Realm interface is open, AR objects are displayed in the user’s vicinity on the camera feed. \\
\textbf{Input/Condition:} Tester changes their position and angle in relation to an AR object. \\
\textbf{Output/Result:} The AR object adjusts perspective appropriately, reflecting the new camera position and angle. \\ \textbf{How test will be performed:} The tester will move around the AR object, observing whether it adjusts in real time and maintains correct perspective relative to the user’s view.

\item \textbf{Name:} Validate AR Object Clutter Management \label{itm:Test-RI2} \\
\textbf{Test ID:} Test-RI2 \\
\textbf{Type:} Functional, Manual \\
\textbf{Initial State:} Realm interface is open, tester is in an area with overlapping AR object instances \\
\textbf{Input/Condition:} Tester moves camera over a crowded area where multiple AR objects are present. \\
\textbf{Output/Result:} The interface selectively displays a manageable number of AR objects without overwhelming the user’s view. \\ \textbf{How test will be performed:} Tester observes the interface to confirm only a few objects are displayed at once, reducing clutter in the view.

\item \textbf{Name:} Validate AR Object Placement Accuracy \label{itm:Test-RI3} \\
\textbf{Test ID:} Test-RI3 \\
\textbf{Type:} Functional, Automated and Manual \\
\textbf{Initial State:} Realm interface is open, test AR object instance is nearby \\
\textbf{Input/Condition:} Test AR object instance is placed with a known alignment in the real world, and reference screenshots \\ 
\textbf{Output/Result:} Test AR object appears in correct position and orientation as expected from the known alignment and reference screenshots, position also matches stored object instance data \\
\textbf{How test will be performed:} Automated tests will compare object instance data, while a manual test will involve the tester visually confirming the position accuracy in the camera view.


\item \textbf{Name:} Validate Sub-Realm Selection Indicator \label{itm:Test-RI4} \\
\textbf{Test ID:} Test-RI4 \\
\textbf{Type:} Functional, Manual \\
\textbf{Initial State:} Realm interface is open with a sub-realm selected. \\
\textbf{Input/Condition:} Tester views the interface with a sub-realm selected. \\
\textbf{Output/Result:} The current sub-realm is clearly indicated on the interface. \\
\textbf{How test will be performed:} Tester observes the interface to ensure that the active sub-realm is always visually indicated.

\item \textbf{Name:} Validate Sub-Realm Selection Change \label{itm:Test-RI5} \\
\textbf{Test ID:} Test-RI5 \\
\textbf{Type:} Functional, Manual \\
\textbf{Initial State:} Realm interface is open with a sub-realm selected, tester is in area with object instances from multiple sub-realms \\
\textbf{Input/Condition:} Tester attempts to change the sub-realm using the interface controls. \\
\textbf{Output/Result:} The interface updates to display the newly selected sub-realm, and the correct object instances appear. \\ \textbf{How test will be performed:} Tester selects a new sub-realm and observes if the displayed objects and interface reflect the new selection.


\item \textbf{Name:} Validate Object Placement Workflow Control \label{itm:Test-RI6} \\
\textbf{Test ID:} Test-RI6 \\
\textbf{Type:} Functional, Manual \\
\textbf{Initial State:} Realm interface is open. \\
\textbf{Input/Condition:} Tester attempts to access the object placement workflow via the provided control. \\
\textbf{Output/Result:} Tester is successfully redirected to the object placement workflow. \\
\textbf{How test will be performed:} Tester selects the object placement control and verifies redirection to the appropriate workflow screen.

\item \textbf{Name:} Validate Object Scanning Workflow Control \label{itm:Test-RI7} \\
\textbf{Test ID:} Test-RI7 \\
\textbf{Type:} Functional, Manual \\
\textbf{Initial State:} Realm interface is open. \\
\textbf{Input/Condition:} Tester attempts to access the object scanning workflow via the provided control. \\
\textbf{Output/Result:} Tester is successfully redirected to the object scanning workflow. \\
\textbf{How test will be performed:} Tester selects the object scanning control and verifies redirection to the appropriate workflow screen.


\item \textbf{Name:} Validate Nearby Tour Indication \label{itm:Test-RI8} \\
\textbf{Test ID:} Test-RI8 \\
\textbf{Type:} Functional, Manual \\
\textbf{Initial State:} Realm interface is open, tester is near the starting point of a tour. \\
\textbf{Input/Condition:} Tester moves within range of the tour start point. \\
\textbf{Output/Result:} The interface displays a clear indication of the nearby tour and a link to the tour preview. \\
\textbf{How test will be performed:} Tester observes if the indication and link appear when near the tour start point.

\item \textbf{Name:} Validate Hazard Warning \label{itm:Test-RI9} \\
\textbf{Test ID:} Test-RI9 \\
\textbf{Type:} Functional, Manual \\
\textbf{Initial State:} Realm interface is open, tester is approaching a real-world hazard. \\
\textbf{Input/Condition:} Tester moves closer to a hazard in real space. \\
\textbf{Output/Result:} Interface displays a clear warning when user approaches the hazard. \\
\textbf{How test will be performed:} Tester approaches a wall with the realm interface open, and verifies that a warning appears. 


\item \textbf{Name:} Validate Offline Mode for Interactive Components \label{itm:Test-RI10} \\
\textbf{Test ID:} Test-RI10 \\
\textbf{Type:} Functional, Manual \\
\textbf{Initial State:} Realm interface is open and disconnected from the internet. \\
\textbf{Input/Condition:} Tester attempts to interact with various components of the interface in offline mode. \\
\textbf{Output/Result:} Interactive components function normally, but location-based features are disabled. \\
\textbf{How test will be performed:} Tester verifies the functionality of object scanning and the unavailability of object placement, maps, and other internet dependent features.

\end{enumerate}

\subsubsection{Object Placement Testing}

\begin{enumerate}


\item \textbf{Name:} Validate Object Selection Stage \label{itm:Test-OP1} \\
\textbf{Test ID:} Test-OP1 \\
\textbf{Type:} Functional, Manual \\
\textbf{Initial State:} Tester has progressed object placement workflow to object selection stage \\
\textbf{Input/Condition:} Tester selects object from one of: inventory, new object scan, new prompt generation \\
\textbf{Output/Result:} Interface successfully proceeds to sub-realm selection step with the selected object \\
\textbf{How test will be performed:} Tester initiates object placement workflow and progresses to object selection step. They then use one of the object selection methods to select an object and validate that the interface moves on to sub-realm selection with the selected object. They select the option to return to object selection, and repeat the process for all  object selection methods.

\item \textbf{Name:} Validate Sub-Realm Selection Stage \label{itm:Test-OP2} \\
\textbf{Test ID:} Test-OP2 \\
\textbf{Type:} Functional, Manual \\
\textbf{Initial State:} Tester has progressed object placement workflow to sub-realm selection stage, and is part of multiple sub-realms \\
\textbf{Input/Condition:} Tester selects multiple sub-realms for sharing the object instance. \\
\textbf{Output/Result:} The system associates the selected sub-realms with the object instance upon placement. \\
\textbf{How test will be performed:} Tester initiates object placement workflow and progresses to sub-realm selection step. They then select multiple sub-realms before placing the object, verifying that the correct sub-realms are associated by checking the object instance database entries.

\item \textbf{Name:} Validate Object Placement Stage \label{itm:Test-OP3} \\
\textbf{Test ID:} Test-OP3 \\
\textbf{Type:} Functional, Manual \\
\textbf{Initial State:} Tester has progressed object placement workflow object placement stage \\
\textbf{Input/Condition:} Tester rotates, resizes, and translates the object in real space, then confirms placement. \\
\textbf{Output/Result:} Rotation, resizing, and translating are all functional, and the AR object is positioned accurately in real space with the correct orientation, aligning with the tester's intent. \\
\textbf{How test will be performed:} Tester initiates object placement workflow and progresses to object placement step. Tester rotates, resizes, and translates the object, verifying visually that it appears in the intended location and orientation in the real-world view, and finally confirms the placement of the object.

\item \textbf{Name:} Validate Object Instance Storage \label{itm:Test-OP4} \\
\textbf{Test ID:} Test-OP4 \\
\textbf{Type:} Functional, Manual \\
\textbf{Initial State:} Tester has just placed an AR object instance \\
\textbf{Input/Condition:} Tester checks the AR object instance database \\
\textbf{Output/Result:} The AR object instance that the tester placed is present with correct details, including object type, sub-realm(s), position, and orientation. \\
\textbf{How test will be performed:} Tester completes the object placement workflow then checks the AR object instance database and verifies the presence and correctness of their newly created object instance.


\item \textbf{Name:} Validate Area Based Placement Limit \label{itm:Test-OP5} \\
\textbf{Test ID:} Test-OP5 \\
\textbf{Type:} Functional, Automated and Manual \\
\textbf{Initial State:} Test user has sufficient AR object instances recorded to reach the object placement limit of an area  \\
\textbf{Input/Condition:} Tester attempts to place another object in the same area \\
\textbf{Output/Result:} System prevents additional placements once the area limit is reached, displaying a relevant warning. \\
\textbf{How test will be performed:} Automated test script creates AR object instance entries in database to reach limit. Tester manually attempts to place another object in selected area, and validates that they are prevented from doing so, and are presented with a warning.

\item \textbf{Name:} Validate Time Based Placement Limit \label{itm:Test-OP6} \\
\textbf{Test ID:} Test-OP6 \\
\textbf{Type:} Functional, Automated and Manual \\
\textbf{Initial State:} Test user has sufficient AR object instances recorded to reach the time-based object placement limit for a user \\
\textbf{Input/Condition:} Tester attempts to place another object within a short period \\
\textbf{Output/Result:} System restricts further placements once the time-based limit is reached, displaying a relevant warning. \\
\textbf{How test will be performed:} Automated test script creates AR object instance entries in database to reach limit. Tester manually attempts to place another object within a short period, and validates that they are prevented from doing so, and are presented with a warning.

\item \textbf{Name:} Validate Automated Retry for Failed Object Storage \label{itm:Test-OP7} \\
\textbf{Test ID:} Test-OP7 \\
\textbf{Type:} Functional, Automated \\
\textbf{Initial State:} System is running, with simulated conditions preventing initial object storage (e.g., network issues). \\
\textbf{Input/Condition:} User places an object, but initial storage attempt fails due to simulated conditions. \\
\textbf{Output/Result:} System automatically retries object storage until success or retry limit reached. \\
\textbf{How test will be performed:} Simulate a storage failure on the initial attempt, monitoring logs to confirm retry attempts are made until storage is successful, validating the success scenario. Repeat this process, but simulate continued storage failure, and verify that the number of retries attempted is equal to the retry limit.

\end{enumerate}

\subsubsection{Database Testing}

\begin{enumerate}

\item \textbf{Name:} Validate Periodic Database Backup \label{itm:Test-DB1} \\
    \textbf{Test ID:} Test-DB1 \\
    \textbf{Type:} Functional, Automated \\
    \textbf{Initial State:} System is running, database is available, periodic backup is set up. \\
    \textbf{Input/Condition:} Periodic backup run is completed.  \\
    \textbf{Output/Result:} Automated monitor verifies that the database backup is present and correct.  \\
    \textbf{How test will be performed:} An automated monitor will wait for periodic backups, then restore the database from the backup in a sandbox environment and check that the data is present as expected.

\item \textbf{Name:} Validate Database Encryption \label{itm:Test-DB2} \\
    \textbf{Test ID:} Test-DB2 \\
    \textbf{Type:} Functional, Automated \\
    \textbf{Initial State:} System is running, database is available. \\
    \textbf{Input/Condition:} Command to check encryption status is inputted into DBMS for all databases \\
    \textbf{Output/Result:} DBMS response shows that all databases are encrypted \\
    \textbf{How test will be performed:} Will depend on the database platform used, for example on SQL Server the following query would be ran, and the output would be checked: \\
    SELECT db\_name(database\_id), encryption\_state \\
    FROM sys.dm\_database\_encryption\_keys;

\end{enumerate}


\subsection{Tests for Nonfunctional Requirements}

\wss{The nonfunctional requirements for accuracy will likely just reference the
  appropriate functional tests from above.  The test cases should mention
  reporting the relative error for these tests.  Not all projects will
  necessarily have nonfunctional requirements related to accuracy.}

\wss{For some nonfunctional tests, you won't be setting a target threshold for
passing the test, but rather describing the experiment you will do to measure
the quality for different inputs.  For instance, you could measure speed versus
the problem size.  The output of the test isn't pass/fail, but rather a summary
table or graph.}

\wss{Tests related to usability could include conducting a usability test and
  survey.  The survey will be in the Appendix.}

\wss{Static tests, review, inspections, and walkthroughs, will not follow the
format for the tests given below.}

\wss{If you introduce static tests in your plan, you need to provide details.
How will they be done?  In cases like code (or document) walkthroughs, who will
be involved? Be specific.}

\subsubsection{Usability Testing}



\begin{enumerate}

    \item \textbf{Name:} Validate Localization \label{itm:Test-QS-U1} \\
    \textbf{Test ID:} Test-QS-U1 \\
    \textbf{Type:} Non-Functional, Manual \\
    \textbf{Initial State:} App is open on settings page. \\
    \textbf{Input/Condition:} Language setting is changed to English, Mandarin Chinese, Hindi, Spanish, and French (The 5 most spoken languages in the world) in turn.  \\
    \textbf{Output/Result:} Text in the app correctly changes to the selected language, with understandable translations. \\
    \textbf{How test will be performed:} Tester navigates to settings menu, selects one of the languages to be tested, and verifies that text in the settings page, realm interface, and maps interface are all correctly displayed in the selected language

    \item \textbf{Name:} Validate User Intuitiveness and Satisfaction \label{itm:Test-QS-U2} \\
    \textbf{Test ID:} Test-QS-U2 \\
    \textbf{Type:} Non-Functional, Manual \\
    \textbf{Initial State:} App is open, and tester is logged in as a new user with no prior experience using the app. \\
    \textbf{Input/Condition:} Tester performs common workflows such as account setup, navigating between interfaces, creating AR objects, viewing and placing AR objects, and finding objects on the map, all without guidance or assistance  \\
    \textbf{Output/Result:} 80\% of testers complete each task and report that the app is easy and satisfying to use, and rate each workflow as highly intuitive.\\
    \textbf{How test will be performed:} A group of new users will perform specified workflows and complete a post-test \hyperref[sub:usability_survey]{survey} rating the intuitiveness and satisfaction of their experience. Quantitative results should show that most users rate the app as highly intuitive and satisfying to use.

\end{enumerate}
\subsubsection{Availability Testing}

\begin{enumerate}
    
\item \textbf{Name:} Validate Server Availability \label{itm:Test-QS-A1} \\
    \textbf{Test ID:} Test-QS-A1 \\
    \textbf{Type:} Non-Functional, Automated \\
    \textbf{Initial State:} Server is running, with monitoring tools actively tracking uptime. \\
    \textbf{Input/Condition:} Automated monitoring scripts track server uptime and downtime continuously over a one-week period \\
    \textbf{Output/Result:} Monitoring scripts log any downtime, with server uptime recorded at 99\% or higher over the test period. \\
    \textbf{How test will be performed:} Monitoring scripts will check server availability at regular intervals and log any downtime events, ensuring the server meets the 99\% availability criteria.


\item \textbf{Name:} Validate User Feedback on Server Availability \label{itm:Test-QS-A2} \\
    \textbf{Test ID:} Test-QS-A2 \\
    \textbf{Type:} Non-Functional, Manual \\
    \textbf{Initial State:} Server is running, and a test group of users has been granted access to the app. \\
    \textbf{Input/Condition:} Over a one-week period, users in the test group access the app multiple times per day, as they normally would, at varying times. \\
    \textbf{Output/Result:} No user complaints about server unavailability during the test period. Users report no issues with app access. \\
    \textbf{How test will be performed:} Test group participants will be \hyperref[sub:usability_survey]{surveyed} at the end of the test period regarding any access issues they encountered. Any user-reported issues will be logged and reviewed to assess the server's availability from the user’s perspective.

\end{enumerate}
\subsubsection{Maintainability Testing}

\begin{enumerate}
    
\item \textbf{Name:} Validate API Error Message Clarity \label{itm:Test-DI-M1} \\
    \textbf{Test ID:} Test-DI-M1 \\
    \textbf{Type:} Non-Functional, Manual and Automated \\
    \textbf{Initial State:} System is running, with logging enabled for internal API calls. \\
    \textbf{Input/Condition:} Simulate the following types of system failures in internal APIs and observe the resulting error messages: database connection failure, invalid input data, service timeout. \\
    \textbf{Output/Result:} Error messages generated by the APIs clearly indicate the source and nature of the error in at least 90\% of cases, helping developers quickly identify issues. \\
    \textbf{How test will be performed:} Automated scripts will be used to simulate common errors and log the resulting API responses. The resulting error messages will be manually reviewed to be evaluated on detail (e.g., error type, location, and possible causes), and clarity (accurate indication of what is causing error). Success is achieved if 90\% of the messages help to identify the error source.


\end{enumerate}

\subsection{Traceability Between Test Cases and Requirements}

\wss{Provide a table that shows which test cases are supporting which
  requirements.}

\begin{table}[h!]
    \centering
    \begin{tabular}{|l|l|l|}
        \hline
        \textbf{Test-ID} & \textbf{Test Name} &\textbf{Requirements} \\
        \hline
        \hyperref[itm:Test-RI1]{Test-RI1} & Validate AR Object Perspective Adjustment & RI-FR1.1 \\
        \hline
        \hyperref[itm:Test-RI2]{Test-RI2} & Validate AR Object Clutter Management & RI-FR1.2 \\
        \hline
        \hyperref[itm:Test-RI3]{Test-RI3} & Validate AR Object Placement Accuracy & RI-FR1.2 \\
        \hline
        \hyperref[itm:Test-RI4]{Test-RI4} & Validate Sub-Realm Selection Indicator & RI-FR2.1 \\
        \hline
        \hyperref[itm:Test-RI5]{Test-RI5} & Validate Sub-Realm Selection Change & RI-FR2.2 \\
        \hline
        \hyperref[itm:Test-RI6]{Test-RI6} & Validate Object Placement Workflow Control & RI-FR3 \\
        \hline
        \hyperref[itm:Test-RI7]{Test-RI7} & Validate Object Scanning Workflow Control & RI-FR4 \\
        \hline
        \hyperref[itm:Test-RI8]{Test-RI8} & Validate Nearby Tour Indication & RI-FR5 \\
        \hline
        \hyperref[itm:Test-RI9]{Test-RI9} & Validate Hazard Warning & RI-FR6 \\
        \hline
        \hyperref[itm:Test-RI10]{Test-RI10} & Validate Offline Mode for Interactive Components & RI-FR7 \\
        \hline
        \hyperref[itm:Test-OP1]{Test-OP1} & Validate Object Selection Stage & OP-FR2.1 \\
        \hline
        \hyperref[itm:Test-OP2]{Test-OP2} & Validate Sub-Realm Selection Stage & OP-FR2.2 \\
        \hline
        \hyperref[itm:Test-OP3]{Test-OP3} & Validate Object Placement Stage & OP-FR2.3 \\
        \hline
        \hyperref[itm:Test-OP4]{Test-OP4} & Validate Object Instance Storage & OP-FR1 \\
        \hline
        \hyperref[itm:Test-OP5]{Test-OP5} & Validate Area Based Placement Limit & OP-FR3.1 \\
        \hline
        \hyperref[itm:Test-OP6]{Test-OP6} & Validate Time Based Placement Limit & OP-FR3.2 \\
        \hline
        \hyperref[itm:Test-OP7]{Test-OP7} & Validate Automated Retry for Failed Object Storage & OP-FR1 \\
        \hline
        \hyperref[itm:Test-DB1]{Test-DB1} & Validate Periodic Database Backup & DB-FR1 \\
        \hline
        \hyperref[itm:Test-DB2]{Test-DB2} & Validate Database Encryption & DB-FR2 \\
        \hline
        \hyperref[itm:Test-QS-U1]{Test-QS-U1} & Validate Localization & QS-U1 \\
        \hline
        \hyperref[itm:Test-QS-U2]{Test-QS-U2} & Validate User Intuitiveness and Satisfaction & QS-U2 \\
        \hline
        \hyperref[itm:Test-QS-A1]{Test-QS-A1} & Automated Server Availability Monitoring & QS-A1 \\
        \hline
        \hyperref[itm:Test-QS-A2]{Test-QS-A2} & User Feedback on Server Availability & QS-A1 \\
        \hline
        \hyperref[itm:Test-DI-M1]{Test-DI-M1} & Validate API Error Message Clarity & DI-M1 \\
        \hline
    \end{tabular}
    \caption{Mapping of Tests to Requirements}
    \label{tab:test_requirements}
\end{table}

\pagebreak
  

\section{Unit Test Description}

This section will contain descriptions of the unit tests derived from the MIS, which will be completed in a later iteration of this document

\begin{comment}
\wss{This section should not be filled in until after the MIS (detailed design
  document) has been completed.}

\wss{Reference your MIS (detailed design document) and explain your overall
philosophy for test case selection.}  

\wss{To save space and time, it may be an option to provide less detail in this section.  
For the unit tests you can potentially layout your testing strategy here.  That is, you 
can explain how tests will be selected for each module.  For instance, your test building 
approach could be test cases for each access program, including one test for normal behaviour 
and as many tests as needed for edge cases.  Rather than create the details of the input 
and output here, you could point to the unit testing code.  For this to work, you code 
needs to be well-documented, with meaningful names for all of the tests.}

\subsection{Unit Testing Scope}

\wss{What modules are outside of the scope.  If there are modules that are
  developed by someone else, then you would say here if you aren't planning on
  verifying them.  There may also be modules that are part of your software, but
  have a lower priority for verification than others.  If this is the case,
  explain your rationale for the ranking of module importance.}

\subsection{Tests for Functional Requirements}

\wss{Most of the verification will be through automated unit testing.  If
  appropriate specific modules can be verified by a non-testing based
  technique.  That can also be documented in this section.}

\subsubsection{Module 1}

\wss{Include a blurb here to explain why the subsections below cover the module.
  References to the MIS would be good.  You will want tests from a black box
  perspective and from a white box perspective.  Explain to the reader how the
  tests were selected.}

\begin{enumerate}

\item{test-id1\\}

Type: \wss{Functional, Dynamic, Manual, Automatic, Static etc. Most will
  be automatic}
					
Initial State: 
					
Input: 
					
Output: \wss{The expected result for the given inputs}

Test Case Derivation: \wss{Justify the expected value given in the Output field}

How test will be performed: 
					
\item{test-id2\\}

Type: \wss{Functional, Dynamic, Manual, Automatic, Static etc. Most will
  be automatic}
					
Initial State: 
					
Input: 
					
Output: \wss{The expected result for the given inputs}

Test Case Derivation: \wss{Justify the expected value given in the Output field}

How test will be performed: 

\item{...\\}
    
\end{enumerate}

\subsubsection{Module 2}

...

\subsection{Tests for Nonfunctional Requirements}

\wss{If there is a module that needs to be independently assessed for
  performance, those test cases can go here.  In some projects, planning for
  nonfunctional tests of units will not be that relevant.}

\wss{These tests may involve collecting performance data from previously
  mentioned functional tests.}

\subsubsection{Module ?}
		
\begin{enumerate}

\item{test-id1\\}

Type: \wss{Functional, Dynamic, Manual, Automatic, Static etc. Most will
  be automatic}
					
Initial State: 
					
Input/Condition: 
					
Output/Result: 
					
How test will be performed: 
					
\item{test-id2\\}

Type: Functional, Dynamic, Manual, Static etc.
					
Initial State: 
					
Input: 
					
Output: 
					
How test will be performed: 

\end{enumerate}

\subsubsection{Module ?}

...

\subsection{Traceability Between Test Cases and Modules}

\wss{Provide evidence that all of the modules have been considered.}
				
\bibliographystyle{plainnat}

\bibliography{../../refs/References}

\end{comment}

\newpage

\section{Appendix}

This is where you can place additional information.

\subsection{Symbolic Parameters}

The definition of the test cases will call for SYMBOLIC\_CONSTANTS.
Their values are defined in this section for easy maintenance.

\subsection{Usability Survey Questions} \label{sub:usability_survey}

\begin{itemize}
    \item For the following statements, please indicate your level of agreement between Strongly Disagree, Disagree, Neither Agree nor Disagree, Agree, and Strongly Agree:
    \begin{itemize}
        \item Navigating between interfaces is intuitive
        \item Placing objects is intuitive
        \item Scanning objects is intuitive
        \item Generating objects by prompt is intuitive
        \item Embarking on tours is intuitive
        \item Creating tours is intuitive (ONLY FOR ORG USERS)
        \item Managing tours is intuitive (ONLY FOR ORG USERS)
        \item Changing user settings is intuitive
        \item Interacting with other's objects is intuitive
        \item Reporting other's objects is intuitive
        \item Adding friends, and creating and managing sub-realms is intuitive
        \item Using the app is generally satisfying
    \end{itemize}
    \item Did you experience any service interruptions, including but not limited to excessive load times for elements like navigation and object placement, while testing the app? \\
\end{itemize}



\newpage{}
\section*{Appendix --- Reflection}

\wss{This section is not required for CAS 741}

The information in this section will be used to evaluate the team members on the
graduate attribute of Lifelong Learning.

\input{../Reflection.tex}

\begin{enumerate}
  \item What went well while writing this deliverable? 
  \item What pain points did you experience during this deliverable, and how
    did you resolve them?
  \item What knowledge and skills will the team collectively need to acquire to
  successfully complete the verification and validation of your project?
  Examples of possible knowledge and skills include dynamic testing knowledge,
  static testing knowledge, specific tool usage, Valgrind etc.  You should look to
  identify at least one item for each team member.
  \item For each of the knowledge areas and skills identified in the previous
  question, what are at least two approaches to acquiring the knowledge or
  mastering the skill?  Of the identified approaches, which will each team
  member pursue, and why did they make this choice?
\end{enumerate}

\end{document}